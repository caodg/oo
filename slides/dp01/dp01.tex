\documentclass[compress]{beamer}

\usepackage[nofonts]{ctex}
\setCJKmainfont[ItalicFont={Kaiti SC}]{Kaiti SC}%
%\setCJKmainfont[ItalicFont={AR PL KaitiM GB}]{AR PL KaitiM GB}%
%\setCJKsansfont{WenQuanYi Zen Hei}% 文泉驿的黑体

\mode<beamer>
{
     \useinnertheme{rectangles}
     %\useoutertheme{infolines}
     %\useoutertheme{split}
     \usecolortheme{rose}
     \usecolortheme{seahorse}

     \setbeamertemplate{navigation symbols}{}%remove navigation symbols

%     \expandafter\def\expandafter\insertshorttitle\expandafter{%
%     \insertshorttitle\hfill%
%     \insertframenumber\,/\,\inserttotalframenumber}
}

\defbeamertemplate*{footline}{mytheme}
{
  \leavevmode%
  \hbox{%
  \begin{beamercolorbox}[wd=1.0\paperwidth,ht=2.25ex,dp=1ex,center]{author in head/foot}%
    \raisebox{-1ex}{\includegraphics[width=3ex]{Overlays/logo.pdf}}%
    \hspace*{6ex}\insertframenumber{} / \inserttotalframenumber
  \end{beamercolorbox}}%
  \vskip0pt%
}
\usebeamertemplate{mytheme}

\setbeamertemplate{headline}{}

%\setbeamercovered{transparent}

\mode<handout>
{
	\usetheme{default}
	\usepackage{pgfpages}
	\pgfpagesuselayout{2 on 1}[a4paper,landscape,border shrink=5mm]
}


\usepackage{amsmath,latexsym,amssymb,amsfonts,amsbsy}
\usepackage{graphicx}
\usepackage{array}
\usepackage{hyperref}
\usepackage{textpos}
\usepackage{ulem}
\usepackage{comment}
\usepackage{fancyvrb}
\fvset{frame=single, numbers=left, fontsize=\small}
\usepackage{tikz}
\usetikzlibrary{calc,arrows.meta, graphs, trees, shapes, positioning, decorations.markings, intersections, decorations.text}
\usepackage{tikz-uml}

\newcommand{\romannumber}[1]{{\textrm{\uppercase\expandafter{\romannumeral
#1}}}}

\setbeamercolor{dblue}{fg=white,bg=gray!70!blue} % for beamercolorbox
 \newenvironment{pblock}{\begin{beamercolorbox}[rounded=true,
          shadow=false]{dblue}}{\end{beamercolorbox}}
 \newenvironment{bblock}{\begin{beamercolorbox}[rounded=true,
          shadow=false]{fg=black, bg=white}}{\end{beamercolorbox}}

\graphicspath{{figure/}}

%%%%%%%%%%%%%%%%%%%%%%%%%%%%%%%%%%%%%%%%%%%%%%%%%%%%%%%%%%%%%%%%%
%    body                                                       %
%%%%%%%%%%%%%%%%%%%%%%%%%%%%%%%%%%%%%%%%%%%%%%%%%%%%%%%%%%%%%%%%%


\begin{document}

%\AtBeginSubsection[]
%\AtBeginSection[]
%{ 
%    \begin{frame}<beamer> 
%		\frametitle{内容提要} 
%		\tableofcontents[currentsection,currentsubsection] 
%	\end{frame} 
%} 

					
\title[观察者模式]{面向对象设计模式 ~~ 观察者}

\author[曹东刚]
{曹东刚\\\href{mailto:caodg@pku.edu.cn}{caodg@pku.edu.cn}}

\institute[北京大学]{北京大学信息学院研究生课程 - 面向对象的分析与设计 \\
    \href{http://sei.pku.edu.cn/~caodg/course/oo}{http://sei.pku.edu.cn/\~{}caodg/course/oo}}

\date{}

\titlegraphic{\includegraphics[height=0.10\textwidth]{Overlays/logo.pdf}}

\begin{frame}[plain]
	\titlepage
\end{frame}

\setcounter{framenumber}{0}

%\includeonlyframes{current}

\begin{frame}
  \frametitle{一个应用: 建造下一代Internet气象观察站}

  \textcolor{blue}{已有基础:}
  \begin{itemize}
    \item 一个已开发的WeatherData对象,负责追踪目前的天气状况(温度、湿
      度、气压)
  \end{itemize}

  \textcolor{blue}{要求开发一个气象观察应用:}
  \begin{itemize}
    \item 有三种布告板,分别显示目前的状况、气象统计及简单
      的预报,当WeatherData 对象获得最新的气象数据时,三种布告板必须实时
      更新
    \item 气象站必须可扩展,能公开一组API,允许其他开发人员开发出自己的
      气象布告板,并插入此应用中
  \end{itemize}

\end{frame}

\begin{frame}
  \frametitle{气象观察应用的概况}

  \noindent\resizebox{\hsize}{!}{
  \begin{tikzpicture}[node distance=1cm]
    \node (station) [rectangle, draw, minimum width=1.5cm, minimum height=2cm] 
    {气象站} ;

    \node (temp) [circle, minimum size=1cm, left= 2cm of station, draw] {} ;
    \node [below=0.2cm of temp] {温度感应装置} ;

    \node (humid) [circle, minimum size=1cm, above left=2cm of station, draw] {} ;
    \node [below=0.2cm of humid, xshift=-0.5cm] {湿度感应装置} ;

    \node (pres) [circle, minimum size=1cm, below left=2cm of station, draw] {} ;
    \node [below=0.2cm of pres, xshift=-0.5cm] {气压感应装置} ;

    \draw (station) -- (temp) ;
    \draw (station) -- (humid) ;
    \draw (station) -- (pres) ;

    \node (wdata) [circle, minimum size=1.5cm, right= 3cm of station,
    draw, fill=green!20] {} ;
    \node [below=0.2cm of wdata] {WeatherData对象} ;

    \draw [-Stealth] (wdata) -> (station.east) node [anchor=south west]{取得
    数据};

    \node (board) [rectangle, rounded corners,  minimum width=1.5cm,
      minimum height=2cm, right= 2cm of wdata, draw, fill=green!20] {} ;
    \node [below=0.2cm of board] {气象布告板} ;

    \draw [-Stealth] (wdata) -> (board.west) node [anchor=south east]{显
    示};

    \draw [dashed] ([xshift=-1cm, yshift=3cm]wdata.west) --
    ([xshift=-1cm, yshift=-3cm]wdata.west) ;

    \node [below=3cm of wdata.center, font=\Large] {待开发应用} ;
    \node [below=3cm of station.center, font=\Large] {已有基础} ;

  \end{tikzpicture}
  }

\end{frame}

{
  \defverbatim{\verbmchanged}{%
\footnotesize
\begin{verbatim}
/*  
 * Will be called when measurements
 * changed 
 */ 

public void measurementsChanged() { 
   //Write your code here 
}
\end{verbatim}
}

\begin{frame}
  \frametitle{WeatherData类}

  \noindent\begin{tikzpicture}
    \tikzumlset{fill class=white}

    \umlclass{WeatherData}{}{getTemperature() \\
      getHumidity() \\
      getPressure() \\
      measurementsChanged()
    }

    \umlnote[x=6, width=5.8cm, anchor2=-50]{WeatherData} {\verbmchanged }
  \end{tikzpicture}
\end{frame}
}

\begin{frame}
  \frametitle{我们知道的信息}
  \begin{itemize}
    \item  WeatherData类的getter方法可以取得温度、湿度、气压的值
    \item 当有新数据的时候,measurementsChanged方法就会被调用
      \begin{itemize}
        \item 我们不关心它如何被调用,只知道它一定会被调用
      \end{itemize}
    \item 我们需要实现三个布告板:目前状况布告、气象统计布告、天气预报布
      告,这些布告必须能够\uline{实时更新}
    \item 系统必须允许在\uline{运行时添加}新的布告板
  \end{itemize}
\end{frame}

\begin{frame}[fragile]
  \frametitle{常见做法及问题分析}

\begin{Verbatim}[label=一种常见的做法]
public void WeatherData {
  public void measurementsChanged() {
    float temp = getTemerature() ;
    float humidity = getHumidity() ;
    float pressure = getPressure() ;

    currentConditionsDisplay.update(
        temp, humidity, pressure) ;
    statisticsDisplay.update(temp, humidity, pressure) ;
    forecastDisplay.update(temp, humidity, pressure) ;
  }
}
\end{Verbatim}
\end{frame}

\begin{frame}
  \frametitle{观察者模式: 从报纸订阅开始}
  \begin{enumerate}
    \item 报社的业务是\uline{出版}报纸
    \item 向某家报社\uline{订阅}报纸,以后只要他们有新报纸出版,就会给你送来
    \item 当你不想再看报的时候,可以\uline{取消订阅},他们就不会送报给你
    \item 只要报社在运营,就会一直有人向他们订阅或取消订阅报纸
  \end{enumerate}
\end{frame}

\begin{frame}
  \frametitle{出版者 $+$ 订阅者 $=$ 观察者模式 }

  \only<1> {
  \begin{tikzpicture}

    \node {2} ;
    \node (Subject) [circle, draw, minimum size=1cm] {};
    \path [postaction={decoration={text along path, text={主 题 对 象}, 
      text align={left indent=3.3cm},
    }, decorate}] (0, 0) circle (0.9cm) ;

    \node (Cat) [circle, draw, minimum size=1cm, right=6cm of Subject] {} ;
    \node [below=0.2cm of Cat] {猫对象} ;

    \node (Dog) [circle, draw, minimum size=1cm, above left=1cm of Cat] {} ;
    \node [below=0.2cm of Dog] {狗对象} ;

    \node (Mice) [circle, draw, minimum size=1cm, below left=1cm of Cat] {} ;
    \node [below=0.2cm of Mice] {鼠对象} ;

    \draw ([xshift=-1.5cm, yshift=-1.5cm]Mice.center) [rounded corners]
    rectangle ([xshift=1cm, yshift=3cm]Cat.center)
    node [anchor=north, xshift=-2cm] {观察者对象} ;

    \draw [-Stealth, dashed, postaction={decoration={raise=2pt, text along path,
    text={2}, text align={left indent=2cm}}, decorate}] (Subject) -- (Dog) ;
    \draw [-Stealth, dashed, postaction={decoration={raise=2pt, text along path,
    text={2}, text align={left indent=2cm}}, decorate}] (Subject) -- (Cat) ;
    \draw [-Stealth, dashed, postaction={decoration={raise=2pt, text along path,
    text={2}, text align={left indent=2cm}}, decorate}] (Subject) -- (Mice) ;

    \node (Duck) [circle, draw, minimum size=1cm, fill=green!20, above right=2.5cm of
    Subject] {} ;
    \node [below=0.2cm of Duck] {鸭对象} ;

  \end{tikzpicture}
  }

  \only<2> {
  \begin{tikzpicture}

    \node {2} ;
    \node (Subject) [circle, draw, minimum size=1cm] {};
    \path [postaction={decoration={text along path, text={主 题 对 象}, 
      text align={left indent=3.3cm},
    }, decorate}] (0, 0) circle (0.9cm) ;

    \node (Cat) [circle, draw, minimum size=1cm, right=6cm of Subject] {} ;
    \node [below=0.2cm of Cat] {猫对象} ;

    \node (Dog) [circle, draw, minimum size=1cm, above left=1cm of Cat] {} ;
    \node [below=0.2cm of Dog] {狗对象} ;

    \node (Mice) [circle, draw, minimum size=1cm, below left=1cm of Cat] {} ;
    \node [below=0.2cm of Mice] {鼠对象} ;

    \draw ([xshift=-1.5cm, yshift=-1.5cm]Mice.center) [rounded corners]
    rectangle ([xshift=1cm, yshift=3cm]Cat.center)
    node [anchor=north, xshift=-2cm] {观察者对象} ;

    \draw [-Stealth, dashed, postaction={decoration={raise=2pt, text along path,
    text={2}, text align={left indent=2cm}}, decorate}] (Subject) -- (Dog) ;
    \draw [-Stealth, dashed, postaction={decoration={raise=2pt, text along path,
    text={2}, text align={left indent=2cm}}, decorate}] (Subject) -- (Cat) ;
    \draw [-Stealth, dashed, postaction={decoration={raise=2pt, text along path,
    text={2}, text align={left indent=2cm}}, decorate}] (Subject) -- (Mice) ;

    \node (Duck) [circle, draw, minimum size=1cm, fill=green!20, above right=2.5cm of
    Subject] {} ;
    \node [below=0.2cm of Duck] {鸭对象} ;
    \draw [-Stealth, red, thick, postaction={decoration={raise=2pt,text along path,
    text={|\color{red}|订 阅}, text align={left indent=0.6cm}}, decorate}] (Duck) --
    (Subject) ;


  \end{tikzpicture}
  }

  \only<3> {
  \begin{tikzpicture}

    \node {2} ;
    \node (Subject) [circle, draw, minimum size=1cm] {};
    \path [postaction={decoration={text along path, text={主 题 对 象}, 
      text align={left indent=3.3cm},
    }, decorate}] (0, 0) circle (0.9cm) ;

    \node (Cat) [circle, draw, minimum size=1cm, right=6cm of Subject] {} ;
    \node [below=0.2cm of Cat] {猫对象} ;

    \node (Dog) [circle, draw, minimum size=1cm, above left=1cm of Cat] {} ;
    \node [below=0.2cm of Dog] {狗对象} ;

    \node (Mice) [circle, draw, minimum size=1cm, below left=1cm of Cat] {} ;
    \node [below=0.2cm of Mice] {鼠对象} ;

    \draw ([xshift=-2.2cm, yshift=-1.5cm]Mice.center) [rounded corners]
    rectangle ([xshift=1cm, yshift=3.5 cm]Cat.center)
    node [anchor=north, xshift=-2cm] {观察者对象} ;

    \node (Duck) [circle, draw, minimum size=1cm, above left=1cm of Dog] {} ;
    \node [below=0.2cm of Duck] {鸭对象} ;

  \end{tikzpicture}
  }

  \only<4> {
  \begin{tikzpicture}

    \node {8} ;
    \node (Subject) [circle, draw, minimum size=1cm] {};
    \path [postaction={decoration={text along path, text={主 题 对 象}, 
      text align={left indent=3.3cm},
    }, decorate}] (0, 0) circle (0.9cm) ;

    \node (Cat) [circle, draw, minimum size=1cm, right=6cm of Subject] {} ;
    \node [below=0.2cm of Cat] {猫对象} ;

    \node (Dog) [circle, draw, minimum size=1cm, above left=1cm of Cat] {} ;
    \node [below=0.2cm of Dog] {狗对象} ;

    \node (Mice) [circle, draw, minimum size=1cm, below left=1cm of Cat] {} ;
    \node [below=0.2cm of Mice] {鼠对象} ;

    \draw ([xshift=-2.2cm, yshift=-1.5cm]Mice.center) [rounded corners]
    rectangle ([xshift=1cm, yshift=3.5 cm]Cat.center)
    node [anchor=north, xshift=-2cm] {观察者对象} ;

    \node (Duck) [circle, draw, minimum size=1cm, above left=1cm of Dog] {} ;
    \node [below=0.2cm of Duck] {鸭对象} ;

    \draw [-Stealth, dashed, postaction={decoration={raise=2pt, text along path,
    text={8}, text align={left indent=2cm}}, decorate}] (Subject) -- (Dog) ;
    \draw [-Stealth, dashed, postaction={decoration={raise=2pt, text along path,
    text={8}, text align={left indent=2cm}}, decorate}] (Subject) -- (Cat) ;
    \draw [-Stealth, dashed, postaction={decoration={raise=2pt, text along path,
    text={8}, text align={left indent=2cm}}, decorate}] (Subject) -- (Mice) ;
    \draw [-Stealth, dashed, postaction={decoration={raise=2pt, text along path,
    text={8}, text align={left indent=2cm}}, decorate}] (Subject) -- (Duck) ;

  \end{tikzpicture}
  }

  \only<5> {
  \begin{tikzpicture}

    \node {8} ;
    \node (Subject) [circle, draw, minimum size=1cm] {};
    \path [postaction={decoration={text along path, text={主 题 对 象}, 
      text align={left indent=3.3cm},
    }, decorate}] (0, 0) circle (0.9cm) ;

    \node (Cat) [circle, draw, minimum size=1cm, right=6cm of Subject] {} ;
    \node [below=0.2cm of Cat] {猫对象} ;

    \node (Dog) [circle, draw, minimum size=1cm, above left=1cm of Cat] {} ;
    \node [below=0.2cm of Dog] {狗对象} ;

    \node (Mice) [circle, draw, minimum size=1cm, below left=1cm of Cat] {} ;
    \node [below=0.2cm of Mice] {鼠对象} ;

    \draw ([xshift=-2.2cm, yshift=-1.5cm]Mice.center) [rounded corners]
    rectangle ([xshift=1cm, yshift=3.5 cm]Cat.center)
    node [anchor=north, xshift=-2cm] {观察者对象} ;

    \node (Duck) [circle, draw, minimum size=1cm, above left=1cm of Dog] {} ;
    \node [below=0.2cm of Duck] {鸭对象} ;

    \draw [-Stealth, red, thick, postaction={decoration={raise=2pt,
      text along path,
    text={|\color{red}|取消订阅}, text align={left indent=0.6cm}},
  decorate}] (Mice) -- (Subject) ;
  \end{tikzpicture}
  }

  \only<6> {
  \begin{tikzpicture}

    \node {8} ;
    \node (Subject) [circle, draw, minimum size=1cm] {};
    \path [postaction={decoration={text along path, text={主 题 对 象}, 
      text align={left indent=3.3cm},
    }, decorate}] (0, 0) circle (0.9cm) ;

    \node (Cat) [circle, draw, minimum size=1cm, right=6cm of Subject] {} ;
    \node [below=0.2cm of Cat] {猫对象} ;

    \node (Dog) [circle, draw, minimum size=1cm, above left=1cm of Cat] {} ;
    \node [below=0.2cm of Dog] {狗对象} ;

    \node (Mice) [circle, draw, fill=gray!80, minimum size=1cm, below right=2cm of
    Subject] {} ;
    \node [below=0.2cm of Mice] {鼠对象} ;

    \draw ([xshift=-3.5cm, yshift=-3.0cm]Cat.center) [rounded corners]
    rectangle ([xshift=1cm, yshift=3.5 cm]Cat.center)
    node [anchor=north, xshift=-2cm] {观察者对象} ;

    \node (Duck) [circle, draw, minimum size=1cm, above left=1cm of Dog] {} ;
    \node [below=0.2cm of Duck] {鸭对象} ;

  \end{tikzpicture}
  }

  \only<7> {
  \begin{tikzpicture}

    \node {14} ;
    \node (Subject) [circle, draw, minimum size=1cm] {};
    \path [postaction={decoration={text along path, text={主 题 对 象}, 
      text align={left indent=3.3cm},
    }, decorate}] (0, 0) circle (0.9cm) ;

    \node (Cat) [circle, draw, minimum size=1cm, right=6cm of Subject] {} ;
    \node [below=0.2cm of Cat] {猫对象} ;

    \node (Dog) [circle, draw, minimum size=1cm, above left=1cm of Cat] {} ;
    \node [below=0.2cm of Dog] {狗对象} ;

    \node (Mice) [circle, draw, fill=gray!80, minimum size=1cm, below right=2cm of
    Subject] {} ;
    \node [below=0.2cm of Mice] {鼠对象} ;

    \draw ([xshift=-3.5cm, yshift=-3.0cm]Cat.center) [rounded corners]
    rectangle ([xshift=1cm, yshift=3.5 cm]Cat.center)
    node [anchor=north, xshift=-2cm] {观察者对象} ;

    \node (Duck) [circle, draw, minimum size=1cm, above left=1cm of Dog] {} ;
    \node [below=0.2cm of Duck] {鸭对象} ;

    \draw [-Stealth, dashed, postaction={decoration={raise=2pt, text along path,
    text={14}, text align={left indent=2cm}}, decorate}] (Subject) -- (Dog) ;
    \draw [-Stealth, dashed, postaction={decoration={raise=2pt, text along path,
    text={14}, text align={left indent=2cm}}, decorate}] (Subject) -- (Cat) ;
    \draw [-Stealth, dashed, postaction={decoration={raise=2pt, text along path,
    text={14}, text align={left indent=2cm}}, decorate}] (Subject) -- (Duck) ;
  \end{tikzpicture}
  }


\end{frame}


\begin{frame}
  \frametitle{观察者模式: 定义}

  \begin{block}{观察者模式}
    观察者模式定义了对象间的\uline{一对多}的\uline{依赖}关系,当一个对象
    \uline{改变状态}时,它所
    有的依赖者都会\uline{收到通知}并\uline{自动更新}
  \end{block}
\end{frame}

\begin{frame}
  \frametitle{观察者模式: 类图}

  \scalebox{0.8} {
  \begin{tikzpicture}
    \tikzumlset{fill class=white}

    \umlclass[type=interface]{Subject}{}{
      registerObserver() \\
      removeObserver() \\
      notifyObservers()
    }

    \umlclass[type=interface, x=7] {Observer}{}{
      update()
    }

    \umlclass[y=-5]{ConcreteSubject}{}{
      registerObserver() \\
      removeObserver() \\
      notifyObservers() \\
      getState() \\
      setState()
    }

    \umlclass[x=7,y=-5]{ConcreteObserver}{}{
      update()
    }

    \umlimpl{ConcreteSubject}{Subject}
    \umlimpl{ConcreteObserver}{Observer}

    \umluniassoc[arg=许多观察者, pos=0.5]{Subject}{Observer}
    \umluniassoc[arg=主题, pos=0.5]{ConcreteObserver}{Subject}

  \end{tikzpicture}
}

\end{frame}

\begin{frame}
  \frametitle{观察者模式使得对象间松耦合}

  \only<1> {
  松耦合的威力 --- 彼此能够交互,但互相知之甚少
  \begin{itemize}
    \item 主题对象只知道观察者对象实现了一个\uline{接口}
    \item 可以在\uline{任何时刻}添加各种观察者对象
    \item 增加新的观察者时,\uline{不需要更改}主题对象
    \item 可以\uline{独立复用}主题对象或者观察者对象
    \item 主题对象或观察者对象发生的改动\uline{不会影响}对方
  \end{itemize}
  }

  \only<2> {
    \begin{block}{设计原则: 松耦合}
      尽最大可能将要交互的对象设计为松耦合的
    \end{block}

    \vspace*{2ex}

    松耦合的设计允许我们构建灵活的OO系统,以应对\uline{未来的变动},因为松耦合可
    以使对象间的彼此\uline{依赖降低}
  }
  
\end{frame}

\begin{frame}
  \frametitle{设计气象站应用}
  \scalebox{0.8} {
  \begin{tikzpicture}
    \tikzumlset{fill class=white}

    \umlclass[type=interface]{Subject}{}{
      registerObserver() \\
      removeObserver() \\
      notifyObservers()
    }

    \umlclass[type=interface, x=6] {Observer}{}{
      update()
    }

    \umlclass[type=interface, x=9] {Display}{}{
      display()
    }

    \umlclass[y=-5]{WeatherData}{}{
      registerObserver() \\
      removeObserver() \\
      notifyObservers() \\
      getTemperature() \\
      getHumidity() \\
      getPressure() \\
      measurementsChanged()
    }

    \umlclass[x=7.5,y=-5, fill=yellow!20]{CurrentConditionsDisplay}{}{
      update() \\
      display()
    }

    \umlimpl{WeatherData}{Subject}
    \umlimpl{CurrentConditionsDisplay}{Observer}
    \umlimpl{CurrentConditionsDisplay}{Display}

    \umluniassoc[arg=observers, pos=0.5]{Subject}{Observer}
    \umluniassoc[arg=subject,
    pos=0.5]{CurrentConditionsDisplay}{Subject}

  \end{tikzpicture}
}


\end{frame}

{
\defverbatim{\verbinterfaces}{%
\begin{Verbatim}[label=建立接口]
public interface Subject {
    public void registerObserver(Observer o); 
    public void removeObserver(Observer o); 
    public void notifyObservers();
}
public interface Observer {
    public void update(float temp, float humidity, 
      float pressure);
}
public interface Display{ 
    public void display();
}
\end{Verbatim}
}

\defverbatim{\verbweatherdataa}{%
\begin{Verbatim}[label=WeatherData对象]
public class WeatherData implements Subject { 
  private ArrayList observers;
  private float temperature;
  private float humidity;
  private float pressure;

  public WeatherData() { 
    observers = new ArrayList();
  }

  public void registerObserver(Observer o) { 
    observers.add(o);
  }
\end{Verbatim}
}

\defverbatim{\verbweatherdatab}{%
\begin{Verbatim}[firstnumber=last, label=WeatherData对象]
  public void removeObserver(Observer o) { 
    int i = observers.indexOf(o);
    if (i >= 0) {
      observers.remove(i); 
    }
  }

  public void notifyObservers() {
    for (int i = 0; i < observers.size(); i++) {
      Observer observer = (Observer)observers.get(i);
      observer.update(temperature, humidity, pressure); 
    }
  }
\end{Verbatim}
}

\defverbatim{\verbweatherdatac}{%
\begin{Verbatim}[firstnumber=last, label=WeatherData对象]
  public void measurementsChanged() { 
      notifyObservers();
  }

  public void setMeasurements(float temperature, 
    float humidity, float pressure) { 
      this.temperature = temperature;
      this.humidity = humidity;
      this.pressure = pressure;
      measurementsChanged(); 
  }
  // other methods ...
}
\end{Verbatim}
}

\defverbatim{\verbcurrentdispa}{%
\begin{Verbatim}[label=显示当前天气布告板类]
public class CurrentConditionsDisplay 
  implements Observer, Display{ 
  private float temperature;
  private float humidity;
  private Subject weatherData;

  public CurrentConditionsDisplay(Subject weatherData) { 
    this.weatherData = weatherData; 
    weatherData.registerObserver(this);
  }
\end{Verbatim}
}

\defverbatim{\verbcurrentdispb}{%
\begin{Verbatim}[firstnumber=last, label=显示当前天气布告板类]
  public void update(float temperature, float humidity, 
    float pressure) { 
    this.temperature = temperature;
    this.humidity = humidity;
    display();
  }

  public void display() {
    System.out.println("Current conditions:"+ temperature
      + "F degrees and " + humidity + "% humidity");
  }
}
\end{Verbatim}
}

\defverbatim{\verbweatherstation}{%
\renewcommand{\baselinestretch}{1.2}
\begin{Verbatim}[label=测试]
public class WeatherStation {
  public static void main(String[] args) {
   WeatherData wd = new WeatherData();

    CurrentConditionsDisplay cd =
      new CurrentConditionsDisplay(wd);
    StatisticsDisplay sd = new StatisticsDisplay(wd); 
    ForecastDisplay fd = new ForecastDisplay(wd);

    wd.setMeasurements(80, 65, 30.4f); 
    wd.setMeasurements(82, 70, 29.2f); 
    wd.setMeasurements(78, 90, 29.2f);
  }
}
\end{Verbatim}
}

\defverbatim{\verboutput}{%
\begin{Verbatim}[label=输出结果]
%java WeatherStation
Current conditions: 80.0F degrees and 65.0% humidity 
Avg/Max/Min temperature = 80.0/80.0/80.0
Forecast: Improving weather on the way!
Current conditions: 82.0F degrees and 70.0% humidity 
Avg/Max/Min temperature = 81.0/82.0/80.0
Forecast: Watch out for cooler, rainy weather 
Current conditions: 78.0F degrees and 90.0% humidity 
Avg/Max/Min temperature = 80.0/82.0/78.0
Forecast: More of the same
%
\end{Verbatim}
}

\begin{frame}
  \frametitle{实现气象站应用}

  \only<1> {
    \verbinterfaces
  }

  \only<2> {
    \verbweatherdataa
  }

  \only<3> {
    \verbweatherdatab
  }

  \only<4> {
    \verbweatherdatac
  }

  \only<5> {
    \verbcurrentdispa
  }

  \only<6> {
    \verbcurrentdispb
  }

  \only<7> {
    \verbweatherstation
  }

  \only<8> {
    \verboutput
  }

\end{frame}
}

{
\defverbatim{\verboutput}{%
\begin{Verbatim}[commandchars=\\\{\}, label=新的输出结果]
%java WeatherStation
Current conditions: 80.0F degrees and 65.0% humidity 
Avg/Max/Min temperature = 80.0/80.0/80.0
Forecast: Improving weather on the way!
\uline{Heat index is 82.95535}
Current conditions: 82.0F degrees and 70.0% humidity 
Avg/Max/Min temperature = 81.0/82.0/80.0
Forecast: Watch out for cooler, rainy weather 
\uline{Heat index is 86.90124}
Current conditions: 78.0F degrees and 90.0% humidity 
Avg/Max/Min temperature = 80.0/82.0/78.0
Forecast: More of the same
\uline{Heat index is 83.64967}
\end{Verbatim}
}

\begin{frame}
\frametitle{新的需求: 酷热指数布告板}

\only<1> {
系统开发完成后,用户来电告知其需要酷热指数布告板。\\
酷热指数是一个结合温度$t$和湿度$h$的指数,用于显示人的温度感受。酷热指数的计算公式:
{
\footnotesize
\begin{eqnarray*}
\lefteqn{heatindex  = } \\
& & 16.923 + 1.85212 \times 10^{-1}\times t + 5.37941 \times h  - 1.00254 \times 10^{-1} \times{}t\times{}h + \\
& & 9.41695 \times 10^{-3}\times t^2 + 7.28898\times10^{-3} \times h^2 + 3.45372\times10^{-4}\times t^2\times h -\\
& & 8.14971\times 10^{-4} \times t\times h^2 + 1.02102\times10^{-5} \times t^2 \times h^2 -3.8646\times10^{-5} \times t^3 +\\ 
& & 2.91583\times10^{-5}\times h^3 + 1.42721\times 10^{-6}\times t^3 \times h + 1.97483\times10^{-7} \times t \times h^3 -\\
& & 2.18429\times10^{-8}\times t^3 \times h^2 + 8.43296\times10^{-10} \times t^2 \times h^3  - 4.81975\times10^{-11}\times\\
& & t^3 \times h^3
\end{eqnarray*}
}
}

\only<2> {
  \verboutput
}

\end{frame}

}

\begin{frame}
  \frametitle{Java内置的观察者模式}

  \noindent\scalebox{0.75}{
  \begin{tikzpicture}
    \tikzumlset{fill class=white}

    \umlclass[fill=yellow!20]{Observable}{}{
      addObserver() \\
      deleteObserver() \\
      notifyObservers() \\
      setChanged()
    }

    \umlclass[type=interface, x=8.5, fill=yellow!20] {Observer}{}{
      update()
    }

    \umlclass[y=-5]{WeatherData}{}{
      getTemperature() \\
      getHumidity() \\
      getPressure() 
    }

    \umlclass[x=5,y=-5]{GeneralDisplay}{}{
      update() \\
      display()
    }

    \umlclass[x=8.5,y=-5]{StatisticsDisplay}{}{
      update() \\
      display()
    }

    \umlclass[x=12,y=-5]{ForecastDisplay}{}{
      update() \\
      display()
    }

    \umlinherit{WeatherData}{Subject}
    \umlimpl{GeneralDisplay}{Observer}
    \umlimpl{StatisticsDisplay}{Observer}
    \umlimpl{ForecastDisplay}{Observer}

    \umluniassoc[arg=observers, pos=0.5]{Observable}{Observer}
    \umluniassoc[arg=subject,
    pos=0.5]{GeneralDisplay}{Observable}

  \end{tikzpicture}
}

\end{frame}

{
\defverbatim{\verbobservable}{%
\renewcommand{\baselinestretch}{1.2}
\begin{Verbatim}[commandchars=\\\[\], label=Observable的部分伪码]
setChanged() {
  \uline[changed = true]
}
notifyObservers(Object arg) { 
  \uline[if (changed)] {
    for every observer on the list {
      call update (this, arg) 
    }
    \uline[changed = false ]
  }
}
notifyObservers() {
  notifyObservers(null) 
}
\end{Verbatim}
}

\begin{frame}
  \frametitle{Java内置观察者模式如何工作}

  \only<1> {
    \begin{itemize} 
      \item 如何成为观察者?
        \begin{itemize}
          \item [] 实现 java.util.Observer 接口,调用Observable类的addObserver方法
        \end{itemize}
      \item 被观察者如何发送数据改变的通知?
        \begin{enumerate}
          \item 首先,调用setChanged方法,通告该对象状态已改变
          \item 其次,调用 notifyObservers()  或者\\
            notifyObservers(Object arg)
        \end{enumerate}

      \item 观察者如何接收通知?
        \begin{itemize}
          \item [] 实现 update(Observable o, Object arg) 方法,支持
            ``推''和``拉''两种获得数据的方式
        \end{itemize}

    \end{itemize}
  }

  \only<2> {
    \verbobservable
  }
\end{frame}
}

{
\defverbatim{\verbweatherdataa}{%
\begin{Verbatim}[commandchars=\\\[\], label=WeatherData部分代码]
import java.util.Observable; 
import java.util.Observer;

public class WeatherData extends Observable { 
  private float temperature;
  private float humidity;
  private float pressure;

  public WeatherData() { }
  public void measurementsChanged() {
    setChanged(); 
    notifyObservers(); 
  }
\end{Verbatim}
}

\defverbatim{\verbweatherdatab}{%
\begin{Verbatim}[firstnumber=last, commandchars=\\\[\], 
  label=WeatherData部分代码]
  public void setMeasurements(float temperature, 
    float humidity, float pressure) { 
    this.temperature = temperature;
    this.humidity = humidity;
    this.pressure = pressure;
    measurementsChanged(); 
  }

  public float getTemperature() { 
    return temperature; // pulled by observers
  }
  // other methods including getHumidity() getPressure() 
}
\end{Verbatim}
}

\defverbatim{\verbcurrentconda}{%
\begin{Verbatim}[commandchars=\\\[\], 
  label=新的CurrentConditionsDisplay]
import java.util.Observable; 
import java.util.Observer;

public class CurrentConditionsDisplay 
  implements Observer, Display{
  Observable observable;
  private float temp; 
  private float humidity;

  public CurrentConditionsDisplay(Observable observable){ 
    this.observable = observable ;
    observable.addObserver(this) ;
  }
\end{Verbatim}
}

\defverbatim{\verbcurrentcondb}{%
\begin{Verbatim}[firstnumber=last, commandchars=\\\[\], 
  label=新的CurrentConditionsDisplay]
  public void update(Observable obs, Object arg) { 
    if (obs instanceof WeatherData) {
      WeatherData weatherData = (WeatherData)obs; 
      this.temp = weatherData.getTemperature(); 
      this.humidity = weatherData.getHumidity(); 
      display();
    }
  }
  public void display() {
    System.out.println("Current conditions: " + temp
      + "F degrees and " + humidity + "% humidity");
  }
}
\end{Verbatim}
}

\defverbatim{\verboutput}{%
\begin{Verbatim}[commandchars=\\\[\], label=运行新代码,观察输出和之前版本的差别]
%java WeatherStation
\uline[Forecast: Improving weather on the way!]
Avg/Max/Min temperature = 80.0/80.0/80.0
\dashuline[Current conditions: 80.0F degrees and 65.0% humidity]
Forecast: Watch out for cooler, rainy weather 
Avg/Max/Min temperature = 81.0/82.0/80.0
Current conditions: 82.0F degrees and 70.0% humidity 
Forecast: More of the same
Avg/Max/Min temperature = 80.0/82.0/78.0
Current conditions: 78.0F degrees and 90.0% humidity 
\end{Verbatim}
}
\begin{frame}
\frametitle{用Java内置观察者模式实现气象站应用}
\only<1> {
  \verbweatherdataa
}
\only<2> {
  \verbweatherdatab
}
\only<3> {
  \verbcurrentconda
}
\only<4> {
  \verbcurrentcondb
}
\only<5> {
  \verboutput

相同的计算结果,不同的输出次序,为什么? 有无危害?
}

\end{frame}
}

\begin{frame}
\frametitle{java.util.Observable的问题}
\begin{itemize}
\item Observable是一个\uline{类} $\Longrightarrow$ 用户必须设计一个类\uline{继承}它 \\
\begin{itemize}
\item 某类难以同时具有Observable和另一个超类的行为 
\item 没有Observable接口,无法建立用户自己的实现
\end{itemize}
\item Observable将关键的方法setChanged\uline{保护}起来
\begin{itemize}
\item 只能继承Observable,不能创建Observable实例聚合到自己的对象中
\item 违反了 "\textcolor{blue}{多用聚合、少用继承}"的设计原则
\end{itemize}
\end{itemize}
\end{frame}

\begin{frame}[label=current]
\frametitle{小结}
\only<1> {
  \textcolor{blue}{关于观察者模式}:
\begin{itemize}
\item 观察者模式定义了对象之间\uline{一对多}的关系
\item 主题(被观察者)用一个\uline{共同的接口}更新观察者
\item 主题和观察者之间是\uline{松耦合}的,主题不知道具体观察者的细节,只知道
  其实现了观察者接口
\item 可以用\uline{推(push)}或\uline{拉(pull)}的方式将数据传给观察者 
\item 有多个观察者时,不可以依赖特定的通知次序
\item Java的Observable实现有一些问题
\end{itemize}
}

\only<2> {
  \textcolor{blue}{关于设计原则}: 
\begin{enumerate}
\item 封装变化
\item 多用聚合、少用继承
\item 针对接口编程,不针对实现编程
\item 尽最大可能将要交互的对象设计为松耦合的
\end{enumerate}
}
\end{frame}

\end{document}
