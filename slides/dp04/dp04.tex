\documentclass[compress]{beamer}

\usepackage[nofonts]{ctex}
\setCJKmainfont[ItalicFont={Kaiti SC}]{Kaiti SC}%
%\setCJKmainfont[ItalicFont={AR PL KaitiM GB}]{AR PL KaitiM GB}%
%\setCJKsansfont{WenQuanYi Zen Hei}% 文泉驿的黑体

\mode<beamer>
{
     \useinnertheme{rectangles}
     %\useoutertheme{infolines}
     %\useoutertheme{split}
     \usecolortheme{rose}
     \usecolortheme{seahorse}

     \setbeamertemplate{navigation symbols}{}%remove navigation symbols

%     \expandafter\def\expandafter\insertshorttitle\expandafter{%
%     \insertshorttitle\hfill%
%     \insertframenumber\,/\,\inserttotalframenumber}
}

\defbeamertemplate*{footline}{mytheme}
{
  \leavevmode%
  \hbox{%
  \begin{beamercolorbox}[wd=\paperwidth,ht=2.25ex,dp=1ex,center]{author in head/foot}%
    \raisebox{-1ex}{\includegraphics[width=3ex]{Overlays/logo.pdf}}%
    \hspace*{6ex}\insertframenumber{} / \inserttotalframenumber 
  \end{beamercolorbox}}%
  \vskip0pt%
}
\usebeamertemplate{mytheme}

\setbeamertemplate{headline}{}

%\setbeamercovered{transparent}

\mode<handout>
{
	\usetheme{default}
	\usepackage{pgfpages}
	\pgfpagesuselayout{2 on 1}[a4paper,landscape,border shrink=5mm]
}


\usepackage{amsmath,latexsym,amssymb,amsfonts,amsbsy}
\usepackage{graphicx}
\usepackage{array}
\usepackage{hyperref}
\usepackage{textpos}
\usepackage{ulem}
\usepackage{comment}
\usepackage{fancyvrb}
\fvset{frame=single, numbers=left, fontsize=\small}
\usepackage{tikz}
\usetikzlibrary{calc,arrows.meta, graphs, trees, shapes, positioning, decorations.markings, intersections, decorations.text}
\usepackage{tikz-uml}

\newcommand{\romannumber}[1]{{\textrm{\uppercase\expandafter{\romannumeral
#1}}}}

\newcommand{\textblue}[1]{\textcolor{blue}{#1}}

\setbeamercolor{dblue}{fg=white,bg=gray!70!blue} % for beamercolorbox
 \newenvironment{pblock}{\begin{beamercolorbox}[rounded=true,
          shadow=false]{dblue}}{\end{beamercolorbox}}
 \newenvironment{bblock}{\begin{beamercolorbox}[rounded=true,
          shadow=false]{fg=black, bg=white}}{\end{beamercolorbox}}

\graphicspath{{figure/}}

%%%%%%%%%%%%%%%%%%%%%%%%%%%%%%%%%%%%%%%%%%%%%%%%%%%%%%%%%%%%%%%%%
%    body                                                       %
%%%%%%%%%%%%%%%%%%%%%%%%%%%%%%%%%%%%%%%%%%%%%%%%%%%%%%%%%%%%%%%%%


\begin{document}

%\AtBeginSubsection[]
\AtBeginSection[]
{ 
    \begin{frame}<beamer> 
		\frametitle{内容提要} 
		\tableofcontents[currentsection,currentsubsection] 
	\end{frame} 
} 

					
\title[单件模式]{面向对象设计模式 ~~ 单件}

\author[曹东刚]
{曹东刚\\\href{mailto:caodg@pku.edu.cn}{caodg@pku.edu.cn}}

\institute[北京大学]{北京大学信息学院研究生课程 - 面向对象的分析与设计 \\
    \href{http://sei.pku.edu.cn/~caodg/course/oo}{http://sei.pku.edu.cn/\~{}caodg/course/oo}}

\date{}

\titlegraphic{\includegraphics[height=0.10\textwidth]{Overlays/logo.pdf}}

\begin{frame}[plain]
	\titlepage
\end{frame}

\setcounter{framenumber}{0}
%\includeonlyframes{current}

\begin{frame}
  \frametitle{单件模式(Singleton Pattern)}
  单件模式用于创建\uline{唯一}的对象实例
  \begin{itemize}
    \item 很多对象只需一个,如线程池、缓存、日志、打印机驱动等
    \item 用全局变量或静态变量也能保证确保创建唯一的实例,但单件模式提供
      了全局的访问点,和全局变量一样方便,又没有全局变量的缺点
    \item 单件模式可以允许在需要的时候才创建对象
  \end{itemize}

  \vspace*{2ex}
  \framebox{\parbox{0.9\hsize}{\centering 核心问题:如何保证一个对象只被实
  例化一次?}}
\end{frame}

\begin{frame}[fragile]
  \frametitle{通常如何创建对象}

\begin{Verbatim}[label=常见的创建对象的方式, commandchars=\\\^\$]
    MyObject o = \textblue^new$ MyObject() ;
\end{Verbatim}

\alert{问题}: 对象的创建不受限制,可创建任意多个 \\
\alert{办法}: 限制用\textblue{new}操作符创建对象 \\[4ex]

\end{frame}

\begin{frame}[fragile]
  \frametitle{限制用new创建对象}
\begin{Verbatim} [label=将构造函数声明为私有的, commandchars=\\\^\$]
  public MyClass {
    \alert^private$ MyClass() {}
  }
\end{Verbatim}
\vspace*{3ex}
只有MyClass的实例才能调用该构造函数,但由于其他类无法实例化MyClass,故产
生``鸡生蛋、蛋生鸡''的问题
\end{frame}

\begin{frame}[fragile]
  \frametitle{私有构造函数 $+$ 公有类方法}
\begin{Verbatim} [label=用静态方法避开必须实例化的问题, commandchars=\\\^\$]
  public MyClass {
    \alert^private$ MyClass() {}

    public \alert^static$ MyClass getInstance() {
      return new MyClass() ;
    }
  }
\end{Verbatim}
\vspace*{2ex}
\verb~getInstance()~ 是一个类方法(静态方法),不需要对象实例
\end{frame}

\begin{frame}[fragile]
  \frametitle{经典的单件模式实现}
\begin{Verbatim} [label=一个初步的实现, commandchars=\\\^\$]
  public class Singleton {
    \alert^private static$ Singeleton uniqueInstance ;

    \alert^private$ Singleton () {} 

    \alert^public static$ Singleton getInstance() {
      if (uniqueInstance == null) 
        uniqueInstance = new Singleton() ;
      return uniqueInstance ;
    }
  }
\end{Verbatim}
\end{frame}

\begin{frame}
  \frametitle{定义单件模式}
  \begin{block}{单件模式(Singleton Pattern)}
    单件模式确保一个类只有一个实例,并提供一个全局的访问点
  \end{block}

  要点: 类肩负了管理实例的额外责任! \\[2ex]

  \qquad\begin{tikzpicture}
    \umlclass[fill=white]{Singleton}{static uniqueInstance}{static
    getInstance()}
  \end{tikzpicture}

\end{frame}

\begin{frame}[fragile]
  \frametitle{经典的单件模式实现在多线程下的表现}
\begin{Verbatim} [label=多线程下有问题, commandchars=\\\^\$]
  public class Singleton {
    private static Singeleton uniqueInstance ;

    private Singleton {} 

    \uline^public static Singleton getInstance()$ {
      if (uniqueInstance == null) 
        uniqueInstance = new Singleton() ;
      return uniqueInstance ;
    }
  }
\end{Verbatim}
\end{frame}

\begin{frame}[fragile]
  \frametitle{利用synchronize处理多线程问题}
\begin{Verbatim} [label=synchronized保护整个方法,性能会有影响, commandchars=\\\^\$]
  public class Singleton {
    private static Singeleton uniqueInstance ;

    private Singleton {} 

    public static \alert^synchronized$ Singleton getInstance() {
      if (uniqueInstance == null) 
        uniqueInstance = new Singleton() ;
      return uniqueInstance ;
    }
  }
\end{Verbatim}
\end{frame}


\begin{frame}[fragile]
  \frametitle{提早创建实例}
\begin{Verbatim} [label=在类加载时就创建实例, commandchars=\\\^\$]
  public class Singleton {
    \uline^private static Singeleton uniqueInstance =$ 
      \uline^new Singleton() $;

    private Singleton {} 

    public static Singleton getInstance() {
      return uniqueInstance ;
    }
  }
\end{Verbatim}
\end{frame}

\begin{frame}[fragile]
  \frametitle{用双重检查加锁方法减少同步}
\begin{Verbatim} [label=只有第一次会同步, commandchars=\\\^\$]
  public class Singleton {
    private \alert^volatile$ static Singeleton uniqueInstance ;
    private Singleton {} 
    public static Singleton getInstance() {
      \uline^if (uniqueInstance == null)$ {
        \uwave^synchronized (Singleton.class)$ {
          \uline^if (uniqueInstance ==  null)$ 
            uniqueInstance = new Singleton() ;
        }
      }
      return uniqueInstance ;
    }
  }
\end{Verbatim}

\end{frame}

\begin{frame}
  \frametitle{单件模式小结}
  \begin{itemize}
    \item 单件模式确保程序中一个类最多只有一个实例
    \item 单件模式也提供这个实例的全局访问点
    \item 考虑\uline{性能}和\uline{可用资源}的因素小心设计单件
    \item 实现时要考虑运行环境的支持,如JVM的垃圾回收机制、类加载器特性
      等
  \end{itemize}
\end{frame}

\begin{frame}
\frametitle{关于设计原则的小结}
  \begin{enumerate}
    \item 封装变化
    \item 多用聚合、少用继承
    \item 针对接口编程,不针对实现编程
    \item 尽最大可能将要交互的对象设计为松耦合的
    \item 对扩展开放,对修改封闭
    \item 依赖抽象,不要依赖具体类
  \end{enumerate}
\end{frame}

\end{document}
