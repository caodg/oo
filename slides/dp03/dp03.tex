\documentclass[compress]{beamer}

\usepackage[nofonts]{ctex}
\setCJKmainfont[ItalicFont={Kaiti SC}]{Kaiti SC}%
%\setCJKmainfont[ItalicFont={AR PL KaitiM GB}]{AR PL KaitiM GB}%
%\setCJKsansfont{WenQuanYi Zen Hei}% 文泉驿的黑体

\mode<beamer>
{
     \useinnertheme{rectangles}
     %\useoutertheme{infolines}
     %\useoutertheme{split}
     \usecolortheme{rose}
     \usecolortheme{seahorse}

     \setbeamertemplate{navigation symbols}{}%remove navigation symbols

%     \expandafter\def\expandafter\insertshorttitle\expandafter{%
%     \insertshorttitle\hfill%
%     \insertframenumber\,/\,\inserttotalframenumber}
}

\defbeamertemplate*{footline}{mytheme}
{
  \leavevmode%
  \hbox{%
  \begin{beamercolorbox}[wd=1.0\paperwidth,ht=2.25ex,dp=1ex,center]{author in head/foot}%
    \raisebox{-1ex}{\includegraphics[width=3ex]{Overlays/logo.pdf}}%
    \hspace*{6ex}\insertframenumber{} / \inserttotalframenumber 
  \end{beamercolorbox}}%
  \vskip0pt%
}
\usebeamertemplate{mytheme}

\setbeamertemplate{headline}{}

%\setbeamercovered{transparent}

\mode<handout>
{
	\usetheme{default}
	\usepackage{pgfpages}
	\pgfpagesuselayout{2 on 1}[a4paper,landscape,border shrink=5mm]
}


\usepackage{amsmath,latexsym,amssymb,amsfonts,amsbsy}
\usepackage{graphicx}
\usepackage{array}
\usepackage{hyperref}
\usepackage{textpos}
\usepackage{ulem}
\usepackage{comment}
\usepackage{fancyvrb}
\fvset{frame=single, numbers=left, fontsize=\small}
\usepackage{tikz}
\usetikzlibrary{calc,arrows.meta, graphs, trees, shapes, positioning, decorations.markings, intersections, decorations.text}
\usepackage{tikz-uml}

\newcommand{\romannumber}[1]{{\textrm{\uppercase\expandafter{\romannumeral
#1}}}}

\newcommand{\textblue}[1]{\textcolor{blue}{#1}}

\setbeamercolor{dblue}{fg=white,bg=gray!70!blue} % for beamercolorbox
 \newenvironment{pblock}{\begin{beamercolorbox}[rounded=true,
          shadow=false]{dblue}}{\end{beamercolorbox}}
 \newenvironment{bblock}{\begin{beamercolorbox}[rounded=true,
          shadow=false]{fg=black, bg=white}}{\end{beamercolorbox}}

\graphicspath{{figure/}}

%%%%%%%%%%%%%%%%%%%%%%%%%%%%%%%%%%%%%%%%%%%%%%%%%%%%%%%%%%%%%%%%%
%    body                                                       %
%%%%%%%%%%%%%%%%%%%%%%%%%%%%%%%%%%%%%%%%%%%%%%%%%%%%%%%%%%%%%%%%%


\begin{document}

%\AtBeginSubsection[]
\AtBeginSection[]
{ 
    \begin{frame}<beamer> 
		\frametitle{内容提要} 
		\tableofcontents[currentsection,currentsubsection] 
	\end{frame} 
} 

					
\title[工厂模式]{面向对象设计模式 ~~ 工厂}

\author[曹东刚]
{曹东刚\\\href{mailto:caodg@pku.edu.cn}{caodg@pku.edu.cn}}

\institute[北京大学]{北京大学信息学院研究生课程 - 面向对象的分析与设计 \\
    \href{http://sei.pku.edu.cn/~caodg/course/oo}{http://sei.pku.edu.cn/\~{}caodg/course/oo}}

\date{}

\titlegraphic{\includegraphics[height=0.10\textwidth]{Overlays/logo.pdf}}

\begin{frame}[plain]
	\titlepage
\end{frame}

\setcounter{framenumber}{0}

%\includeonlyframes{current}

\section{简单工厂}

\begin{frame}[fragile]
  \frametitle{new 操作符的问题}
  \verb~Duck duck = new MallardDuck() ;~ \\
  当使用 \alert{new} 操作符实例化对象时,是在实例化具体类,所以是针对实
  现编程,而不是针对接口编程,会使代码\uline{缺乏弹性}  \\[3ex]
\begin{Verbatim}[label=使用大量具体类而难以更新和维护的代码]
  Duck duck;
  if (picnic) {
    duck = new MallardDuck();
  } else if (hunting) {
    duck = new DecoyDuck();
  } else if (inBathTub) {
    duck = new RubberDuck();
  }
\end{Verbatim}

\end{frame}

\begin{frame}
\frametitle{怎么办?}
到OO设计原则中寻找答案:\\[2ex]
\begin{block}{设计原则: 封装变化性}
    找出应用中可能需要变化之处, 把它们独立出来, 不要和那些不需要变化
    的代码混在一起。 \\
    即: 系统中的某部分改变不要影响其他部分。
\end{block}

\vspace*{3ex}
如何将实例化具体类的代码抽离或封装起来?

\end{frame}

{
\defverbatim{\verbpizzanew}{%
  %\renewcommand{\baselinestretch}{1.1}
\begin{Verbatim}[label=订购Pizza的代码, commandchars=\\\[\]]
Pizza orderPizza() {
  \uline[Pizza pizza = new Pizza()];

  pizza.prepare(); 
  pizza.bake(); 
  pizza.cut(); 
  pizza.box(); 
  return pizza;
}
\end{Verbatim}
}

\defverbatim[colored]{\verborderpizza}{%
  %\renewcommand{\baselinestretch}{1.1}
\begin{Verbatim}[label=根据Pizza的类型实例化具体类, commandchars=\\\[\]]
Pizza orderPizza(String type) { 
  Pizza pizza;
  \textblue[if (type.equals("cheese")) { ]
  \textblue[  pizza = new CheesePizza();]
  \textblue[} else if (type.equals("greek") { ]
  \textblue[  pizza = new GreekPizza();]
  \textblue[} else if (type.equals("pepperoni") { ]
  \textblue[  pizza = new PepperoniPizza();]
  \textblue[}]

  pizza.prepare();
  // the following procedure is the same as the previous
\end{Verbatim}
}

\defverbatim[colored]{\verborderpizzamore}{%
  %\renewcommand{\baselinestretch}{1.1}
\begin{Verbatim}[label=但随后需要增加和删除某些Pizza, commandchars=\\\[\]]
Pizza orderPizza(String type) { 
  Pizza pizza;
  if (type.equals("cheese")) { 
    pizza = new CheesePizza();
  } \sout[else if (type.equals("greek") { ]
    \sout[pizza = new GreekPizza();]
  \sout[}] else if (type.equals("pepperoni") { 
    pizza = new PepperoniPizza();
  }\textblue[ else if (type.equals("clam") { ]
  \textblue[  pizza = new ClamPizza();]
  \textblue[} else if (type.equals("veggie") { ]
  \textblue[  pizza = new VeggiePizza();]
  \textblue[}]
\end{Verbatim}
}


\defverbatim[colored]{\verborderpizzaseal}{%
  %\renewcommand{\baselinestretch}{1.1}
\begin{Verbatim}[label=将创建对象代码移出orderPizza, commandchars=\\\[\]]
Pizza orderPizza(String type) { 
  Pizza pizza;

  \framebox[ 将创建对象代码从此处移走 \quad]
  
  pizza.prepare(); 
  pizza.bake(); 
  pizza.cut(); 
  pizza.box(); 
  return pizza;
}
\end{Verbatim}
}

\defverbatim[colored]{\verbsimpfactory}{%
  \renewcommand{\baselinestretch}{1.2}
\begin{Verbatim}[label=将创建具体对象的代码放到一个工厂类里, commandchars=\\\[\]]
public class SimplePizzaFactory {
  \textblue[public Pizza createPizza(String type) {]
    Pizza pizza = null;
    if (type.equals("cheese")) { 
      pizza = new CheesePizza();
    } else if (type.equals("pepperoni")) { 
      pizza = new PepperoniPizza();
    } else if (type.equals("clam")) {
      pizza = new ClamPizza();
    } else if (type.equals("veggie")) {
      pizza = new VeggiePizza(); 
    }
   return pizza; 
  \textblue[}]
\end{Verbatim}
}

\defverbatim[colored]{\verbstore}{%
  %\renewcommand{\baselinestretch}{1.2}
\begin{Verbatim}[label=重新实现PizzaStore, commandchars=\\\[\]]
public class PizzaStore {
  \textblue[SimplePizzaFactory factory;]

  \textblue[public PizzaStore(SimplePizzaFactory factory) { ]
  \textblue[  this.factory = factory;]
  \textblue[}]

  public Pizza orderPizza(String type) { 
    Pizza pizza;
    \textblue[pizza = factory.createPizza(type);]
    pizza.prepare();
    //the following procedure is the same as the previous
\end{Verbatim}
}
\begin{frame}
\frametitle{识别变化的方面:从一个Pizza店应用开始}
  \only<1> {
    \verbpizzanew

    为了让系统有弹性,将Pizza设计为抽象类,但这样就无法实例化各种具体的Pizza类。因此需要增加代码来决定创建何种Pizza
  }

  \only<2> {
    \verborderpizza
  }
  \only<3> {
    \verborderpizzamore
  }
  \only<4> {
    \verborderpizzaseal
  }
  \only<5> {
  \vspace*{2ex}
    \verbsimpfactory
  }
  \only<6> {
    \verbstore
  }
\end{frame}
}

\begin{frame}
\frametitle{简单工厂类下的Pizza店新类图}
\noindent\begin{tikzpicture}
\tikzumlset{fill class=white}

\umlclass{SimplePizzaFactory}{}{createPizza()}
\umlclass[x=-4.5]{PizzaStore}{}{orderPizza()}
\umlclass[x=4.5, type=abstract]{Pizza}{}{prepare()\\
  bake() \\
  cut() \\
  box() }

\umlsimpleclass[x=0, y=-2]{CheesePizza}
\umlsimpleclass[x=0.5, y=-3]{ViggiePizza}
\umlsimpleclass[x=2, y=-4]{ClamPizza}
\umlsimpleclass[x=4, y=-3]{PepperoniPizza}

\umluniassoc{PizzaStore}{SimplePizzaFactory}
\umluniassoc{SimplePizzaFactory}{Pizza}

\umlinherit{CheesePizza}{Pizza}
\umlinherit{ViggiePizza}{Pizza}
\umlinherit{ClamPizza}{Pizza}
\umlinherit{PepperoniPizza}{Pizza}

\end{tikzpicture}
\end{frame}

\section{工厂方法}

\begin{frame}
\frametitle{加盟Pizza店}
很多Pizza店希望能成为我们的加盟店,他们应使用我们现有的OO代码。
但是这些加盟店希望提供\uline{不同风味}的Pizza\\[3ex]

\begin{tikzpicture}
\node (PizzaStore) at (0,0) [shape=circle,minimum size=1.5cm, draw] {复用}; 
\draw[decoration={text along path, raise=-3pt, text={PizzaStore},
  text align={left}},decorate] (-1,0) arc(180:360:1 and 1) ;

\node (NYFactory) at (5,1.5) [shape=circle,fill=orange!80, minimum size=1.5cm, draw] {定制}; 
\draw[decoration={text along path, raise=-3pt, text={NYPizzaFactory},
  text align={center}},decorate] (4,1.5) arc(180:360:1 and 1);

\node (LAFactory) at (5,-1.5) [shape=circle,fill=cyan!20, minimum size=1.5cm, draw] {定制}; 
\draw[decoration={text along path, raise=-3pt, text={LAPizzaFactory},
  text align={center}},decorate] (4,-1.5) arc(180:360:1 and 1);

\draw [->, thick] (PizzaStore.north east) to [bend left=30] (NYFactory.west) ; 
\draw [->, thick] (PizzaStore.south east) to [bend right=30] (LAFactory.west) ; 
\end{tikzpicture}
\end{frame}

\begin{frame}[fragile]
\frametitle{各加盟店不同的工厂类}
\begin{Verbatim}[label=纽约风味的工厂]
NYPizzaFactory nyFactory = new NYPizzaFactory(); 
PizzaStore nyStore = new PizzaStore(nyFactory); 
nyStore.order("Veggie");
\end{Verbatim}

\vspace*{2ex}
\begin{Verbatim}[label=洛杉矶风味的工厂]
LAPizzaFactory laFactory = new LAPizzaFactory(); 
PizzaStore laStore = new PizzaStore(laFactory); 
laStore.order("Veggie");
\end{Verbatim}
\end{frame}

\begin{frame}
\frametitle{既有严格控制,又允许灵活性}
SimpleFactory工厂很容易被各加盟店采用,但他们只是利用SimpleFactory创建对象,自顾自地定义了自己的Pizza加工流程:例如使用第三方包装,忘记将Pizza切分,等等\\[2ex]

因此,我们需要的是一个\uline{框架},将Pizza店和Pizza创建\uline{绑定}到一起,并允许一定的\uline{灵活性}\\[2ex]

方法:取消独立的工厂类,在PizzaStore中设立一个\uline{工厂方法}
\end{frame}

\begin{frame}
\frametitle{让子类决定如何创建对象}
\begin{tikzpicture}
\tikzumlset{fill class=white}
\umlclass[type=abstract]{PizzaStore}{}{\umlvirt{createPizza()} \\ orderPizza()}

\umlclass[y=-4]{NYStylePizzaStore}{}{createPizza()}
\umlclass[x=5, y=-4]{LAStylePizzaStore}{}{createPizza()}

\umlinherit{NYStylePizzaStore}{PizzaStore}
\umlVHVinherit{LAStylePizzaStore}{PizzaStore}

\umlnote[x=5, width=5cm]{PizzaStore}{每个子类都重载createPizza()方法,并调用orderPizza()方法。可以将orderPizza()设置为final的,以防止子类重载}
\end{tikzpicture}
\end{frame}

{
\defverbatim[colored]{\verbnypizzastore}{%
  %\renewcommand{\baselinestretch}{1.1}
\begin{Verbatim}[label=纽约地区的PizzaStore, commandchars=\\\[\]]
public class NYPizzaStore extends PizzaStore { 
  \alert[Pizza createPizza(String item)] {
    if (item.equals("cheese")) {
      return new NYStyleCheesePizza();
    } else if (item.equals("veggie")) { 
      return new NYStyleVeggiePizza();
    } else if (item.equals("clam")) { 
      return new NYStyleClamPizza();
    } else if (item.equals("pepperoni")) { 
      return new NYStylePepperoniPizza();
    } else return null; 
  }
}
\end{Verbatim}
}

\defverbatim[colored]{\verbpizzastore}{%
  %\renewcommand{\baselinestretch}{1.1}
\begin{Verbatim}[label=PizzaStore, commandchars=\\\[\]]
public abstract class PizzaStore {
  \textblue[public final Pizza orderPizza(String type) ]{ 
    Pizza pizza;
    pizza = createPizza(type);
    pizza.prepare(); 
    pizza.bake(); 
    pizza.cut(); 
    pizza.box();
    return pizza; 
  }
  \alert[protected abstract Pizza createPizza(String type);]
  // other methods here 
}
\end{Verbatim}
}

\defverbatim[colored]{\verbpizza}{%
  %\renewcommand{\baselinestretch}{1.1}
\begin{Verbatim}[fontsize=\small, label=Pizza, commandchars=\\\[\]]
public abstract class Pizza { 
  String name;
  String dough;
  String sauce;
  ArrayList toppings = new ArrayList();
  void prepare() {
    System.out.println("Preparing " + name); 
    System.out.println("Tossing dough..."); 
    System.out.println("Adding sauce..."); 
    System.out.println("Adding toppings: "); 
    for (int i = 0; i < toppings.size(); i++) {
      System.out.println(" " + toppings.get(i)); 
    } 
  }
\end{Verbatim}
}

\defverbatim[colored]{\verbpizzab}{%
  %\renewcommand{\baselinestretch}{1.1}
\begin{Verbatim}[firstnumber=last, fontsize=\small,xrightmargin=-3pt,  
label=Pizza, commandchars=\\\[\]]
  void bake() {
    System.out.println("Bake for 25 minutes at 350");
  }

  void cut() {
    System.out.println("Cutting the pizza into diagonal slices");
  }

  void box() {
    System.out.println("Place pizza in official PizzaStore box");
  }

  public String getName() { return name; } 
}
\end{Verbatim}
}

\defverbatim[colored]{\verblapizza}{%
  %\renewcommand{\baselinestretch}{1.1}
\begin{Verbatim}[label=定义具体的Pizza子类, commandchars=\\\[\]]
public class LAStyleCheesePizza extends Pizza { 
  public LAStyleCheesePizza() {
    name = "LA Style Deep Dish Cheese Pizza"; 
    dough = "Extra Thick Crust Dough";
    sauce = "Plum Tomato Sauce";
    toppings.add("Shredded Mozzarella Cheese"); 
  }

  \alert[void cut()] {
    System.out.println("Cutting the pizza into 
      square slices");
  } 
}
\end{Verbatim}
}

\defverbatim[colored]{\verbclient}{%
  %\renewcommand{\baselinestretch}{1.1}
\begin{Verbatim}[label=测试]
public class PizzaTestDrive {
  public static void main(String[] args) {
    PizzaStore nyStore = new NYPizzaStore(); 

    Pizza pizza = nyStore.orderPizza("cheese"); 
    System.out.println("Ethan ordered a " + 
      pizza.getName() + "\n");
  }
}
\end{Verbatim}
}

\defverbatim[colored]{\verboutput}{%
  %\renewcommand{\baselinestretch}{1.1}
\begin{Verbatim}[label=输出结果]
%java PizzaTestDrive
Preparing NY Style Sauce and Cheese Pizza 
Tossing dough...
Adding sauce...
Adding toppings:
    Grated Regiano cheese
Bake for 25 minutes at 350
Cutting the pizza into diagonal slices
Place pizza in official PizzaStore box
Ethan ordered a NY Style Sauce and Cheese Pizza
%
\end{Verbatim}
}

\begin{frame}
\frametitle{采用工厂方法的新设计}
  \only<1> {
    \vspace*{1ex}
  \verbpizzastore
  }
  \only<2> {
    \vspace*{1ex}
  \verbnypizzastore
  }
  \only<3> {
    \vspace*{1ex}
  \verbpizza
  }
  \only<4> {
  \verbpizzab
  }
  \only<5> {
  \vspace*{2ex}
  \verblapizza
  }
  \only<6> {
  \verbclient
  }
  \only<7> {
  \verboutput
  }
\end{frame}
}

\begin{frame}
  \frametitle{认识工厂方法模式(Factory Method Pattern)}
  \only<1> {
    \begin{tikzpicture}
      \tikzumlset{fill class=white}

      \umlclass[type=abstract]{PizzaStore}{}{\umlvirt{createPizza()} \\
    orderPizza() }

    \umlclass[x=-3, y=-4]{NYPizzaStore}{}{createPizza()}
    \umlclass[x=3, y=-4]{LAPizzaStore}{}{createPizza()}
    \umlinherit{NYPizzaStore}{PizzaStore}
    \umlinherit{LAPizzaStore}{PizzaStore}

    \umlnote[x=4]{PizzaStore}{创建者(Creator)类}
    \end{tikzpicture}
  }

  \only<2> {
    \noindent\begin{tikzpicture}
      \tikzumlset{fill class=white}

      \umlemptyclass[type=abstract]{Pizza}

    \umlemptyclass[x=-3, y=-3]{NYStyleCheesePizza}
    \umlemptyclass[x=-2.5, y=-4]{NYStylePeeperoniPizza}
    \umlemptyclass[x=-2, y=-5]{NYStyleClamPizza}

    \umlemptyclass[x=3, y=-3]{LAStyleCheesePizza}
    \umlemptyclass[x=3.5, y=-4]{LAStylePeeperoniPizza}
    \umlemptyclass[x=4, y=-5]{LAStyleClamPizza}

    \umlinherit{NYStyleCheesePizza}{Pizza}
    \umlinherit{LAStyleCheesePizza}{Pizza}
    \umlnote[x=4]{Pizza}{产品(Product)类}
    \end{tikzpicture}
  }

  \only<3> {
    \noindent\scalebox{0.75}{
    \begin{tikzpicture}
      \tikzumlset{fill class=white}

      \umlclass[type=abstract]{PizzaStore}{}{\umlvirt{createPizza()} \\
    orderPizza() }
      \umlclass[x=-1.5, y=-4]{NYPizzaStore}{}{createPizza()}
      \umlclass[x=2, y=-4]{LAPizzaStore}{}{createPizza()}
      \umlinherit{NYPizzaStore}{PizzaStore}
      \umlinherit{LAPizzaStore}{PizzaStore}

      \umlemptyclass[x=8, type=abstract]{Pizza}
      \umlemptyclass[x=6, y=-4]{NYStyleCheesePizza}
      \umlemptyclass[x=10.5, y=-4]{LAStyleCheesePizza}
      \umlinherit{NYStyleCheesePizza}{Pizza}
      \umlinherit{LAStyleCheesePizza}{Pizza}

      \draw [-Stealth, thick, red] (LAPizzaStore.south) to [bend
      right=30](LAStyleCheesePizza.south) ;

      \draw [-Stealth, thick, red] (NYPizzaStore.south) to [bend
      right=30](NYStyleCheesePizza.south) ;
    \end{tikzpicture}
  }
  }

\end{frame}

\begin{frame}
  \frametitle{定义工厂方法模式(Factory Method Pattern)}

  \only<1> {
  \begin{block}{工厂方法模式}
    定义了一个创建对象的接口,但由子类\uline{决定}要实例化的类是哪一个。工厂方法
    让类把如何实例化对象交由子类决定
  \end{block}

  \begin{itemize}
    \item 工厂方法模式能够封装具体类型的实例化
    \item 所谓``\uline{由子类决定}'',并不是允许子类在运行时做决定,而是指在编写
      创建者类时,不需要知道实际创建的产品是什么
    \item 选择使用哪一个子类,自然就决定了实际创建的产品是什么
  \end{itemize}
}

  \only<2> {
    \begin{tikzpicture}
      \tikzumlset{fill class=white}

      \umlclass[type=abstract]{Creator}{}{factoryMethod() \\
      anOperation() }
      \umlclass[y=-4]{ConcreteCreator}{}{factoryMethod()}

      \umlemptyclass[type=abstract, x=6]{Product}
      \umlemptyclass[x=6, y=-4]{ConcreteProduct}
      
      \umlinherit{ConcreteProduct}{Product}
      \umlinherit{ConcreteCreator}{Creator}

      \umluniassoc{ConcreteCreator}{ConcreteProduct}
    \end{tikzpicture}
  }

\end{frame}

\begin{frame}
  \frametitle{关于工厂方法模式的讨论}
  \begin{itemize}[<+-|alert@+>] 
    \item [Q:] 当只有一个ConcreteFactory的时候,还有必要采用工厂方法模
  式吗?
    \item [Q:] 工厂方法和创建者是否总是抽象的?
    \item [Q:] 工厂方法的字符串参数总是必须的吗?传错了怎么办?
    \item [Q:] 简单工厂和创建者这两种方法的区别是什么?
    \item [Q:] 所谓的``工厂''究竟能带来什么好处?
    \item [Q:] 为什么``工厂''方法里仍然使用了具体化的类来实例化对象?
  \end{itemize}
\end{frame}

\begin{frame}
  \frametitle{最开始版本的PizzaStore中的对象依赖}
  \noindent\begin{tikzpicture}
    \tikzumlset{fill class=white}

    \umlsimpleclass{PizzaStore}
    \umlsimpleclass[x=-3, y=-1.7]{NYStyleCheesePizza}
    \umlsimpleclass[x=-2.5, y=-3]{NYStyleVeggiePizza}
    \umlsimpleclass[x=3, y=-1.7]{LAStyleCheesePizza}
    \umlsimpleclass[x=2.5, y=-3]{LAStyleVeggiePizza}
    \umlsimpleclass[x=-3, y=-4.3]{NYStylePepperoniPizza}
    \umlsimpleclass[x=-2.5, y=-5.5]{NYStyleClamPizza}
    \umlsimpleclass[x=3, y=-4.3]{LAStylePepperoniPizza}
    \umlsimpleclass[x=2.5, y=-5.5]{LAStyleClamPizza}

    \umluniassoc{PizzaStore}{NYStyleCheesePizza}
    \umluniassoc{PizzaStore}{NYStyleVeggiePizza}
    \umluniassoc{PizzaStore}{NYStylePepperoniPizza}
    \umluniassoc{PizzaStore}{NYStyleClamPizza}
    \umluniassoc{PizzaStore}{LAStyleCheesePizza}
    \umluniassoc{PizzaStore}{LAStyleVeggiePizza}
    \umluniassoc{PizzaStore}{LAStylePepperoniPizza}
    \umluniassoc{PizzaStore}{LAStyleClamPizza}

    \umlnote[x=-4, width=3.5cm]{PizzaStore}{被依赖的类有任何变化,它都要进行修改}
  \end{tikzpicture}
\end{frame}

\begin{frame}
  \frametitle{依赖反转原则}

  应该尽量减少对具体类的依赖:

  \begin{block}{设计原则: 依赖反转(Dependency Inversion Principle)}
    依赖抽象,不要依赖具体类
  \end{block}

  比较:\\
  \qquad \uline{对接口编程} \\[2ex]

  启发:\\
  \qquad 高层对象不应该依赖低层对象,双方都应依赖 \uline{抽象}

\end{frame}

\begin{frame}
  \frametitle{应用依赖反转原则}
  \noindent\begin{tikzpicture}
    \tikzumlset{fill class=white}

    \umlsimpleclass[y=1.2]{PizzaStore}
    \umlsimpleclass[type=abstract, y=-0.2, fill=yellow!80]{Pizza}
    \umlsimpleclass[x=-3, y=-1.7]{NYStyleCheesePizza}
    \umlsimpleclass[x=-2.5, y=-3]{NYStyleVeggiePizza}
    \umlsimpleclass[x=3, y=-1.7]{LAStyleCheesePizza}
    \umlsimpleclass[x=2.5, y=-3]{LAStyleVeggiePizza}
    \umlsimpleclass[x=-3, y=-4.3]{NYStylePepperoniPizza}
    \umlsimpleclass[x=-2.5, y=-5.5]{NYStyleClamPizza}
    \umlsimpleclass[x=3, y=-4.3]{LAStylePepperoniPizza}
    \umlsimpleclass[x=2.5, y=-5.5]{LAStyleClamPizza}

    \umluniassoc{PizzaStore}{Pizza}
    \umluniassoc{NYStyleCheesePizza}{Pizza}
    \umluniassoc{NYStyleVeggiePizza}{Pizza}
    \umluniassoc{NYStylePepperoniPizza}{Pizza}
    \umluniassoc{NYStyleClamPizza}{Pizza}
    \umluniassoc{LAStyleCheesePizza}{Pizza}
    \umluniassoc{LAStyleVeggiePizza}{Pizza}
    \umluniassoc{LAStylePepperoniPizza}{Pizza}
    \umluniassoc{LAStyleClamPizza}{Pizza}

    \umlnote[x=4]{Pizza}{抽象类,一种抽象}
  \end{tikzpicture}
\end{frame}

\begin{frame}
  \frametitle{理解``反转'': 以PizzaStore为例}
  \begin{itemize}[<+-|alert@+>] 
    \item [架构师] 如果要实现一个PizzaStore系统,首先会想到什么?
    \item [程序员] PizzaStore无非准备、烘烤、包装Pizza,所以要能生产
      各种Pizza:CheesePizza, VeggiePizza, ClamPizza, 等等
    \item [架构师] 没错。但就要了解各种具体的Pizza,进而对他们产生
      依赖。所以不妨\uline{反过来想},从Pizza着手,看能不能\uline{抽象}
    \item [程序员] 那样的话,CheesePizza, VeggiePizza, ClamPizza都是Pizza,因此
      他们应该共享一个\textit{Pizza}接口
  \end{itemize}
\end{frame}

\begin{frame}
  \frametitle{理解``反转'': 以PizzaStore为例}
  \begin{itemize}[<+-|alert@+>] 
    \item [架构师] 对。\textit{Pizza}就是抽象。现在可以重新设计了
    \item [程序员] 由于有了\textit{Pizza}抽象,所以可以据此设计
      PizzaStore,使之不依赖于具体的Pizza类了
    \item [架构师] 对。还需要一个工厂方法,来实例化各种具体的类。这
      样不同的具体类就只依赖一个抽象的\textit{Pizza},如此以来,该设计就把原来的依赖关系\uline{反转}了
  \end{itemize}
\end{frame}

\begin{frame}
  \frametitle{帮助遵循依赖反转原则}
  \begin{itemize}
    \item 不要让变量持有具体类的引用
      \begin{itemize}
        \item 不要用new,用工厂方法
      \end{itemize}
    \item 不要让类派生自具体类
      \begin{itemize}
        \item 让类派生自\uline{抽象}(接口或抽象类)
      \end{itemize}
    \item 不要覆盖父类中已实现的方法
      \begin{itemize}
        \item 如果覆盖了父类已实现的方法,那么父类就不是一个真正适合被继
          承的抽象。父类中已实现的方法,应该由所有的子类共享
      \end{itemize}
  \end{itemize}
  \framebox{\parbox{\hsize}{\centering 完全遵循这三条是很困难的,但应有
  意识尽量遵循}}
\end{frame}

\section{抽象工厂}

\begin{frame}
  \frametitle{PizzaStore新的需求}

  \alert{问题}: \\
  PizzaStore获得成功的一个重要原因是坚持使用新鲜、优质的原
  料。但某些加盟店为了降低成本,偷偷使用低质廉价的原料,减少馅的用量。长
  此以往必将对品牌造成伤害。 \\[3ex]
  \textblue{需求}:\\
  \uline{保证每家店用同样质量的原料}。可以设计一个工厂生产原料,然后发送
  给各店 \\
  \alert{另一个问题:} 不同店的相同名字的原料可能是两种东西,例如纽约和洛
  杉矶的红酱料,所以需要配备几组不同的原料

\end{frame}

\begin{frame}
  \frametitle{原料家族}
  \begin{tikzpicture}
    \tikzumlset{fill class=white, fill package=white}
    \begin{umlpackage}{New York}
      \umlsimpleclass{MarinaraSauce}
      \umlsimpleclass[y=-1]{FreshClams}
      \umlsimpleclass[y=-2]{ReggianoCheese}
      \umlsimpleclass[y=-3]{ThinCrustDough}
    \end{umlpackage}
    \begin{umlpackage}[x=5]{Los Angeles}
      \umlsimpleclass{Calamari}
      \umlsimpleclass[y=-1]{BruschettaSauce}
      \umlsimpleclass[y=-2]{GoatCheese}
      \umlsimpleclass[y=-3]{VeryThinCrust}
    \end{umlpackage}
  \end{tikzpicture}

  \vspace*{2ex}
  所有Pizza都由面团(dough)、酱料(sauce)、芝士(cheese)及若干佐料加工而成
  ,不同店对这些原料的实现可能是不同的
\end{frame}

{
\defverbatim[colored]{\verbingredientf}{%
  %\renewcommand{\baselinestretch}{1.1}
\begin{Verbatim}[label=定义原料工厂接口, commandchars=\\\<\>]
public interface PizzaIngredientFactory {
  public Dough createDough();
  public Sauce createSauce();
  public Cheese createCheese(); 
  public Veggies[] createVeggies(); 
  public Pepperoni createPepperoni(); 
  public Clams createClam();
}
\end{Verbatim}
}

\defverbatim[colored]{\verbnyingredientf}{%
  %\renewcommand{\baselinestretch}{1.2}
\begin{Verbatim}[label=纽约原料工厂部分代码, commandchars=\\\<\>]
public class NYPizzaIngredientFactory 
  implements PizzaIngredientFactory {
  public Dough createDough() { 
    return new ThinCrustDough();
  }
  public Veggies[] createVeggies() {
    Veggies veggies[] = { new Garlic(), new Onion(), 
        new Mushroom(), new RedPepper() }; 
    return veggies;
  }
  public Pepperoni createPepperoni() { 
    return new SlicedPepperoni();
  }
\end{Verbatim}
}

\defverbatim[colored]{\verbpizza}{%
  %\renewcommand{\baselinestretch}{1.1}
\begin{Verbatim}[label=新的Pizza类部分代码, commandchars=\\\^\$]
public abstract class Pizza { 
  String name;
  Dough dough;
  Sauce sauce; 
  Veggies veggies[]; 
  Cheese cheese; 
  Pepperoni pepperoni; 
  Clams clam;

  //将准备原料的prepare方法声明为抽象的
  \alert^abstract$ void prepare();
\end{Verbatim}
}



\defverbatim[colored]{\verbpizzab}{%
  %\renewcommand{\baselinestretch}{1.1}
\begin{Verbatim}[firstnumber=last, label=新的Pizza类部分代码,其他方法不变, commandchars=\\\^\$]
  void bake() {
    System.out.println("Bake for 25 minutes at 350");
  }
  void cut() {
    System.out.println("Cutting the pizza into 
      diagonal slices");
  }
  void box() {
    System.out.println("Place pizza in official 
      PizzaStore box");
  }
  // other methods such as setName, getName, toString
}
\end{Verbatim}
}


\defverbatim[colored]{\verbcheesepizza}{%
  %\renewcommand{\baselinestretch}{1.1}
\begin{Verbatim}[label=不同风味的Pizza做法一样,原料不同, commandchars=\\\^\$]
public class CheesePizza extends Pizza { 
  PizzaIngredientFactory ingf;
  public CheesePizza(PizzaIngredientFactory ingf) { 
    this.ingf = inf;
  }
  \textblue^void prepare()$ {
    System.out.println("Preparing " + name); 
    dough = ingf.createDough(); 
    sauce = ingf.createSauce(); 
    cheese = ingf.createCheese();
  } 
}
\end{Verbatim}
}

\defverbatim[colored]{\verbclampizza}{%
  %\renewcommand{\baselinestretch}{1.1}
\begin{Verbatim}[label=多一种原料的另一种Pizza, commandchars=\\\^\$]
public class ClamPizza extends Pizza { 
  PizzaIngredientFactory ingf;
  public ClamPizza(PizzaIngredientFactory ingf) { 
    this.ingf = ingf;
  }
  \textblue^void prepare()$ { 
    System.out.println("Preparing " + name); 
    dough = ingf.createDough(); 
    sauce = ingf.createSauce(); 
    cheese = ingf.createCheese(); 
    clam = ingf.createClam();
  } 
}
\end{Verbatim}
}

\defverbatim[colored]{\verbnypizzastore}{%
  %\renewcommand{\baselinestretch}{1.1}
\begin{Verbatim}[label=使用了本地原料工厂的PizzaStore, commandchars=\\\^\$]
public class NYPizzaStore extends PizzaStore {
  \textblue^protected Pizza createPizza(String item)$ {
    Pizza pizza = null; 
    PizzaIngredientFactory ingf = 
      new NYPizzaIngredientFactory(); 
    if (item.equals("cheese")) {
      pizza = new CheesePizza(ingredientFactory); 
      pizza.setName("New York Style Cheese Pizza");
    } else if (item.equals("veggie")) {
      // 制造各种本地风味的Pizza
    }
}
\end{Verbatim}
}
\begin{frame}
  \frametitle{构建原料工厂}

\only<1> {
  \verbingredientf
}

\only<2> {
  接下来需要
  \begin{enumerate}
    \item 为每个地区设计一个工厂类,实现PizzaIngredientFactory接口
    \item 实现该PizzaFactory需要的各种原料类,如ReggianoCheese, RedPeppers等
    \item 将新的原料工厂相关类和原有PizzaStore代码组合到一起
  \end{enumerate}
}

\only<3> {
  \vspace*{1ex}
  \verbnyingredientf
}

\only<4> {
  \verbpizza
}

\only<5> {
  \vspace*{1ex}
  \verbpizzab
}

\only<6> {
  \vspace*{1ex}
  \verbcheesepizza
}

\only<7> {
  \vspace*{1ex}
  \verbclampizza
}

\only<8> {
  \vspace*{1ex}
  \verbnypizzastore
}
\end{frame}
}

\begin{frame}
\frametitle{这些代码有何效果?}
\begin{itemize}
\item 引入新类型的工厂,即\uline{抽象工厂},来创建原料家族
\item 通过抽象工厂提供的\uline{接口},创建产品的代码将和实际工厂解耦,从而可以在不同上下文中实现不同工厂,制造不同产品
\item 由于代码从实际的产品中解耦,所以可以\uline{替换}不同的工厂来取得不同的行为
\item 客户代码始终\uline{保持不变}
\end{itemize}
\end{frame}

\begin{frame}
\frametitle{订购一个Pizza的顺序图}
\noindent\scalebox{0.75}{
\begin{tikzpicture}
\begin{umlseqdiag}
\umlobject{a}
\umlcreatecall[class=NYPizzaStore]{a}{n}
\begin{umlcall}[op=orderPizza("cheese")]{a}{n}
  \begin{umlcallself}[op=createPizza("cheese")]{n}
    \begin{umlcreatecall}[class=NYIngredientFactory]{n}{f}
    \end{umlcreatecall}
    \umlcreatecall[class=CheesePizza]{n}{p}
  \end{umlcallself}
  \begin{umlcall}[op=prepare()]{n}{p}
    \begin{umlcall}[op=createDough()]{p}{f}
    \end{umlcall}
    \begin{umlcall}[op=createSauce()]{p}{f}
    \end{umlcall}
    \begin{umlcall}[op=createCheese()]{p}{f}
    \end{umlcall}
  \end{umlcall}
  \begin{umlcall}[op=bake、cut、box]{n}{p}
  \end{umlcall}
\end{umlcall}
\end{umlseqdiag}
\end{tikzpicture}
}
\end{frame}

\begin{frame}
\frametitle{定义抽象工厂模式}

\only<1> {
\begin{block}{抽象工厂模式}
抽象工厂模式提供一个接口,用于创建相关或依赖对象的家族,而不需要明确指定具体类
\end{block}

抽象工厂允许客户使用抽象的接口来创建一组相关的产品,而不需要知道实际产出的具体产品是什么。如此一来,客户就从具体的产品中被\uline{解耦}
}

\only<2> {
\noindent\hspace*{-2ex}\scalebox{0.8}{
\begin{tikzpicture}
\tikzumlset{fill class=white}

\umlclass[type=interface]{AbstractFactory}{}{\umlvirt{createProductA()} \\
\umlvirt{createProductB()}}
\umlclass[y=3.5]{ConcreteFactoryA}{}{createProductA() \\
createProductB()}
\umlclass[y=-3.5]{ConcreteFactoryB}{}{createProductA() \\
createProductB()}

\umlimpl{ConcreteFactoryA}{AbstractFactory}
\umlimpl{ConcreteFactoryB}{AbstractFactory}

\umlsimpleclass[type=interface, x=7]{AbstractProductA}
\umlsimpleclass[x=7, y=2.5]{ProductA1}
\umlsimpleclass[x=7, y=-2.5]{ProductA2}
\umlimpl{ProductA1}{AbstractProductA}
\umlimpl{ProductA2}{AbstractProductA}

\umlsimpleclass[type=interface, x=11]{AbstractProductB}
\umlsimpleclass[x=11, y=2.5]{ProductB1}
\umlsimpleclass[x=11, y=-2.5]{ProductB2}
\umlimpl{ProductB1}{AbstractProductB}
\umlimpl{ProductB2}{AbstractProductB}

\umluniassoc[geometry=-|]{ConcreteFactoryA}{ProductA1}
\umluniassoc[geometry=-|]{ConcreteFactoryA}{ProductB1}

\umluniassoc[geometry=-|]{ConcreteFactoryB}{ProductA2}
\umluniassoc[geometry=-|]{ConcreteFactoryB}{ProductB2}

\umlsimpleclass[x=4, y=2.5]{Client}
\umluniassoc{Client}{AbstractFactory}
\umluniassoc{Client}{AbstractProductA}
\umluniassoc{Client}{AbstractProductB}

\end{tikzpicture}
}
}
\end{frame}

\begin{frame}
\frametitle{从PizzaStore的观点看抽象工厂}

\noindent\hspace*{-2ex}\scalebox{0.8}{
\begin{tikzpicture}
\tikzumlset{fill class=white}

\umlclass[type=interface]{PizzaIngredientFactory}{}{\umlvirt{createDough()} \\
\umlvirt{createSauce()}}
\umlclass[y=-3.5]{NYPizzaIngredientFactory}{}{createDough() \\
createSauce()}
\umlclass[y=3.5]{LAPizzaIngredientFactory}{}{createDough() \\
createSauce()}

\umlimpl{NYPizzaIngredientFactory}{PizzaIngredientFactory}
\umlimpl{LAPizzaIngredientFactory}{PizzaIngredientFactory}

\umlsimpleclass[type=interface, x=6.8]{Dough}
\umlsimpleclass[x=6.8, y=2.5]{ThickCrustDough}
\umlsimpleclass[x=6.8, y=-2.5]{ThinCrustDough}
\umlimpl{ThickCrustDough}{Dough}
\umlimpl{ThinCrustDough}{Dough}

\umlsimpleclass[type=interface, x=10.5]{Sauce}
\umlsimpleclass[x=10.5, y=2.5]{PlumTomatoSauce}
\umlsimpleclass[x=10.5, y=-2.5]{MarinaraSauce}
\umlimpl{PlumTomatoSauce}{Sauce}
\umlimpl{MarinaraSauce}{Sauce}

\umluniassoc[geometry=-|]{LAPizzaIngredientFactory}{ThickCrustDough}
\umluniassoc[geometry=-|]{LAPizzaIngredientFactory}{PlumTomatoSauce}

\umluniassoc[geometry=-|]{NYPizzaIngredientFactory}{ThinCrustDough}
\umluniassoc[geometry=-|]{NYPizzaIngredientFactory}{MarinaraSauce}

\umlsimpleclass[x=3.8, y=2.5]{Client}
\umluniassoc{Client}{PizzaIngredientFactory}
\umluniassoc{Client}{Dough}
\umluniassoc{Client}{Sauce}

\end{tikzpicture}
}
\end{frame}

\begin{frame}
\frametitle{关于工厂模式的小结}
\begin{itemize}
\item 所有的工厂都是用来\uline{封装}对象的创建,
\item 所有工厂模式都通过减少应用程序和具体类之间的依赖,促进\uline{松耦合}
\item 简单工厂不是真正的模式,但常常\uline{简单}有效
\item 工厂方法使用\uline{继承}
\item 抽象工厂使用对象\uline{聚合}
\item 依赖倒置原则指导我们编程时尽量\uline{依赖抽象}
\end{itemize}
\end{frame}

\begin{frame}
\frametitle{关于设计原则的小结}
  \begin{enumerate}
    \item 封装变化
    \item 多用聚合、少用继承
    \item 针对接口编程,不针对实现编程
    \item 尽最大可能将要交互的对象设计为松耦合的
    \item 对扩展开放,对修改封闭
    \item 依赖抽象,不要依赖具体类
  \end{enumerate}
\end{frame}

\end{document}
