\documentclass[compress]{beamer}

\usepackage[nofonts]{ctex}
\setCJKmainfont[ItalicFont={Kaiti SC}]{Kaiti SC}%
%\setCJKmainfont[ItalicFont={AR PL KaitiM GB}]{AR PL KaitiM GB}%
%\setCJKsansfont{WenQuanYi Zen Hei}% 文泉驿的黑体

\mode<beamer>
{
     \useinnertheme{rectangles}
     %\useoutertheme{infolines}
     %\useoutertheme{split}
     \usecolortheme{rose}
     \usecolortheme{seahorse}

     \setbeamertemplate{navigation symbols}{}%remove navigation symbols

%     \expandafter\def\expandafter\insertshorttitle\expandafter{%
%     \insertshorttitle\hfill%
%     \insertframenumber\,/\,\inserttotalframenumber}
}

\defbeamertemplate*{footline}{mytheme}
{
  \leavevmode%
  \hbox{%
  \begin{beamercolorbox}[wd=\paperwidth,ht=2.25ex,dp=1ex,center]{author in head/foot}%
    \raisebox{-1ex}{\includegraphics[width=3ex]{Overlays/logo.pdf}}%
    \hspace*{6ex}\insertframenumber{} / \inserttotalframenumber\hspace*{2ex} 
  \end{beamercolorbox}}%
  \vskip0pt%
}
\usebeamertemplate{mytheme}

\setbeamertemplate{headline}{}

%\setbeamercovered{transparent}

\mode<handout>
{
	\usetheme{default}
	\usepackage{pgfpages}
	\pgfpagesuselayout{4 on 1}[a4paper,landscape,border shrink=5mm]
}


\usepackage{amsmath,latexsym,amssymb,amsfonts,amsbsy}
\usepackage{graphicx}
\usepackage{array}
\usepackage{hyperref}
\usepackage{textpos}
\usepackage{ulem}
\usepackage{comment}
\usepackage{fancyvrb}
\fvset{frame=single, numbers=left, fontsize=\small}
\usepackage{tikz}
\usetikzlibrary{calc,arrows.meta, graphs, trees, shapes, positioning,
decorations.markings, patterns, decorations.text}
\usepackage{tikz-uml}

\newcommand{\romannumber}[1]{{\textrm{\uppercase\expandafter{\romannumeral
#1}}}}

\newcommand{\textblue}[1]{\textcolor{blue}{#1}}

\setbeamercolor{dblue}{fg=white,bg=gray!70!blue} % for beamercolorbox
 \newenvironment{pblock}{\begin{beamercolorbox}[rounded=true,
          shadow=false]{dblue}}{\end{beamercolorbox}}
 \newenvironment{bblock}{\begin{beamercolorbox}[rounded=true,
          shadow=false]{fg=black, bg=white}}{\end{beamercolorbox}}

\graphicspath{{figure/}}

%%%%%%%%%%%%%%%%%%%%%%%%%%%%%%%%%%%%%%%%%%%%%%%%%%%%%%%%%%%%%%%%%
%    body                                                       %
%%%%%%%%%%%%%%%%%%%%%%%%%%%%%%%%%%%%%%%%%%%%%%%%%%%%%%%%%%%%%%%%%


\begin{document}

%\AtBeginSubsection[]
\AtBeginSection[]
{ 
    \begin{frame}<beamer> 
		\frametitle{内容提要} 
		\tableofcontents[currentsection,currentsubsection] 
	\end{frame} 
} 

					
\title[适配器模式]{面向对象设计模式 ~~ 适配器与外观}

\author[曹东刚]
{曹东刚\\\href{mailto:caodg@pku.edu.cn}{caodg@pku.edu.cn}}

\institute[北京大学]{北京大学信息学院研究生课程 - 面向对象的分析与设计 \\
    \href{http://sei.pku.edu.cn/~caodg/course/oo}{http://sei.pku.edu.cn/\~{}caodg/course/oo}}

\date{}

\titlegraphic{\includegraphics[height=0.10\textwidth]{Overlays/logo.pdf}}

\begin{frame}[plain]
	\titlepage
\end{frame}

\setcounter{framenumber}{0}

%\includeonlyframes{current}

\section{适配器}

\begin{frame}
  \frametitle{面向对象适配器}
  \textblue{作用}: 将对象接口转换成另一种所期望的接口 \\[3ex]
  \only<1> {
  \begin{tikzpicture}
    %\draw [help lines, step=0.5cm] (0,0) grid (8, 3) ; 

    %\draw[thick] (0,0) rectangle (7,3) ;
    %\draw[thick] (2.5,3) -- (2.5,2) -- (3,1.5) -- (2.5,1) -- (2.5,0) ;
    %\draw[thick] (4.5,3) -- (4.5,2) arc [start angle =90, delta
    %angle=-180, radius=0.5cm] -- (4.5,0) ;

    \draw[] (0,0) node
    [xshift=1.25cm, yshift=1.5cm, text width=2ex, fill=white] {现有系统} |-
    ++(2.5,3) -- ++(0,-1) -- ++(0.5,-0.5) -- ++(-0.5,-0.5) |-
    cycle ;

    %\draw[pattern=dots] (3,3) node [xshift=1cm, yshift=-1.5cm, 
    %text width=2ex, fill=white] {适配器}  -- ++(0,-1) -- ++(0.5,-0.5) --
    %++(-0.5,-0.5) -- ++(0,-1) -- ++(2, 0) -- ++(0,1) arc [start angle=-90,
    %delta angle=180, radius=0.5cm] |- cycle ;

    \draw[] (5.5, 0) 
    node [xshift=1.25cm, yshift=1.5cm, text width=2ex, fill=white] {协
    作系统} --
    ++(0,1) arc [start angle=-90, delta angle=180, radius=0.5cm] |-
    ++(2.5, 1) |- cycle ;

    \node (note1) at (1.25, -1) {\uwave{不须改变代码}} ;
    \node (note2) at (6.75, -1) {\uwave{不须改变代码}} ;
    %\node (note3) at (4, -1) {\uline{新代码}} ;

    \draw [->] (note1) -- (1.25, -0.1) ;
    \draw [->] (note2) -- (6.75, -0.1) ;
    %\draw [->] (note3) -- (4, -0.1) ;
  \end{tikzpicture}
  }

  \only<2> {
  \begin{tikzpicture}
    %\draw [help lines, step=0.5cm] (0,0) grid (8, 3) ; 

    %\draw[thick] (0,0) rectangle (7,3) ;
    %\draw[thick] (2.5,3) -- (2.5,2) -- (3,1.5) -- (2.5,1) -- (2.5,0) ;
    %\draw[thick] (4.5,3) -- (4.5,2) arc [start angle =90, delta
    %angle=-180, radius=0.5cm] -- (4.5,0) ;

    \draw[] (0,0) node
    [xshift=1.25cm, yshift=1.5cm, text width=2ex, fill=white] {现有系统} |-
    ++(2.5,3) -- ++(0,-1) -- ++(0.5,-0.5) -- ++(-0.5,-0.5) |-
    cycle ;

    \draw[pattern=dots] (3,3) node [xshift=1cm, yshift=-1.5cm, text width=2ex, fill=white] {适配器}  -- ++(0,-1) -- ++(0.5,-0.5) --
    ++(-0.5,-0.5) -- ++(0,-1) -- ++(2, 0) -- ++(0,1) arc [start angle=-90,
    delta angle=180, radius=0.5cm] |- cycle ;

    \draw[] (5.5, 0) 
    node [xshift=1.25cm, yshift=1.5cm, text width=2ex, fill=white] {协
    作系统} --
    ++(0,1) arc [start angle=-90, delta angle=180, radius=0.5cm] |-
    ++(2.5, 1) |- cycle ;

    \node (note1) at (1.25, -1) {\uwave{不须改变代码}} ;
    \node (note2) at (6.75, -1) {\uwave{不须改变代码}} ;
    \node (note3) at (4, -1) {\uline{新代码}} ;

    \draw [->] (note1) -- (1.25, -0.1) ;
    \draw [->] (note3) -- (4, -0.1) ;
    \draw [->] (note2) -- (6.75, -0.1) ;
  \end{tikzpicture}
  }
\end{frame}

\begin{frame}
  \frametitle{鸭子适配器}
  如果它走路像鸭子,叫起来像鸭子,那么它: \\
  \quad \alt<1>{一定}{\alert{\sout{一定}}}\uncover<2>{\textbf{可能}} \\
  \quad\qquad 是一只\alt<1>{鸭子}{\alert{\sout{鸭子}}}\uncover<2>{\textbf{包装了鸭子
    适配器的\textblue{火鸡}\dots}}
\end{frame}

{
\defverbatim[colored]{\verbduck}{%
    %\renewcommand{\baselinestretch}{1.1}
\begin{Verbatim}[label=定义鸭子接口并实现一只鸭子子类, commandchars=\\\^\$]
public interface Duck { 
  public void quack(); 
  public void fly();
}

public class MallardDuck implements Duck { 
  public void quack() { 
    System.out.println("Quack"); 
  }
  public void fly() { 
    System.out.println("I’m flying"); 
  } 
}
\end{Verbatim}
}

\defverbatim[colored]{\verbturkey}{%
    %\renewcommand{\baselinestretch}{1.1}
\begin{Verbatim}[label=定义火鸡接口并实现一只火鸡子类]
public interface Turkey { 
  public void gobble(); 
  public void fly();
}

public class WildTurkey implements Turkey { 
  public void gobble() {
    System.out.println("Gobble gobble"); 
  }
  public void fly() {
    System.out.println("I'm flying a short distance");
  } 
}
\end{Verbatim}
}

\defverbatim[colored]{\verbadapter}{%
    %\renewcommand{\baselinestretch}{1.1}
\begin{Verbatim}[label=鸭子适配器, commandchars=\\\^\$]
public class TurkeyAdapter \uline^implements Duck$ { 
  Turkey turkey;
  public TurkeyAdapter(Turkey turkey) { 
    this.turkey = turkey;
  }
  \uline^public void quack()$ { 
    turkey.gobble(); 
  }
  \uline^public void fly()$ {
    for(int i=0; i < 5; i++)
      turkey.fly(); 
  }
}
\end{Verbatim}
}

\defverbatim[colored]{\verbtest}{%
    %\renewcommand{\baselinestretch}{1.1}
\begin{Verbatim}[label=测试代码, commandchars=\\\^\$]
public class DuckTestDrive {
  public static void main(String[] args) {
    WildTurkey turkey = new WildTurkey();
    Duck turkeyAdapter = new TurkeyAdapter(turkey);
    System.out.println("The TurkeyAdapter says...");
    testDuck(turkeyAdapter); 
  }

  static void testDuck(Duck duck) { 
    duck.quack();
    duck.fly(); 
  }
}
\end{Verbatim}
}

\defverbatim[colored]{\verboutput}{%
    %\renewcommand{\baselinestretch}{1.1}
\begin{Verbatim}[label=测试结果, commandchars=\\\^\$]
%java DuckTestDrive 
The TurkeyAdapter says... 
Gobble gobble
I'm flying a short distance 
I'm flying a short distance 
I'm flying a short distance 
I'm flying a short distance 
I'm flying a short distance 
\end{Verbatim}
}

\begin{frame}
  \frametitle{一个简易鸭子适配器的实现}
  \only<1> {
    \vspace*{1ex}
    \verbduck
  }
  \only<2> {
    \vspace*{1ex}
    \verbturkey
  }
  \only<3> {
    \vspace*{1ex}
    \verbadapter
  }
  \only<4> {
    \vspace*{1ex}
    \verbtest
  }
  \only<5> {
    \vspace*{1ex}
    \verboutput
  }
\end{frame}
}

\begin{frame}
\frametitle{鸭子适配器解析}
客户使用适配器的过程
\begin{enumerate}
\item 客户通过目标接口(Duck)调用适配器(TurkeyAdapter)的方法对适配器发出请求(quack(), fly())
\item 适配器使用被适配者接口(Turkey)把请求转换成被适配者的一个或多个调用请求(gobble(), fly())
\item 客户收到调用结果,但并未察觉这一切是适配器在起转换作用
\end{enumerate}
\end{frame}

\begin{frame}
\frametitle{思考}
\begin{itemize}
\item [Q:] <1-> 一个适配器需要做多少工作?和目标接口的大小有何关系?
\item [Q:] <2-> 一个适配器能封装几个类?
\item [Q:] <3-> 如果系统中新旧代码并存,旧的部分使用旧的目标接口,新的部分使用新的目标接口,会怎样?
\end{itemize}
\end{frame}

\begin{frame}
\frametitle{定义适配器模式}
\only<1> {
\begin{block}{适配器模式(Adapter Pattern) }
适配器模式将一个类的接口,转换成客户期望的另一个接口。适配器让原本接口不兼容的类可以合作无间
\end{block}
\vspace*{2ex}
客户和目标接口\uline{解耦}\\
适配器可以\uline{封装}目标接口的改变 \\

}

\only<2> {
\begin{tikzpicture}
\tikzumlset{fill class=white}
\umlemptyclass{Client}
\umlclass[type=interface, x=3]{Target}{}{request()}
\umlclass[x=3, y=-4]{Adapter}{}{request()}
\umlclass[x=7, y=-4]{Adaptee}{}{specificRequest()}

\umluniassoc{Client}{Target}
\umluniassoc{Adapter}{Adaptee}

\umlimpl{Adapter}{Target}
\end{tikzpicture}
}

\end{frame}

\begin{frame}
\frametitle{应用适配器模式的鸭子应用}
\begin{tikzpicture}
\tikzumlset{fill class=white}
\umlemptyclass{Client}
\umlclass[type=interface, x=3]{Duck}{}{quack() \\fly()}
\umlclass[x=3, y=-4]{TurkeyAdapter}{}{quack() \\fly()}
\umlclass[x=7, y=-4]{Turkey}{}{gobble() \\fly()}

\umluniassoc{Client}{Duck}
\umluniassoc{TurkeyAdapter}{Turkey}

\umlimpl{TurkeyAdapter}{Duck}
\end{tikzpicture}
\end{frame}

{
\defverbatim[colored]{\verbadapter}{%
    \renewcommand{\baselinestretch}{1.15}
\begin{Verbatim}[label=编写适配器代码, commandchars=\\\^\$]
public class EnumerationIterator implements Iterator {
  Enumeration enum;
  public EnumerationIterator(Enumeration enum) { 
    this.enum = enum;
  }
  public boolean hasNext() {
    return enum.hasMoreElements();
  }
  public Object next() {
    return enum.nextElement();
  }
  public void remove() {
    \alert^throw new UnsupportedOperationException()$;
  } 
}
\end{Verbatim}
}
\begin{frame}
\frametitle{真实世界的适配器}
\only<1> {
早期JDK的Collection类型(Vector、Stack、Hashtable等)都实现了一个elements()方法,该方法会返回一个Enumeration接口,通过该接口可以遍历集合内的所有元素 \\[3ex]
\centering\begin{tikzpicture}
\tikzumlset{fill class=white}
\umlclass[type=interface]{Enumeration}{}{hasMoreElements() \\nextElement()}
\end{tikzpicture}
}

\only<2> {
较新的JDK更新了Collection类型,开始使用Iterator接口,该接口和Enumeration接口相似,也可以遍历集合中的每个元素,但不同的是该迭代器还可以删除元素 \\[3ex]
\centering\begin{tikzpicture}
\tikzumlset{fill class=white}
\umlclass[type=interface]{Iterator}{}{hasNext() \\next() \\remove()}
\end{tikzpicture}
}

\only<3> {
\alert{\textbf{问题}}: \\
我们要调用遗留代码提供的服务,但遗留代码暴露的是老式的Enumeration接口,而我们希望在新的代码中只使用Iterator,应该如何处理? \\[2ex]
\begin{tikzpicture}
\tikzumlset{fill class=white}
\umlclass[type=interface]{Iterator}{}{hasNext() \\next() \\remove()}
\umlclass[type=interface, x=6]{Enumeration}{}{hasMoreElements() \\nextElement()}
\end{tikzpicture}
}

\only<4> {
\begin{tikzpicture}
\tikzumlset{fill class=white}
\umlclass[type=interface]{Iterator}{}{hasNext() \\next() \\remove()}
\umlclass[y=-4]{EnumerationIterator}{}{hasNext() \\next() \\\alert{remove()}}
\umlclass[type=interface, x=6, y=-4]{Enumeration}{}{hasMoreElements() \\nextElement()}

\umlimpl{EnumerationIterator}{Iterator}
\umluniassoc{EnumerationIterator}{Enumeration}

\umlnote[x=5, width=5cm]{EnumerationIterator}{a. 这是适配器,使得被适配者Enumeration变得像Iterator \\ b. 对于不支持的操作,可以抛出异常}
\end{tikzpicture}

}

\only<5> {
\vspace*{1ex}
\verbadapter
}

\end{frame}
}

\section{外观}

\begin{frame}
\frametitle{简化复杂接口--外观模式(Facade)}
适配器(Adapter)模式将一个类的接口转换成另一个符合客户期望的接口。基本做法是将不兼容的对象包装起来,变成兼容对象 \\[3ex]

有时候某些类的接口非常复杂、庞大,为了简化对类的使用,需要改变接口使之简化,这就需要\uline{外观}模式(Facade)

\end{frame}

{
\defverbatim[colored]{\verbbeforewatchdvd}{%
    \renewcommand{\baselinestretch}{1.15}
\begin{Verbatim}[label=观赏DVD需要完成任务的伪代码, commandchars=\\\^\$]
  popper.on(); 
  popper.pop();
  lights.dim(10); 
  screen.down();
  projector.on(); 
  projector.setInput(dvd); 
  projector.wideScreenMode();
  amp.on(); 
  amp.setDvd(dvd); 
  amp.setSurroundSound(); 
  amp.setVolume(5);
  dvd.on(); 
  dvd.play(movie);
\end{Verbatim}
}

\begin{frame}
\frametitle{家庭影院的例子}
\only<1> {
\hspace*{-7ex}\noindent\scalebox{0.65}{
\begin{tikzpicture}
\tikzumlset{fill class=white}

\umlclass{Amplifier}{tuner \\ dvdPlayer \\cdPlayer}{on() \\ off() \\setCd() \\setDvd() \\setStereoSound() \\
  setSurroundSound() \\ setTuner() \\setVolume()}

\umlclass[x=-4.5, y=1]{CdPlayer}{}{on() \\ off() \\eject() \\pause() \\ play() \\ stop()}

\umlclass[x=5.5, y=0.5]{DvdPlayer}{}{on() \\ off() \\eject() \\pause() \\ play() \\ stop() \\ setSurroundAudio() \\ setTwoChannelAudio()}

\umlclass[x=-4.5, y=-5]{Tuner}{}{on() \\ off() \\setAm() \\setFm() \\ setFrequency()}

\umlclass[x=5.5, y=-5]{Projector}{}{on() \\ off() \\tvMode() \\wideScreenMode()}

\umlclass[y=-6]{Screen}{}{up() \\ down()}
\umlclass[x=10.5]{PopcornPopper}{}{on() \\ off() \\pop()}
\umlclass[x=10.5, y=-4]{TheaterLights}{}{on() \\ off() \\dim()}

\umlassoc{Amplifier}{Tuner}
\umlassoc{Amplifier}{CdPlayer}
\umlassoc{Amplifier}{DvdPlayer}

\umlassoc{Projector}{DvdPlayer}

\end{tikzpicture}
}
}

\only<2> {
欣赏DVD影片之前必须执行的任务: \\[-3ex]
\begin{columns}[t]
\column{0.5\hsize}

\begin{enumerate}
\item 打开爆米花机
\item 开始爆米花
\item 将灯光调暗
\item 放下屏幕
\item 打开投影机
\item 将投影机的输入切换到DVD
\end{enumerate}

\column{0.5\hsize}
\begin{enumerate}
\setcounter{enumi}{6}
\item 将投影机设置在宽屏模式
\item 打开功放
\item 将功放的输入设置为DVD
\item 将功放设置为环绕立体声
\item 将功放音量调到中(5)
\item 打开DVD播放器
\item 开始播放DVD
\end{enumerate}
\end{columns}
}

\only<3> {
\vspace*{1ex}
\verbbeforewatchdvd
}

\only<4-5> {
还有:
\begin{itemize}
\item 看完电影后,需要把一切都关掉,怎么办?
\item 如果要听CD或广播,要怎么做?
\item 如果要升级影院系统,要重新学习一套全新的操作流程?
\end{itemize}
}

\only<5> {
\vspace*{4ex}
\framebox{\parbox{0.9\hsize}{\centering 让\uline{外观模式}来解决该问题} }
}

\end{frame}
}

{
\defverbatim[colored]{\verbhtfacade}{%
   \renewcommand{\baselinestretch}{1.2}
\begin{Verbatim}[label=家庭影院外观类设计, commandchars=\\\^\$]
public class HomeTheaterFacade { 
  Amplifier amp;
  Tuner tuner; 
  DvdPlayer dvd; 
  CdPlayer cd; 
  Projector projector; 
  TheaterLights lights; 
  Screen screen; 
  PopcornPopper popper;

  public HomeTheaterFacade(Amplifier amp, Tuner tuner,
    DvdPlayer dvd, CdPlayer cd, Projector projector, 
    Screen screen, TheaterLights lights, 
    PopcornPopper popper) { ... }
\end{Verbatim}
}

\defverbatim[colored]{\verbhtfacadeb}{%
   \renewcommand{\baselinestretch}{1.18}
\begin{Verbatim}[firstnumber=last, label=家庭影院外观类设计, commandchars=\\\^\$]
  \alert^public void watchMovie(String movie)$ { 
    System.out.println("Get ready to watch a movie..."); 
    popper.on();
    popper.pop();
    lights.dim(10);
    screen.down();
    projector.on();
    projector.wideScreenMode();
    amp.on();
    amp.setDvd(dvd);
    amp.setSurroundSound();
    amp.setVolume(5);
    dvd.on();
    dvd.play(movie);
  }
\end{Verbatim}
}

\defverbatim[colored]{\verbhtfacadec}{%
   %\renewcommand{\baselinestretch}{1.18}
\begin{Verbatim}[firstnumber=last, label=家庭影院外观类设计, commandchars=\\\^\$]
  \alert^public void endMovie()$ {
    System.out.println("Shutting movie theater down..."); 
    popper.off();
    lights.on();
    screen.up();
    projector.off();
    amp.off();
    dvd.stop();
    dvd.eject();
    dvd.off();
  }
\end{Verbatim}
}

\defverbatim[colored]{\verbhttest}{%
   %\renewcommand{\baselinestretch}{1.18}
\begin{Verbatim}[label=开始测试家庭影院外观, commandchars=\\\^\$]
public class HomeTheaterTestDrive {
  public static void main(String[] args) {
    // instantiate components here

    HomeTheaterFacade homeTheater =
      new HomeTheaterFacade(amp, tuner, dvd, cd,
        projector, screen, lights, popper);

    homeTheater.\alert^watchMovie("Raiders of the Lost Ark")$;
    homeTheater.\alert^endMovie()$; 
  }
}
\end{Verbatim}
}

\defverbatim[colored]{\verbhtoutput}{%
   \renewcommand{\baselinestretch}{1.35}
\begin{Verbatim}[label=运行 java HomeTheaterTestDrive 测试结果, commandchars=\\\^\$, fontsize=\tiny]
Get ready to watch a movie... 
Popcorn Popper on
Popcorn Popper popping popcorn! 
Theater Ceiling Lights dimming to 10% 
Theater Screen going down
Top-O-Line Projector on
Top-O-Line Projector in widescreen mode (16x9 aspect ratio) 
Top-O-Line Amplifier on
Top-O-Line Amplifier setting DVD player to Top-O-Line DVD Player 
Top-O-Line Amplifier surround sound on (5 speakers, 1 subwoofer) 
Top-O-Line Amplifier setting volume to 5
Top-O-Line DVD Player on
Top-O-Line DVD Player playing "Raiders of the Lost Ark" 
Shutting movie theater down...
Popcorn Popper off
Theater Ceiling Lights on
Theater Screen going up
Top-O-Line Projector off
Top-O-Line Amplifier off
Top-O-Line DVD Player stopped "Raiders of the Lost Ark" 
Top-O-Line DVD Player eject
Top-O-Line DVD Player off
\end{Verbatim}
}

\begin{frame}
\frametitle{为家庭影院设计外观}
\only<1> {
\hspace*{-7ex}\noindent\scalebox{0.6}{
\begin{tikzpicture}
\tikzumlset{fill class=white , fill package=white}

\begin{umlpackage}{HomeTheater}
\umlclass{Amplifier}{tuner \\ dvdPlayer \\cdPlayer}{on() \\ off() \\setCd() \\setDvd() \\setStereoSound() \\
  setSurroundSound() \\ setTuner() \\setVolume()}

\umlclass[x=-4.5, y=1]{CdPlayer}{}{on() \\ off() \\eject() \\pause() \\ play() \\ stop()}

\umlclass[x=5.5, y=0.5]{DvdPlayer}{}{on() \\ off() \\eject() \\pause() \\ play() \\ stop() \\ setSurroundAudio() \\ setTwoChannelAudio()}

\umlclass[x=-4.5, y=-5]{Tuner}{}{on() \\ off() \\setAm() \\setFm() \\ setFrequency()}

\umlclass[x=5.5, y=-5]{Projector}{}{on() \\ off() \\tvMode() \\wideScreenMode()}

\umlclass[y=-6]{Screen}{}{up() \\ down()}
\umlclass[x=10.5, y=-2]{PopcornPopper}{}{on() \\ off() \\pop()}
\umlclass[x=10.5, y=-6]{TheaterLights}{}{on() \\ off() \\dim()}

\umlassoc{Amplifier}{Tuner}
\umlassoc{Amplifier}{CdPlayer}
\umlassoc{Amplifier}{DvdPlayer}

\umlassoc{Projector}{DvdPlayer}
\end{umlpackage}

\umlclass[x=11.5, y=2.5, fill=yellow!20, text=red]{HomeTheaterFacade}{}{watchMovie() \\ endMovie() \\listenToCd() \\ endCd() \\ listenToRadio() \\ endRadio() }

\end{tikzpicture}
}
}

\only<2> {
\vspace*{1ex}
\verbhtfacade
}

\only<3> {
\vspace*{1ex}
\verbhtfacadeb
}

\only<4> {
\verbhtfacadec
}

\only<5> {
\verbhttest
}

\only<6> {
\vspace*{1ex}
\verbhtoutput
}

\end{frame}
}


\begin{frame}
\frametitle{定义外观模式}
\only<1> {
\begin{block}{外观模式(Facade Pattern)}
外观模式提供了一个统一的接口,用来访问子系统中的一群接口。外观定义了一个高层接口,让子系统更容易使用
\end{block}

外观模式可使客户和子系统之间\uline{松耦合} \\
外观模式可帮助遵守``\uline{最少知识原则}'' 
}

\only<2> {
\begin{tikzpicture}
\tikzumlset{fill class=white, fill package=white}

\umlsimpleclass[fill=yellow!40]{Facade}
\umlsimpleclass[x=-4]{Client}
\begin{umlpackage}[y=-3]{Subsystem}
\umlsimpleclass{A}
\umlsimpleclass[x=-2, y=-1]{B}
\umlsimpleclass[x=3, y=-0.5]{C}
\umlsimpleclass[x=2, y=-3]{D}
\umlsimpleclass[x=1, y=-1.5]{E}
\umlsimpleclass[x=-1, y=-3]{F}
\umlassoc{A}{B}
\umlassoc{A}{C}
\umlassoc{A}{E}
\umlassoc{B}{F}
\umlassoc{B}{D}
\umlassoc{B}{D}
\umlassoc{E}{F}
\umlassoc{E}{D}
\umlassoc{C}{D}
\end{umlpackage}

\umlassoc{Facade}{A}
\umlassoc{Facade}{B}
\umlassoc{Facade}{C}

\umluniassoc{Client}{Facade}
\end{tikzpicture}
}

\end{frame}

{
\defverbatim[colored]{\verbcoupled}{%
   %\renewcommand{\baselinestretch}{1.35}
\begin{Verbatim}[commandchars=\\\^\$]
public float getTemp() {
  return station.getThermometer().getTemperature(); 
}
\end{Verbatim}
}

\begin{frame}
\frametitle{最少知识原则}
\only<1> {
最少知识(Least Knowledge)原则告诉我们要减少对象之间的交互,只留下几个``\textbf{密友}'' \\[1ex]
\begin{block}{最少知识设计原则}
只和密友交谈
\end{block}

该原则希望在设计系统时,不要让太多的类耦合到一起,免得修改系统中的一部分,会影响到其他部分 \\
如果许多类彼此依赖,则维护成本就会升高
}

\only<2> {
例: \\[2ex]
\verbcoupled
\begin{itemize}
\item [Q:] 该段代码耦合了多少类? 
\item [Q:] 哪些类是该代码所属类的朋友? 
\end{itemize}
}
\end{frame}
}

{
\defverbatim[colored]{\verbcoupled}{%
   %\renewcommand{\baselinestretch}{1.35}
\begin{Verbatim}[label=不采用该原则的代码, commandchars=\\\^\$]
public float getTemp() {
  Thermometer thermometer = station.getThermometer();
  return thermometer.getTemperature();
}
\end{Verbatim}
}

\defverbatim[colored]{\verbcoupledno}{%
   %\renewcommand{\baselinestretch}{1.35}
\begin{Verbatim}[label=采用该原则的代码, commandchars=\\\^\$]
public float getTemp() {
  return station.getTemperature();
}
\end{Verbatim}
}

\defverbatim[colored]{\verbcar}{%
   %\renewcommand{\baselinestretch}{1.35}
\begin{Verbatim}[label=另一个遵守最少知识原则的代码, commandchars=\\\^\$]
public class Car { 
  Engine \uline^engine$;
  public Car() { // initialize engine, etc. }

  void start(Key \uline^key$) { 
    Doors \uline^doors$ = new Doors();
    boolean authorized = \fbox^key.turns()$;
    if (authorized) {
      \fbox^engine.start()$;
      \fbox^updateDashboardDisplay()$;
      \fbox^doors.lock()$;
    }
  }
\end{Verbatim}
}

\begin{frame}
\frametitle{如何做到``只和密友''交谈?}
\only<1> {
原则上,一个对象应该只和如下对象交互
\begin{itemize}
\item 该对象本身
\item 做为方法参数传递进来的对象
\item 该对象创建或实例化的对象
\item 该对象实例变量所引用的对象(关联对象,聚合对象)
\end{itemize}
}

\only<2-3> {
\verbcoupled
}

\only<3> {
\centering \alert{$\Downarrow$}
\vspace*{2ex}
\verbcoupledno
}

\only<4> {
\vspace*{1ex}
\verbcar
}

\end{frame}
}

\section{小结}

\begin{frame}
\frametitle{关于适配器模式和外观模式的小结}
\begin{itemize}
\item 当要使用一个类而该类接口不合期望时,使用适配器模式
\item 当要简化一个很大的接口或一群复杂接口时,使用外观模式
\item 适配器改变接口
\item 外观封装子系统,将客户从子系统解耦
\item 可以为子系统实现多个外观
\item 适配器将对象包装起来改变其接口;装饰者将对象包装起来增加新行为;外观对象将一群对象包装起来以简化接口
\end{itemize}
\end{frame}


\begin{frame}
\frametitle{关于设计原则的小结}
  \begin{enumerate}
    \item 封装变化
    \item 多用聚合、少用继承
    \item 针对接口编程,不针对实现编程
    \item 尽最大可能将要交互的对象设计为松耦合的
    \item 对扩展开放,对修改封闭
    \item 依赖抽象,不要依赖具体类
    \item 只和朋友交谈
  \end{enumerate}
\end{frame}

\end{document}
