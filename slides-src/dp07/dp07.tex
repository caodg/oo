\documentclass[compress]{beamer}

\usepackage[nofonts]{ctex}
\setCJKmainfont[ItalicFont={Kaiti SC}]{Kaiti SC}%
%\setCJKmainfont[ItalicFont={AR PL KaitiM GB}]{AR PL KaitiM GB}%
%\setCJKsansfont{WenQuanYi Zen Hei}% 文泉驿的黑体

\mode<beamer>
{
     \useinnertheme{rectangles}
     %\useoutertheme{infolines}
     %\useoutertheme{split}
     \usecolortheme{rose}
     \usecolortheme{seahorse}

     \setbeamertemplate{navigation symbols}{}%remove navigation symbols

%     \expandafter\def\expandafter\insertshorttitle\expandafter{%
%     \insertshorttitle\hfill%
%     \insertframenumber\,/\,\inserttotalframenumber}
}

\defbeamertemplate*{footline}{mytheme}
{
  \leavevmode%
  \hbox{%
  \begin{beamercolorbox}[wd=\paperwidth,ht=2.25ex,dp=1ex,center]{author in head/foot}%
  \hspace*{6ex}\insertframenumber{} / \inserttotalframenumber\hspace*{2ex} 
  \end{beamercolorbox}}%
  \vskip0pt%
}
\usebeamertemplate{mytheme}

\setbeamertemplate{headline}{}

%\setbeamercovered{transparent}

\mode<handout>
{
	\usetheme{default}
	\usepackage{pgfpages}
	\pgfpagesuselayout{2 on 1}[a4paper,landscape,border shrink=5mm]
}


\usepackage{amsmath,latexsym,amssymb,amsfonts,amsbsy}
\usepackage{graphicx}
\usepackage{array}
\usepackage{hyperref}
\usepackage{textpos}
\usepackage{ulem}
\usepackage{comment}
\usepackage{fancyvrb}
\fvset{frame=single, numbers=left, fontsize=\small}
\usepackage{tikz}
\usetikzlibrary{calc,arrows.meta, graphs, trees, shapes, positioning, decorations.markings, intersections, decorations.text}
\usepackage{tikz-uml}

\newcommand{\romannumber}[1]{{\textrm{\uppercase\expandafter{\romannumeral
#1}}}}

\newcommand{\textblue}[1]{\textcolor{blue}{#1}}

\setbeamercolor{dblue}{fg=white,bg=gray!70!blue} % for beamercolorbox
 \newenvironment{pblock}{\begin{beamercolorbox}[rounded=true,
          shadow=false]{dblue}}{\end{beamercolorbox}}
 \newenvironment{bblock}{\begin{beamercolorbox}[rounded=true,
          shadow=false]{fg=black, bg=white}}{\end{beamercolorbox}}

\graphicspath{{figure/}}

%%%%%%%%%%%%%%%%%%%%%%%%%%%%%%%%%%%%%%%%%%%%%%%%%%%%%%%%%%%%%%%%%
%    body                                                       %
%%%%%%%%%%%%%%%%%%%%%%%%%%%%%%%%%%%%%%%%%%%%%%%%%%%%%%%%%%%%%%%%%


\begin{document}

%\AtBeginSubsection[]
\AtBeginSection[]
{ 
    \begin{frame}<beamer> 
		\frametitle{内容提要} 
		\tableofcontents[currentsection,currentsubsection] 
	\end{frame} 
} 

					
\title[迭代器与组合]{面向对象设计模式 ~~ 迭代器与组合}

\author[曹东刚]
{曹东刚\\\href{mailto:caodg@pku.edu.cn}{caodg@pku.edu.cn}}

\institute[北京大学]{北京大学信息学院研究生课程 - 面向对象的分析与设计 \\
    \href{http://sei.pku.edu.cn/~caodg/course/oo}{http://sei.pku.edu.cn/\~{}caodg/course/oo}}

\date{}

\titlegraphic{\includegraphics[height=0.10\textwidth]{Overlays/logo.pdf}}

\begin{frame}[plain]
	\titlepage
\end{frame}

\setcounter{framenumber}{0}

%\includeonlyframes{current}

\begin{frame}
  \frametitle{问题的提出}
  有很多种方法可以把对象堆起来形成一个集合(collection),如数组、堆栈、列
  表或散列表等。\\[3ex]

  \alert{问题:}如何遍历集合中的对象,而无需关心集合的具体实现? \\[3ex]

  \framebox{\parbox{0.9\hsize}{\centering \textblue{答案}:迭代器模式(Iterator
  Pattern)}}
\end{frame}

\begin{frame}
  \frametitle{例:处理分别由数组和ArrayList实现的遗产系统}
  \alert{需求}:某遗产系统有两个构件,一个处理早餐订餐业务,一个处理午餐订餐业务,分别
  用ArrayList和数组记录菜单项。现在要实现一个统一的订餐系统,要求能够按
  照顾客的要求打印出定制的菜单,应如何设计? \\[2ex]
  \alert{难点}:程序员不希望改变两个构件的现有存储方式,因为改动的代价太
  大\\[2ex]
\end{frame}

\begin{frame}[fragile]
  \frametitle{现有系统:相同的MenuItem}
\begin{Verbatim}[label=早餐和午餐订餐系统的菜单项是相同的]
public class MenuItem { 
  String name;
  String description; 
  boolean vegetarian; 
  double price;
  pubic MenuItem(String name, String description,
    boolean vegetarian, double price){ //init
  }
  public String getName() { return name; }
  public String getDescription() { return description; }
  public double getPrice() { return price; }
  public boolean isVegetarian() { return vegetarian; } 
}
\end{Verbatim}
\end{frame}

\begin{frame}[fragile]
  \frametitle{两个构件各自不同的遍历方法}
\begin{Verbatim}[label=早餐项遍历需要使用ArrayList的size和get方法, 
  commandchars=\\\^\$]
\alert^ArrayList$ breakfastItems = breakfMenu.getMenuItems() ;
for (int i = 0; i < breakfastItems.size(); i++) {
  MenuItem menuItem = (MenuItem)breakfastItems.get(i); 
}
\end{Verbatim}

\vspace*{3ex}

\begin{Verbatim}[label=午餐项遍历需要使用数组的length字段和下标操作, 
  commandchars=\\\^\$]
\alert^MenuItem[]$ lunchItems = dinerMenu.getMenuItems() ;
for (int i = 0; i < lunchItems.length; i++) {
  MenuItem menuItem = lunchItems[i]; 
}
\end{Verbatim}
\end{frame}

\begin{frame}
\frametitle{两个构件不同的实现方式带来的问题}

\begin{itemize}
\item 统一订餐系统必须针对两个构件的具体实现编程,而不是针对接口
\item 统一订餐系统需要知道每个菜单如何表达内部的菜单项集合细节,违反了封装原则
\end{itemize}

  \textblue{如何解决}:
\begin{enumerate}
\item 让各个菜单实现一个相同的遍历内部元素的接口 
  \item 将遍历(系统中变化的部分)封装
\end{enumerate}
\end{frame}

\begin{frame}[fragile]
  \frametitle{创建迭代器进行遍历}
\begin{Verbatim}[label=遍历ArrayList] 
Iterator iterator = breakfastMenu.createIterator(); 
while (iterator.hasNext()) {
  MenuItem menuItem = (MenuItem)iterator.next();
}
\end{Verbatim}
\vspace*{3ex}
\begin{Verbatim}[label=遍历数组] 
Iterator iterator = lunchMenu.createIterator();
while (iterator.hasNext()) {
  MenuItem menuItem = (MenuItem)iterator.next(); 
}
\end{Verbatim}

\end{frame}

\begin{frame}
\frametitle{认识迭代器模式(Iterator Pattern)}
\begin{tikzpicture}
\tikzumlset{fill class=white}
\umlclass[type=interface]{Iterator}{}{\umlvirt{hasNext()} \\
  \umlvirt{next()}
  }
\umlclass[y=-4]{DinnerMenuIterator}{}{hasNext() \\
  next()
  }
\umlimpl{DinnerMenuIterator}{Iterator}
\umlnote[x=5]{Iterator}{定义了访问集合中元素的接口}
\umlnote[y=-4, x=5]{DinnerMenuIterator}{具体的对象集合知道如何遍历自己的元素}
\end{tikzpicture}

\end{frame}

\begin{frame}[fragile]
\frametitle{在餐厅菜单中加入迭代器}
   \renewcommand{\baselinestretch}{1.2}
\begin{Verbatim}[label=晚餐菜单迭代器:遍历数组]
public class DinerMenuIterator implements Iterator { 
  MenuItem[] items;
  int position = 0;
  public Object next() {
    MenuItem menuItem = items[position]; 
    position = position + 1;
    return menuItem;
  }
  public boolean hasNext() {
    if (position >= items.length || items[position] == null) 
      return false; 
    else 
      return true; 
}
\end{Verbatim}
\end{frame}

\begin{frame}[fragile]
\frametitle{改写晚餐菜单}
\begin{Verbatim}[commandchars=\\\^\$]
public class DinerMenu implements Menu { 
  static final int MAX_ITEMS = 6;
  int numberOfItems = 0;
  MenuItem[] menuItems;
  \sout^public MenuItem[] getMenuItems() {$
    \sout^return menuItems;$
  \sout^}$
  \uline^public Iterator createIterator() {$
    \uline^return new DinerMenuIterator(menuItems);$
  \uline^}$
  // other methods here 
}
\end{Verbatim}
\end{frame}

\begin{frame}
\frametitle{设计的类图}
\scalebox{0.85}{
\begin{tikzpicture}
\tikzumlset{fill class=white}

\umlclass{Waitress}{}{printMenu()}
\umlclass[x=-4,y=1.5]{BreakfastMenu}{menuItems}{createIterator()}
\umlclass[x=-4,y=-1.5]{DinerMenu}{menuItems}{createIterator()}

\umlclass[x=4, type=abstract]{Iterator}{}{\umlvirt{hasNext()} \\
  \umlvirt{next()}}
\umlclass[x=4, y=3.2]{BreakfastMenuIterator}{}{hasNext() \\
  next()}
\umlclass[x=4, y=-3.2]{DinerMenuIterator}{}{hasNext() \\
  next()}
\umlimpl{BreakfastMenuIterator}{Iterator}
\umlimpl{DinerMenuIterator}{Iterator}

\umluniassoc{Waitress}{Iterator}
\umluniassoc{Waitress}{BreakfastMenu}
\umluniassoc{Waitress}{DinerMenu}

\umluniassoc[geometry=|-]{DinerMenu}{DinerMenuIterator}
\umluniassoc[geometry=|-]{BreakfastMenu}{BreakfastMenuIterator}
\end{tikzpicture}
}

\end{frame}

\begin{frame}[fragile]
\frametitle{Java中的迭代器java.util.Iterator}

\begin{center}
\begin{tikzpicture}
\umlclass[fill=white, type=interface]{Iterator}{}{\umlvirt{hasNext()} \\
  \umlvirt{next()} \\
  \textblue{\umlvirt{remove()}} }

\umlnote[x=6, width=6cm]{Iterator}{
~remove()~方法是可选的,如果不打算支持,可以直接扔出异常:
~java.lang.UnsupportedOperationException~
}
\end{tikzpicture}
\end{center}
\end{frame}

\begin{frame}
\frametitle{改进的设计:使用JDK的Iterator}
\only<1>{
\scalebox{0.70}{
\begin{tikzpicture}
\tikzumlset{fill class=white}

\umlclass{Waitress}{}{printMenu()}
\umlclass[x=-4, type=interface]{Menu}{}{\umlvirt{createIterator()}}
\umlclass[x=-4,y=3.7]{BreakfastMenu}{menuItems}{createIterator()}
\umlclass[x=-4,y=-3.7]{DinerMenu}{menuItems}{createIterator()}
\umlimpl{BreakfastMenu}{Menu}
\umlimpl{DinerMenu}{Menu}

\umlclass[x=4, type=abstract]{Iterator}{}{\umlvirt{hasNext()} \\
  \umlvirt{next()} \\
  \umlvirt{remove()}}
\umlclass[x=4, y=3.7, fill=gray!20]{BreakfastMenuIterator}{}{hasNext() \\
  next() \\
  remove()}
\umlclass[x=4, y=-3.7]{DinerMenuIterator}{}{hasNext() \\
  next() \\
  remove()}
\umlimpl{BreakfastMenuIterator}{Iterator}
\umlimpl{DinerMenuIterator}{Iterator}

\umluniassoc{Waitress}{Iterator}
\umluniassoc{Waitress}{Menu}

\umluniassoc{DinerMenu}{DinerMenuIterator}
%\umluniassoc{BreakfastMenu}{BreakfastMenuIterator}

\end{tikzpicture}
}
}

\only<2>{
\scalebox{0.80}{
\begin{tikzpicture}
\tikzumlset{fill class=white}

\umlclass{Waitress}{}{printMenu()}
\umlclass[x=-4, type=interface]{Menu}{}{\umlvirt{createIterator()}}
\umlclass[x=-4,y=3.2]{BreakfastMenu}{menuItems}{createIterator()}
\umlclass[x=-4,y=-3.7]{DinerMenu}{menuItems}{createIterator()}
\umlimpl{BreakfastMenu}{Menu}
\umlimpl{DinerMenu}{Menu}

\umlclass[x=4, type=abstract]{Iterator}{}{\umlvirt{hasNext()} \\
  \umlvirt{next()} \\
  \umlvirt{remove()}}
\umlclass[x=4, y=-3.7]{DinerMenuIterator}{}{hasNext() \\
  next() \\
  remove()}
\umlimpl{DinerMenuIterator}{Iterator}

\umluniassoc{Waitress}{Iterator}
\umluniassoc{Waitress}{Menu}

\umluniassoc{DinerMenu}{DinerMenuIterator}

\umlnote[x=1, y=3.2, width=4cm]{BreakfastMenu}{menuItems是ArrayList,可以直接使用Java的内置Iterator}
\end{tikzpicture}
}
}
\end{frame}



\begin{frame}
  \frametitle{定义迭代器模式}
  \only<1> {
  \begin{block}{迭代器模式(Iterator Pattern)}
    迭代器模式提供了一种方法,能够顺序访问一个集合对象中的各个元素,而又
    不暴露其内部实现细节
  \end{block}

  \vspace*{2ex}
  将\uline{遍历集合中的元素}这件任务交给Iterator,而不是交给集合,简化了集合类的设计,并使得各自的责任明确
  }

  \only<2> {
    \noindent\begin{tikzpicture}
      \tikzumlset{fill class=white}
      \umlemptyclass{Client}
      \umlclass[x=3.5, type=interface]{Iterator}{}{\umlvirt{hasNext()} \\
        \umlvirt{next()} \\
        \umlvirt{remove()} 
      }
      \umlclass[x=3.5, y=-4]{ConcreteIterator}{}{hasNext() \\
        next() \\
        remove() 
      }
      \umlclass[x=-4,
      type=interface]{Aggregate}{}{\umlvirt{createIterator()}}
      \umlclass[x=-4, y=-4]{ConcreteAggregate}{}{createIterator()}

      \umlimpl{ConcreteAggregate}{Aggregate}
      \umlimpl{ConcreteIterator}{Iterator}
      \umluniassoc{Client}{Aggregate}
      \umluniassoc{Client}{Iterator}
      \umluniassoc{ConcreteAggregate}{ConcreteIterator}

    \end{tikzpicture}
  }
\end{frame}

\begin{frame}
  \frametitle{设计原则:单一责任(Single Responsibility)}
  \begin{block}{单一责任原则}
    一个类应该只有一个引起变化的原因
  \end{block}

\vspace*{2ex}

\begin{itemize}
\item \uline{管理}集合里的元素和\uline{遍历}所有元素是\uwave{两件事}
\item 类所承担的每个责任都是潜在的代码变动源头。多个责任意味着多个变动之处
\item 每个类只应当承担单一责任
\end{itemize}
\end{frame}

\begin{frame}
\frametitle{分析类的责任}
\begin{tikzpicture}
\tikzumlset{fill class=white}

\umlclass{Game}{}{login() \\
  signup() \\
  move() \\
  fire() \\
  rest() }
\umlclass[x=3.5]{Person}{}{setName() \\
  setAddress() \\
  setPhoneNumber() \\
  save() \\
  load()}
\umlclass[x=7.5]{DeckOfCards}{} {hasNext() \\
  next() \\
  remove() \\
  addCard() \\
  removeCard() \\
  shuffle() }
\end{tikzpicture}
\end{frame}

%\subsection{组合模式}

\begin{frame}
  \frametitle{新需求: 支持菜单中的菜单}

\begin{tikzpicture} [
    level distance=2cm,
    level 1/.style={sibling distance=3.8cm},
    level 2/.style={sibling distance=1.5cm},
    level 3/.style={sibling distance=1.2cm},
    edge from parent/.style={draw,-latex, shorten <=0.4cm}
  ]

  \tikzstyle{every node}=[circle,draw, thick, minimum size=0.6cm]

  \node (Menu0) {}
    child {
      node (L1) {}
        child { node (L1-0) [fill] {} }
        child { node (L1-1) [fill] {} }
    }
    child {
      node (L2) {}
      child { node (L2-0) [fill] {} }
      child { node (L2-1) [fill] {} }
        child {
          node (L2-2) {}
            child { node (L2-2-0) [fill] {} }
            child { node (L2-2-1) [fill] {} }
          }
    }
    child {
      node (L3)  {}
      child { node (L3-1) [fill] {} }
      child { node (L3-2) [fill] {} }
    };

    {
    \scriptsize\bf
    \draw[decoration={text along path, raise=-8pt, 
        text={总菜单}, text align={center}},decorate] 
        (Menu0.west) arc(180:360:0.3cm and 0.3cm);

    \draw[decoration={text along path, raise=-8pt, 
        text={餐厅菜单}, text align={center}},decorate] 
        (L2.west) arc(180:360:0.3cm and 0.3cm);

    \draw[decoration={text along path, raise=-8pt, 
        text={甜点菜单}, text align={center}},decorate] 
        (L2-2.west) arc(180:360:0.3cm and 0.3cm);
    }
    

    {
    \tiny\bf
    \draw[decoration={text along path, raise=-8pt, 
        text={煎饼屋菜单}, text align={center}},decorate] 
        (L1.west) arc(-180:0:0.3cm and 0.3cm);

    \draw[decoration={text along path, raise=-8pt, 
        text={咖啡厅菜单}, text align={center}},decorate] 
        (L3.west) arc(180:360:0.3cm and 0.3cm);
    }

    {
    \tiny
    \draw[decoration={text along path, raise=-8pt, 
        text={菜单项}, text align={center}},decorate] 
        (L1-0.west) arc(180:360:0.3cm and 0.3cm);
    \draw[decoration={text along path, raise=-8pt, 
        text={菜单项}, text align={center}},decorate] 
        (L1-1.west) arc(180:360:0.3cm and 0.3cm);
    \draw[decoration={text along path, raise=-8pt, 
        text={菜单项}, text align={center}},decorate] 
        (L3-1.west) arc(180:360:0.3cm and 0.3cm);
    \draw[decoration={text along path, raise=-8pt, 
        text={菜单项}, text align={center}},decorate] 
        (L3-2.west) arc(180:360:0.3cm and 0.3cm);
    \draw[decoration={text along path, raise=-8pt, 
        text={菜单项}, text align={center}},decorate] 
        (L2-0.west) arc(180:360:0.3cm and 0.3cm);
    \draw[decoration={text along path, raise=-8pt, 
        text={菜单项}, text align={center}},decorate] 
        (L2-1.west) arc(180:360:0.3cm and 0.3cm);
    \draw[decoration={text along path, raise=-8pt, 
        text={菜单项}, text align={center}},decorate] 
        (L2-2-0.west) arc(180:360:0.3cm and 0.3cm);
    \draw[decoration={text along path, raise=-8pt, 
        text={菜单项}, text align={center}},decorate] 
        (L2-2-1.west) arc(180:360:0.3cm and 0.3cm);
    }

\end{tikzpicture}

\end{frame}


\begin{frame}
  \frametitle{定义组合模式}
  \only<1> {
  \begin{block}{组合模式(Composite Pattern)}
      组合模式允许讲对象组合成树型结构来表现``整体-部分''层次结构,支持
      客户以一致的方式处理个别对象与组合对象
  \end{block}

  \vspace*{2ex}
  使用\uline{组合结构},可以把相同的操作应用在组合对象和个别对象上,即在
  大多数情况下,可以\textcolor{red}{\uline{忽略}}组合对象和个别对象之间的差别
  }

  \only<2> {

\scalebox{0.85}{
    \noindent\begin{tikzpicture}
      \tikzumlset{fill class=white}
      \umlemptyclass{Client}
      \umlclass[x=5, type=abstract]{Component}{}{\umlvirt{operation()} \\
        \umlvirt{add(Component)} \\
        \umlvirt{remove(Component)} \\
        \umlvirt{get(int)}} 
      \umlclass[x=1, y=-5]{Leaf}{}{operation()}
      \umlclass[x=6, y=-5]{Composite}{}{
        operation() \\
        add(Component) \\
        remove(Component) \\
        get(int)} 
        

      \umlinherit{Leaf}{Component}
      \umlinherit{Composite}{Component}
      \umluniassoc{Client}{Component}
      \umlHVHaggreg[arm1=3cm] {Composite}{Component}

    \end{tikzpicture}
}
}
\end{frame}

\begin{frame}
    \frametitle{利用组合模式设计菜单}
\noindent\scalebox{0.85}{
    \begin{tikzpicture}
      \tikzumlset{fill class=white}
      \umlemptyclass{Waitress}
      \umlclass[x=4, y=-2, type=abstract]{MenuComponent}{}{\umlvirt{getName()} \\
        \umlvirt{getDescription()} \\
        \umlvirt{getPrice()} \\
        \umlvirt{add(Component)} \\
        \umlvirt{remove(Component)} \\
        isVegetarian() \\
        print() \\
        getChild(int)
        } 
        \umlclass[x=0, y=-5]{MenuItem}{}{\umlvirt{getName()} \\
            \umlvirt{getDescription()} \\
            \umlvirt{getPrice()} \\
            isVegetarian() \\
            print() 
        }
            
        \umlclass[x=9, y=-5]{Menu}{}{ \umlvirt{getName()} \\
            \umlvirt{getDescription()} \\
            \umlvirt{getPrice()} \\
            print()  \\
            add(Component) \\
            remove(Component) \\
            getChild(int)} 

      \umlinherit{MenuItem}{MenuComponent}
      \umlinherit{Menu}{MenuComponent}
      \umluniassoc{Waitress}{MenuComponent}
      \umlHVaggreg[arm1=-3cm, anchor1=210] {Menu}{MenuComponent}

    \end{tikzpicture}
}
\end{frame}

\begin{frame}
    \frametitle{实现组合菜单}
    留作一个练习
\end{frame}

\begin{frame}
    \frametitle{将组合模式和迭代器结合}
    留作一个练习
\end{frame}

\begin{frame}
\frametitle{关于迭代器和组合模式的小结}
\begin{itemize}
\item 迭代器使得无需了解集合的内部结构即可遍历集合的元素
\item 迭代器将对集合的遍历进行了封装
\item 迭代器将集合遍历的任务从集合分离
\item 迭代器提供了一个遍历集合的公共接口,允许客户利用多态机制编写代码使用集合的元素
\item 程序员应当尽量让类只承担唯一的责任
\item 组合模式提供一种结构,可同时包容个别对象和组合对象
\item 组合模式允许以一致的方式处理个别对象和组合对象
\end{itemize}
\end{frame}

\begin{frame}
\frametitle{关于设计原则的小结}
  \begin{enumerate}
    \item 封装变化
    \item 多用聚合、少用继承
    \item 针对接口编程,不针对实现编程
    \item 尽最大可能将要交互的对象设计为松耦合的
    \item 对扩展开放,对修改封闭
    \item 依赖抽象,不要依赖具体类
    \item 只和朋友交谈
    \item 别找我,我会找你
    \item 类应该只有一个引起改变的理由
  \end{enumerate}
\end{frame}

\end{document}
