\documentclass[compress]{beamer}

\usepackage[nofonts]{ctex}
\setCJKmainfont[ItalicFont={Kaiti SC}]{Kaiti SC}%
%\setCJKmainfont[ItalicFont={AR PL KaitiM GB}]{AR PL KaitiM GB}%
%\setCJKsansfont{WenQuanYi Zen Hei}% 文泉驿的黑体

\mode<beamer>
{
     \useinnertheme{rectangles}
     %\useoutertheme{infolines}
     %\useoutertheme{split}
     \usecolortheme{rose}
     \usecolortheme{seahorse}

     \setbeamertemplate{navigation symbols}{}%remove navigation symbols

     %\expandafter\def\expandafter\insertshorttitle\expandafter{%
     %\insertshorttitle\hfill%
     %\insertframenumber\,/\,\inserttotalframenumber}
     %\raisebox{-1ex}{\includegraphics[width=3ex]{Overlays/logo.pdf}}}

    \setbeamertemplate{footline} {
      \leavevmode%
      \hbox{%
      \begin{beamercolorbox}[wd=.1\paperwidth,ht=2.25ex,dp=1ex,left]{date in head/foot}%
        \usebeamerfont{date in head/foot}%
        \hspace*{1ex}\raisebox{-0.8ex}{\includegraphics[width=3ex]{Overlays/logo.pdf}}%
      \end{beamercolorbox}%
      \begin{beamercolorbox}[wd=.4\paperwidth,ht=2.25ex,dp=1ex,right]{date in head/foot}%
        \usebeamerfont{date in head/foot}\insertsection\hspace*{1ex}
      \end{beamercolorbox}%
      \begin{beamercolorbox}[wd=.4\paperwidth,ht=2.25ex,dp=1ex,left]{date in head/foot}%
        \usebeamerfont{date in head/foot}\hspace*{1ex}\insertsubsection
      \end{beamercolorbox}%
      \begin{beamercolorbox}[wd=.1\paperwidth,ht=2.25ex,dp=1ex,right]{date in head/foot}%
        \insertframenumber{} / \inserttotalframenumber\hspace*{1ex}
      \end{beamercolorbox}}%
      \vskip0pt%
    }
}

%\defbeamertemplate*{footline}{mytheme}
%{
%  \leavevmode%
%  \hbox{%
%  \begin{beamercolorbox}[wd=.5\paperwidth,ht=2.25ex,dp=1ex,center]{author in head/foot}%
%    \usebeamerfont{author in head/foot}\insertshortauthor~~%
%    \raisebox{-1ex}{\includegraphics[width=3ex]{Overlays/logo.pdf}}%
%  \end{beamercolorbox}%
%  \begin{beamercolorbox}[wd=.5\paperwidth,ht=2.25ex,dp=1ex,right]{title in head/foot}%
%    \usebeamerfont{title in head/foot}\insertshorttitle{}\hspace*{2em}
%    \insertframenumber{} / \inserttotalframenumber\hspace*{2ex} 
%  \end{beamercolorbox}}%
%  \vskip0pt%
%}
%\usebeamertemplate{mytheme}

%\setbeamercovered{transparent}

\mode<handout>
{
	\usetheme{default}
	\usepackage{pgfpages}
	\pgfpagesuselayout{4 on 1}[a4paper,landscape,border shrink=5mm]
}


\usepackage{amsmath,latexsym,amssymb,amsfonts,amsbsy}
\usepackage{graphicx}
\usepackage{hyperref}
\usepackage{ulem}
\usepackage{fancybox}
\usepackage{fancyvrb}
\fvset{frame=single, numbers=left, fontsize=\small}
\usepackage{tikz}
\usetikzlibrary{calc,arrows.meta, graphs, trees, shapes, positioning,
decorations.markings, intersections, decorations.text}
\usepackage{tikz-uml}

\newcommand{\romannumber}[1]{{\textrm{\uppercase\expandafter{\romannumeral
#1}}}}

\newcommand{\textblue}[1]{\textcolor{blue}{#1}}

\graphicspath{{figure/}}

%%%%%%%%%%%%%%%%%%%%%%%%%%%%%%%%%%%%%%%%%%%%%%%%%%%%%%%%%%%%%%%%%
%    body                                                       %
%%%%%%%%%%%%%%%%%%%%%%%%%%%%%%%%%%%%%%%%%%%%%%%%%%%%%%%%%%%%%%%%%


\begin{document}

\AtBeginSubsection[]
{ 
    \begin{frame}<beamer> 
		\frametitle{内容提要} 
		\tableofcontents[currentsection,currentsubsection,
        subsectionstyle=show/shaded/hide] 
	\end{frame} 
} 

					
\title{第四章 ~~ 定义对象间的关系 \MakeUppercase{\romannumeral 1}}

\author[面向对象的分析与设计]
{曹东刚\\\href{mailto:caodg@pku.edu.cn}{caodg@pku.edu.cn}}

\institute[北京大学]{北京大学信息学院研究生课程 - 面向对象的分析与设计 \\
\href{http://sei.pku.edu.cn/~caodg/course/oo}{http://sei.pku.edu.cn/\~{}caodg/course/oo}}

\date{}

\titlegraphic{\includegraphics[height=0.1\textwidth]{Overlays/logo.pdf}}


\begin{frame}[plain]
	\titlepage
\end{frame}

\setcounter{framenumber}{0}

%\includeonlyframes{current, problem}

\begin{frame}
  \frametitle{对象间的四种关系}
  \only<1> {
  \begin{block}{一般-特殊关系, 又称继承关系}
    反映事物的分类。由这种关系可以形成一般-特殊结构
  \end{block}
  \begin{block}{整体-部分关系, 又称聚合关系}
    反映事物的构成。由这种关系可以形成整体-部分结构
  \end{block}
}
\only<2> {
  \begin{block}{关联关系}
    对象实例集合(类)上的一个关系,其中的元素提供了被开发系统的应用领域
    中一组有意义的信息
  \end{block}
  \begin{block}{交互关系}
    对象之间的动态联系,即一个对象在执行其操作时,请求其他对象为它执行某
    个操作,或者向其他对象传送某些信息。反映了事物之间的行为依赖关系
  \end{block}
}
\end{frame}

\section{一般-特殊结构}

\subsection{概念与表示法}

\begin{frame}
  \frametitle{相关定义}
  \only<1>{
  \begin{block}{继承(inheritance)}
    是描述一般类和特殊类之间关系的最传统、最经典的术语。
    有时作为动词或形容词出现
  \end{block}
  \begin{block}{分类(classification)}
    接近人类日常的语言习惯, 体现了类的层次划分,
    也作为结构的名称。在许多的场合被作为一种原则
  \end{block}
  \begin{block}{泛化(generalization)}
  UML的做法,比较简练,但是只反映了问题的一方面。作为关系的名称尚可,
  说结构是一个“泛化”则很勉强
  \end{block}
}
\only<2> {
  \begin{block}{一般-特殊(generalization-specialization)}
    含义最准确,而
      且不容易产生误解,恰切地反映了一般类(概念)和特殊类(概念)之间的
      相对(二元)关系;也用于描述结构,即一般-特殊结构。缺点是书写和阅
      读比较累赘
    \end{block}

  相关概念:一般类、特殊类、继承、多继承、多态 \\
  语义:“is a kind of ”
  }
\end{frame}

\begin{frame}
  \frametitle{一般-特殊关系vs 一般-特殊结构}
  %\begin{center}
    %\centering\includegraphics[width=0.5\hsize]{g-s-structure.pdf}
  %\end{center}
  \centering\begin{tikzpicture}
    \tikzumlset{fill class=white}
    \umlemptyclass{人员}
    \umlemptyclass[x=-2, y=-2.5]{股东}
    \umlemptyclass[x=2, y=-2.5]{顾客}
    \umlemptyclass[y=-2.5]{职员}
    \umlemptyclass[x=-1, y=-5]{股东职员}

    \umlinherit{股东}{人员}
    \umlinherit{职员}{人员}
    \umlinherit{顾客}{人员}
    \umlinherit{股东职员}{股东}
    \umlinherit{股东职员}{职员}
  \end{tikzpicture}
\end{frame}

\begin{frame}
  \frametitle{一般类-特殊类的两个定义}
  %\begin{center}
    %\centering\includegraphics[width=1.0\hsize]{super-sub-2-def.pdf}
  %\end{center}
  \noindent\begin{tikzpicture}
    \node [draw, ellipse, minimum width=4cm, minimum height=2cm] (IS)
    {} ;
    \node [below=0.2 of IS.north] {一般类} ;
    \node [above=0.2 of IS.north] {\textbf{对象实例组合}} ;
    \node [draw, ellipse, above=0.3 of IS.south] (SC) {特殊类} ;

    \node [draw, ellipse, minimum width=4cm, minimum height=2cm,
    right=3 of IS] (FS) {} ;
    \node [below=0.2 of FS.north] {特殊类的特征} ;
    \node [above=0.2 of FS.north] {\textbf{特征集合}} ;
    \node [draw, ellipse, above=0.3 of FS.south] (SF) {一般类的特征} ;

    \path [Circle-Stealth] ([xshift=-0.3cm]SC.east) edge [bend right=10]
    node [below, sloped] {必须有这些特征} (FS.south west) 
    ([yshift=-0.3cm]IS.north east) edge [bend left=10] node [above,
    sloped] {只
    要有这些特征} (SF) ; 


  \end{tikzpicture}
\end{frame}

\begin{frame}
  \frametitle{继承关系表示法}
  %\begin{center}
    %\centering\includegraphics[width=0.45\hsize]{gen-notation-c.pdf} %
    %\hfill \centering\includegraphics[width=0.43\hsize]{gen-notation-d.pdf}
  %\end{center}
  \noindent\begin{tikzpicture}
    \tikzumlset{fill class=white}
    \umlemptyclass{一般类}
    \umlemptyclass[x=-1.5, y=-3]{特殊类A}
    \umlemptyclass[x=1.5, y=-3]{特殊类B}
    \umlVHVinherit{特殊类A}{一般类}
    \umlVHVinherit{特殊类B}{一般类}
    \node [below = 3.5 of 一般类] {集中式} ;
  \end{tikzpicture}\quad%
  \begin{tikzpicture}
    \tikzumlset{fill class=white}
    \umlemptyclass{一般类}
    \umlemptyclass[x=-1.5, y=-3]{特殊类A}
    \umlemptyclass[x=1.5, y=-3]{特殊类B}
    \umlinherit{特殊类A}{一般类}
    \umlinherit{特殊类B}{一般类}
    \node [below = 3.5 of 一般类] {分散式} ;
  \end{tikzpicture}
\end{frame}

\begin{frame}
  \frametitle{继承关系表示法示例}
  %\begin{center}
    %\centering\includegraphics[width=0.5\hsize]{gen-notation-example.pdf} %
  %\end{center}
  \centering\begin{tikzpicture}
    \tikzumlset{fill class=white}
    \umlemptyclass{人员}
    \umlemptyclass[x=-2, y=-2.5]{股东}
    \umlemptyclass[x=2, y=-2.5]{顾客}
    \umlemptyclass[y=-2.5]{职员}
    \umlemptyclass[x=-1, y=-5]{股东职员}

    \umlinherit{股东}{人员}
    \umlinherit{职员}{人员}
    \umlinherit{顾客}{人员}
    \umlinherit{股东职员}{股东}
    \umlinherit{股东职员}{职员}
  \end{tikzpicture}
\end{frame}

\subsection{发现与调整}

\begin{frame}
  \frametitle{如何发现一般-特殊结构}
  \begin{enumerate}
    \item 学习当前领域的分类学知识
    \item 按常识考虑事物的分类
    \item 根据一般类和特殊类的两种定义
    \item 考察属性与操作的适应范围
    \item 考虑领域范围内的复用
  \end{enumerate}
\end{frame}

\begin{frame}
  \frametitle{发现一般-特殊结构: 示例}
  \only<1> {
  %\begin{center}
    %\centering\includegraphics[width=0.9\hsize]{gen-find-case1.pdf} %
  %\end{center}
    \begin{tikzpicture}
      \tikzumlset{fill class=white}
      \umlclass{公司人员}{姓名\\身份证号\\\alert{股份}\\\alert{工资}}{}

      \umlclass[x=6, y=2]{公司人员}{姓名\\身份证号}{}
      \umlclass[x=5, y=-2]{股东}{股份}{}
      \umlclass[x=7, y=-2]{职员}{工资}{}

      \umlVHVinherit{股东}{公司人员}
      \umlVHVinherit{职员}{公司人员}

      \draw [dashed] (3, 3) -- (3, -3) ;

      \node [single arrow, draw, text width=0.5cm, fill=blue!50] at (3, 0) {} ;

    \end{tikzpicture}
  }
  \only<2> {
  %\begin{center}
    %\centering\includegraphics[width=0.9\hsize]{gen-find-case2.pdf} %
  %\end{center}
  \noindent\begin{tikzpicture}
      \tikzumlset{fill class=white}
      \umlclass[y=2.3]{?}{}{}
      \umlclass[x=-1.5, y=-1.5]{股东}{\alert{姓名}\\\alert{身份证号}\\股份}{}
      \umlclass[x=1.5, y=-1.5]{职员}{\alert{姓名}\\\alert{身份证号}\\工资}{}

      \node [single arrow, rotate=90, text width=0.5cm, fill=red!30]
        at (0, 0.6) {} ;

      \umlclass[x=6, y=2]{公司人员}{姓名\\身份证号}{}
      \umlclass[x=5, y=-2]{股东}{股份}{}
      \umlclass[x=7, y=-2]{职员}{工资}{}

      \umlVHVinherit{股东}{公司人员}
      \umlVHVinherit{职员}{公司人员}

      \draw [dashed] (3, 3) -- (3, -3) ;

      \node [single arrow, draw, text width=0.5cm, fill=blue!50] at (3, 1) {} ;

    \end{tikzpicture}
  }
  \only<3> {
  %\begin{center}
    %\centering\includegraphics[width=0.8\hsize]{gen-find-case3.pdf} %
  %\end{center}
    \scalebox{0.9} {
    \begin{tikzpicture}
      \renewcommand\baselinestretch{1.2}
      \tikzumlset{fill class=white}
      \umlclass{现钞收款机}{attr1 \\ attr2 \\ attr3 \\ attr4 \\ attr5 \\
      attr6 } {op1 \\ op2 \\op3}

      \node [single arrow, draw, text width=0.5cm, fill=blue!50, right=
      1.5 of 现钞收款机] (Arrow) {} ;

      \umlclass[x=6, y=2.2]{收款机}{attr1 \\ attr2 \\ attr3 } {op1 \\ op2 }
      \umlclass[x=6, y=-2.2]{现钞收款机}{attr4 \\ attr5 \\ attr6 } {op3}
      \umlinherit{现钞收款机}{收款机}

      \umlnote[x=9, y=2.2, width=1.5cm]{收款机}{领域构件}


    \end{tikzpicture}
  }
  }
\end{frame}

\begin{frame}
  \frametitle{审查与调整: 方法}
  \begin{itemize}
    \item 问题域是否需要这样的分类? \\
      \textcolor{blue}{例}:书 — 线装书
    \item 系统责任是否需要这样的分类? \\
      \textcolor{blue}{例}:职员 — 本市职员
    \item 是否符合分类学的常识? \\
      \textcolor{blue}{用}:“is a kind of ”来衡量
    \item 是否真正的继承了一些属性或操作?
  \end{itemize}
\end{frame}

\begin{frame}
  \frametitle{审查与调整: 示例}
  %\begin{center}
    %\centering\includegraphics[width=1.0\hsize]{gen-adjust.pdf} %
  %\end{center}
    \begin{tikzpicture}
        \umlclass[fill=white]{汽车}{发动机\\载重量\\速度}{运输}
      \umlclass[y=-4, fill=white]{飞机}{飞行高度}{自动导航}
      \umlinherit{飞机}{汽车}
    \end{tikzpicture}\qquad\qquad\pause\begin{tikzpicture}
      \node [single arrow, draw, text width=0.5cm, fill=blue!50] {} ;

      \umlclass[x=3, fill=white]{运输工具}{发动机\\载重量\\速度}{运输}
      \umlclass[x=4.5, y=-4, fill=white]{飞机}{飞行高度}{自动导航}
      \umlclass[x=1.5, y=-4, fill=white]{汽车}{}{}
      \umlinherit{飞机}{运输工具}
      \umlinherit{汽车}{运输工具}
    \end{tikzpicture}
\end{frame}

\begin{frame}
  \frametitle{一般-特殊结构的简化}
  \only<1> {
    取消没有特殊性的特殊类 \\[2ex]
    %\begin{center}
      %\centering\includegraphics[width=0.7\hsize]{gen-sim-1.pdf} %
    %\end{center}
    \begin{tikzpicture}
      \renewcommand\baselinestretch{1.0}
      \umlclass[fill=white]{运输工具}{发动机\\载重量\\速度}{运输}
      \umlclass[x=1.5, y=-3.5, fill=white]{飞机}{飞行高度}{自动导航}
      \umlclass[x=-1.5, y=-3.5, fill=white]{汽车}{}{}
      \umlVHVinherit[arm2=-2cm]{飞机}{运输工具}
      \umlVHVinherit[arm2=-2cm]{汽车}{运输工具}

      \node [single arrow, draw, text width=0.5cm, fill=blue!50] at
      (3,0) {} ;

      \umlclass[x=6, fill=white]{运输工具}{发动机\\载重量\\速度}{运输}
      \umlclass[x=6, y=-3.5, fill=white]{飞机}{飞行高度}{自动导航}
      \umlinherit{飞机}{运输工具}

    \end{tikzpicture}
  }
  \only<2> {
    增加属性简化一般-特殊结构 \\[2ex]
    %\begin{center}
      %\centering\includegraphics[width=1.0\hsize]{gen-sim-2.pdf} %
    %\end{center}
    \hspace*{-4ex}\scalebox{0.9}{
    \noindent\begin{tikzpicture}
      \tikzumlset{fill class=white}
      \umlemptyclass{人员}
      \umlemptyclass[x=-3, y=-3]{男人}
      \umlemptyclass[x=-1, y=-3]{女人}
      \umlemptyclass[x=1, y=-3]{中国人}
      \umlemptyclass[x=3, y=-3]{美国人}
      \umlemptyclass[x=5, y=-3]{日本人}
      \umlVHVinherit{男人}{人员}
      \umlVHVinherit{女人}{人员}
      \umlVHVinherit{中国人}{人员}
      \umlVHVinherit{美国人}{人员}
      \umlVHVinherit{日本人}{人员}

      \node [single arrow, draw, text width=0.5cm, fill=blue!50] at
      (6.5,-1) {} ;
      \umlclass[x=8.3, y=-1]{人员}{性别\\国籍}{}
    \end{tikzpicture}
  }
  }
  \only<3> {
    \begin{columns}[c]
    \column{0.5\hsize}
    取消用途单一的一般类,减少继承层次。一般类存在的理由:\\
    \begin{itemize}
      \item 有两个或两个以上的特殊类
      \item 需要用它创建对象实例
      \item 有助于软件复用
    \end{itemize}

      \column{0.5\hsize}

    %\begin{center}
      %\centering\includegraphics[width=0.9\hsize]{gen-sim-radar.pdf} %
    %\end{center}
      \scalebox{0.9}{
      \begin{tikzpicture}
        \tikzumlset{fill class=white}
        \renewcommand\baselinestretch{1.0}
        \umlclass{设备}{型号\\生产厂}{开启\\关闭}
        \umlclass[y=-3.5]{雷达}{安装地点}{监控}
        \umlinherit{雷达}{设备}
        \node [single arrow, draw, text width=0.5cm, fill=blue!50] at
      (2,-1) {} ;
      \umlclass[x=4, y=-1.5]{设备}{型号\\生产厂\\安装地点}{开启\\ 关闭 \\监控}
      \end{tikzpicture}
    }
  \end{columns}
  }
  \only<4> {
    取消用途单一的一般类,减少继承层次 \\[2ex]
    %\begin{center}
      %\centering\includegraphics[width=0.9\hsize]{gen-sim-printer.pdf} %
    %\end{center}
    \noindent\scalebox{0.9}{
    \begin{tikzpicture}
      \tikzumlset{fill class=white}
      \umlsimpleclass{设备}
      \umlsimpleclass[y=-1.5]{电子设备}
      \umlsimpleclass[x=-1.5, y=-3]{计算机设备}
      \umlsimpleclass[x=1.5, y=-3]{通信设备}
      \umlsimpleclass[x=1.5, y=-4.5]{民用通信设备}
      \umlsimpleclass[x=-1.5, y=-4.5]{打印机}
      \umlsimpleclass[x=-1.5, y=-6]{激光打印机}
      \umlsimpleclass[x=1.5, y=-6]{传真机}
      \umlinherit{电子设备}{设备}
      \umlinherit{打印机}{计算机设备}
      \umlinherit{激光打印机}{打印机}
      \umlinherit{传真机}{民用通信设备}
      \umlinherit{民用通信设备}{通信设备}
      \umlVHVinherit{通信设备}{电子设备}
      \umlVHVinherit{计算机设备}{电子设备}

      \node [single arrow, draw, text width=0.5cm, fill=blue!50] at
      (3.5,-2) {} ;

      \umlsimpleclass[x=6.5, y=-2, fill=white]{电子设备}
      \umlsimpleclass[x=8, y=-4, fill=white]{激光打印机}
      \umlsimpleclass[x=5, y=-4, fill=white]{传真机}
      \umlVHVinherit{激光打印机}{电子设备}
      \umlVHVinherit{传真机}{电子设备}
    \end{tikzpicture}
  }
  }

\end{frame}

\begin{frame}
  \frametitle{多继承:特殊类具有多个一般类}
    %\begin{center}
      %\centering\includegraphics[width=1.0\hsize]{gen-multi.pdf} %
    %\end{center}
  \begin{tikzpicture}
    \tikzumlset{fill class=white}
    \renewcommand\baselinestretch{1.0}
    \umlclass{人员}{姓名}{}
    \umlclass[x=-2, y=-3]{研究生}{学号\\班级\\\alert{专业}}{}
    \umlclass[x=2, y=-3]{教职工}{职称\\单位\\\alert{专业}}{}
    \umlclass[y=-5.5]{在职研究生}{}{}
    \umlHVinherit{在职研究生}{研究生}
    \umlHVinherit{在职研究生}{教职工}
    \umlVHinherit{研究生}{人员}
    \umlVHinherit{教职工}{人员}
    \umlnote[x=5, y=-3]{教职工}{在职研究生类继承自研究生和教职工类的\alert{专业}属性会发
  生命名冲突,需要调整设计}
  \end{tikzpicture}
\end{frame}

\begin{frame}
  \frametitle{多态:同一个命名可具有不同的语义}
  \begin{block}{OO方法中的多态}
    常指在一般类中定义的属性或操
    作被特殊类继承之后,可以具有不同的数据类型或表现出不同的行为
  \end{block}
    %\begin{center}
      %\centering\includegraphics[width=0.8\hsize]{gen-polym.pdf} %
    %\end{center}
  \centering\begin{tikzpicture}
    \tikzumlset{fill class=white}
    \umlclass{多边形}{边数\\顶点数据}{绘图}
    \umlclass[x=-2.5, y=-2]{正多边形}{}{}
    \umlclass[x=2.5, y=-2]{矩形}{}{}
    \umlVHinherit{正多边形}{多边形}
    \umlVHinherit{矩形}{多边形}

    \node[regular polygon, regular polygon sides=6, draw, fill=cyan!40,
    inner sep=0.3535cm] at (-4,0) (Poly) {} ;
    \node[rectangle, draw, fill=lime!40, text width=1cm,
    inner sep=0.3535cm] at (4,0) (Rec) {} ;
    \path [->, dashed] (矩形.east) edge [bend right] (Rec) ;
    \path [->, dashed] (正多边形.west) edge [bend left] (Poly) ;
  \end{tikzpicture}

\end{frame}

\section{整体-部分结构}

\subsection{概念与表示法}

\begin{frame}
  \frametitle{整体-部分结构}
  \begin{block}{相关概念}
  聚合(aggregation), 组合(composition)\\
  整体-部分(whole-part )\\
  整体对象,部分对象
\end{block}
\begin{block} {语义}
  “a part of”或“has a”
\end{block}

  聚合关系描述了对象实例之间的构成情况,然而它的定义却是在类的抽象层次给
  出的
\end{frame}

\begin{frame}
  \frametitle{整体-部分关系 vs 整体-部分结构}
  \begin{block}{整体-部分关系(聚合关系)}
    是两个类之间的二元关系,其中一
      个类的某些对象是另一个类的某些对象的组成部分
    \end{block}
    \begin{block}{整体-部分结构}
      是把若干具有聚合关系的类组织在一起所形成
      的结构。它是一个以类为结点,以聚合关系为边的连通有向图
    \end{block}
\end{frame}

\begin{frame}
  \frametitle{组合(composition)}
    组合是聚合关系的一种特殊情况,它表明整体对
    于部分的\alert{\textbf{强拥有}}关系,即整体与部分之间具有紧密、
    固定的组成关系 \\
    \vspace{1cm}
    UML把聚合定义为关联的一种特殊情况,
    而组合关系是聚合关系的特殊情况

\end{frame}

\begin{frame}[label=problem]
  \frametitle{几种说法}
  若类 A 的对象 a 是类 B 对象 b 的一个组成部分,则 \\
  \vspace{1cm}
  \begin{tabular}{lp{1cm}l}
    \onslide<1-> {\alt<1>{\color{blue}对象b和对象a之间具有聚合关系}{对象b和对象a之间具
    有聚合关系 & & $\surd$ } }\\ 
    \onslide<2-> {\alt<2>{\color{blue} 类B和类A之间具有聚合关系}{类B和类
    A之间具有聚合关系   & & $\surd$ }}\\
    \onslide<3-4> {\alt<3>{\color{blue} 类A是类B的一个组成部分} {类A是类B
    的一个组成部分  & & ?}} \\
  \end{tabular}
\end{frame}


\begin{frame}
  \frametitle{表示法}
    %\begin{center}
      %\centering\includegraphics[width=0.6\hsize]{agg-notation.pdf} %
    %\end{center}
  \centering\begin{tikzpicture}
    \tikzumlset{fill class=white}
    \umlemptyclass{整体对象类}
    \umlemptyclass[y=-3.5]{部分对象类}
    \umlaggreg[mult1=数量, mult2=数量]{整体对象类}{部分对象类}

    \umlemptyclass[x=5]{整体对象类}
    \umlemptyclass[x=5, y=-3.5]{部分对象类}
    \umlcompo[mult1=数量, mult2=数量]{整体对象类}{部分对象类}

  \end{tikzpicture}
\end{frame}

\begin{frame}[fragile]
  \frametitle{表示法中的多重性(multiplicity)}

  \begin{columns}[c]
    \column{0.55\hsize}
    在连接符两端通过数字或者符号给出关系双方对象实例的数量约束, 例: \\
\begin{Verbatim}[frame=none, numbers=none, commandchars=\\\^\$]
  确定数量, 如:\textblue^1,2$
  一个范围, 如:\textblue^0‥1,1‥4$ 
  数量不定, 如:\textblue^*$ 
  下界确定, 如:\textblue^0‥*,1‥*$
\end{Verbatim}


    \column{0.45\hsize}
    %\begin{center}
      %\centering\includegraphics[width=1.0\hsize]{agg-notation-multi.pdf} %
    %\end{center}
    \begin{tikzpicture}
      \tikzumlset{fill class=white}
      \umlemptyclass{照相机}
      \umlemptyclass[x=-1.5, y=-3]{机身}
      \umlemptyclass[x=1.5, y=-3]{镜头}
      \umlcompo[mult1=0..1,pos1=0.3, mult2=1]{照相机}{机身}
      \umlaggreg[arg1=0..1,pos1=0.3, arg2=1..*]{照相机}{镜头}
    \end{tikzpicture}
  \end{columns}

\end{frame}


\subsection{发现与调整}

\begin{frame}
  \frametitle{如何发现整体-部分结构}
  \ovalbox{
  \textbf{\textcolor{red}{基本策略}}: 考察问题域中各种具有构成关系的事物
}
  \begin{enumerate}
      \only<1>{
    \item 物理上的整体事物和它的组成部分 \\
      \textcolor{blue}{\textbf{例}}:机器、设备和它的零部件
    \item 组织机构和它的下级组织及部门 \\
      \textcolor{blue}{\textbf{例}}:公司与子公司、部门
    \item 团体(组织)与成员 \\
      \textcolor{blue}{\textbf{例}}:公司与职员
    }
    \only<2>{
    \setcounter{enumi}{3} 
    \item 一种事物在空间上包容其它事物 \\
      \textcolor{blue}{\textbf{例}}:生产车间与机器
    \item 抽象事物的整体与部分 \\
      \textcolor{blue}{\textbf{例}}:学科与分支学科、法律与法律条款
    \item 具体事物和它的某个抽象方面 \\
      \textcolor{blue}{\textbf{例}}:人员与身份、履历
    }
  \end{enumerate}
\end{frame}

\begin{frame}
  \frametitle{审查与筛选}
  \begin{itemize}
    \item 是否属于问题域? \\
      \textcolor{blue}{\textbf{例}}:公司职员与家庭
    \item 是不是系统责任的需要? \\
      \textcolor{blue}{\textbf{例}}:员工与工会
    \item 部分对象是否有一个以上的属性? \\
      \textcolor{blue}{\textbf{例}}:汽车与车轮(规格)
    \item 是否有明显的整体-部分关系? \\
      \textcolor{blue}{\textbf{例}}:学生与课程
  \end{itemize}
\end{frame}

\subsection{应用技巧}

\begin{frame}
  \frametitle{简化对象的定义}
    %\begin{center}
      %\centering\includegraphics[width=0.3\hsize]{agg-sim.pdf} %
      %\hspace{1cm} \pause %
      %\includegraphics[width=0.4\hsize]{agg-sim2.pdf}
    %\end{center}
  \scalebox{0.9}{
  \begin{tikzpicture}
    \renewcommand\baselinestretch{1.0}
    \umlclass[fill=white]{飞机}{$\otimes$ \\ $\otimes$ \\ $\otimes$ \\ $\boxdot$ \\
    $\boxdot$ \\ $\perp$ \\ $\perp$}{$\clubsuit$ \\ $\clubsuit$ \\
      $\spadesuit$ \\ $\spadesuit$ \\
    $\diamondsuit$ \\ $\diamondsuit$}

      \node [single arrow, draw, text width=0.5cm, fill=blue!50] at (3, 0) {} ;

      \umlclass[x=6, y=2, fill=white] {飞机}{$\otimes$ \\ $\otimes$ \\
      $\otimes$}{$\clubsuit$ \\ $\clubsuit$ }
      \umlclass[x=5, y=-3, fill=white]{发动机}{$\boxdot$ \\
      $\boxdot$ }{$\spadesuit$ \\ $\spadesuit$ }
      \umlclass[x=7, y=-3, fill=white]{驾驶室}{$\perp$ \\
      $\perp$}{$\diamondsuit$ \\ $\diamondsuit$}

      \umlcompo[mult1=1, pos1=0.3, mult2=1..4]{飞机}{发动机}
      \umlcompo[arg1=1, pos1=0.3, arg2=1]{飞机}{驾驶室}
  \end{tikzpicture}
}
\end{frame}

\begin{frame}
  \frametitle{支持软件复用}
    %\begin{center}
      %\centering\includegraphics[width=0.9\hsize]{agg-reuse.pdf} %
    %\end{center}
  \noindent\begin{tikzpicture}
    \tikzumlset{fill class=white}
    \umlemptyclass[width=1.5cm]{机床}
    \umlemptyclass[x=-2, y=-3, width=1.5cm]{车床}
    \umlemptyclass[y=-3, width=1.5cm]{刨床}
    \umlemptyclass[x=2, y=-3, width=1.5cm]{钻床}
    \umlVHVinherit{车床}{机床}
    \umlVHVinherit{刨床}{机床}
    \umlVHVinherit{钻床}{机床}

    \umlemptyclass[x=5, y=-3]{电动机}
    \umlemptyclass[x=5]{起重机}
    \umlemptyclass[x=7]{送料车}
    \umlcompo[mult1=0..1, mult2=1]{起重机}{电动机}
    \umlVHcompo[mult1=0..1, mult2=1]{送料车}{电动机}
    \umlHVHcompo[mult1=0..1, mult2=1, pos2=2.8, arm1=3.3cm]{机床}{电动机}
  \end{tikzpicture}
\end{frame}

\begin{frame}
  \frametitle{表示数量不定的组成部分}
    %\begin{center}
    %  \centering\includegraphics[width=0.8\hsize]{agg-uncertain.pdf} %
    %\end{center}
  \centering\begin{tikzpicture}
    \tikzumlset{fill class=white}
    \umlclass{订单行}{商品编号\\订购数量\\成交价\\金额}{}
    \umlclass[x=-4]{订单}{编号\\卖方\\买房\\总金额\\成交日期}{}
    \umlclass[x=4]{商品}{商品编号 \\ 商品名称 \\ 单价 \\ 厂商}{}
    \umlaggreg[mult1=1, mult2=1..*]{订单}{订单行}
    \umlassoc[mult1=1, mult2=*]{商品}{订单行}
  \end{tikzpicture}
    \pause

    \alert{\textbf{思考}}:能否不要订单行,直接用商品作为订单的部分对象?
\end{frame}

\begin{frame}
  \frametitle{表示动态变化的动态特征}
  \only<1>{
    \alert{\textbf{问题}}:对象的属性/操作在运行时刻动态变化,如 \\[2ex]
    %\begin{center}
      %\centering\includegraphics[width=0.6\hsize]{agg-dyna0.pdf} %
    %\end{center}
    \centering\begin{tikzpicture}
      \tikzumlset{fill class=white}
      \umlemptyclass{人员}
      \umlemptyclass[y=-3]{会计师}
      \umlemptyclass[x=-2, y=-3]{营业员}
      \umlemptyclass[x=2, y=-3]{经理}

      \umlinherit{会计师}{人员}
      \umlVHVinherit{营业员}{人员}
      \umlVHVinherit{经理}{人员}
    \end{tikzpicture}
  }
  \only<2>{
    解决方案 1: \\[2ex]
    %\begin{center}
      %\centering\includegraphics[width=0.6\hsize]{agg-dyna1.pdf} %
    %\end{center}
    \begin{tikzpicture}
      \tikzumlset{fill class=white}
      \umlemptyclass{人员}
      \umlemptyclass[y=-3]{会计师身份}
      \umlemptyclass[x=-3, y=-3]{营业员身份}
      \umlemptyclass[x=3, y=-3]{经理身份}

      \umlaggreg[mult1=1, mult2=0..1]{人员}{会计师身份}
      \umlaggreg[mult1=1, mult2=0..1]{人员}{营业员身份}
      \umlaggreg[arg1=1, arg2=0..1]{人员}{经理身份}
    \end{tikzpicture}
  }
  \only<3>{
    解决方案 2: \\[2ex]
    %\begin{center}
      %\centering\includegraphics[width=0.6\hsize]{agg-dyna2.pdf} %
    %\end{center}
    \centering\begin{tikzpicture}
      \tikzumlset{fill class=white}
      \umlemptyclass{人员}
      \umlemptyclass[x=4]{身份}
      \umlemptyclass[x=4, y=-3]{会计师身份}
      \umlemptyclass[x=1, y=-3]{营业员身份}
      \umlemptyclass[x=7, y=-3]{经理身份}

      \umlaggreg[mult1=1, mult2=m]{人员}{身份}
      \umlinherit{会计师身份}{身份}
      \umlVHVinherit{营业员身份}{身份}
      \umlVHVinherit{经理身份}{身份}
    \end{tikzpicture}
  }
\end{frame}

\begin{frame}
  \frametitle{“三友”的解决方法: 继承}
    %\begin{center}
    %  \centering\includegraphics[width=0.48\hsize]{agg-dyna-bang3-1999.pdf} %
    %  \pause %
    %  \includegraphics[width=0.5\hsize]{agg-dyna-bang3-2005.pdf}
    %\end{center}
  \only<1> {
  \centering\begin{tikzpicture}
    \tikzumlset{fill class=white}
    \umlsimpleclass{person}
    \umlsimpleclass[y=3, type=type]{employee}
    \umlsimpleclass[x=-2.5, y=3, type=type]{candidate}
    \umlsimpleclass[x=2.5, y=3, type=type]{retiree}
    \umlinherit{person}{candidate}
    \umlinherit{person}{employee}
    \umlinherit{person}{retiree}
    \node [below = 0.3 of person] {1999年第一版} ; 
  \end{tikzpicture}
  }
  \only<2> {
  \centering\begin{tikzpicture}
    \tikzumlset{fill class=white}
    \umlsimpleclass{person}
    \umlsimpleclass[y=-3, type=dynamic]{employee}
    \umlsimpleclass[x=-2.5, y=-3, type=dynamic]{candidate}
    \umlsimpleclass[x=2.5, y=-3, type=dynamic]{retiree}
    \umlinherit{candidate}{person}
    \umlinherit{employee}{person}
    \umlinherit{retiree}{person}
    \node [below = 0.3 of employee] {2005年第二版} ; 
  \end{tikzpicture}
  }
\end{frame}

\begin{frame}
  \frametitle{用聚合概念解决}
    %\begin{center}
    %  \centering\includegraphics[width=0.8\hsize]{agg-dyna3.pdf} %
    %\end{center}
  \centering\begin{tikzpicture}
    \tikzumlset{fill class=white}
    \umlsimpleclass{person}
    \umlsimpleclass[y=-3]{EmployeeRole}
    \umlsimpleclass[x=-3.5, y=-3]{CandidateRole}
    \umlsimpleclass[x=3.5, y=-3]{RetireeRole}
    \umlaggreg[mult1=1, mult2=0..1]{person}{CandidateRole}
    \umlaggreg[mult1=1, mult2=0..1]{person}{EmployeeRole}
    \umlaggreg[arg1=1, arg2=0..1]{person}{RetireeRole}
  \end{tikzpicture}
\end{frame}

\begin{frame}
  \frametitle{启示}
  \begin{itemize}
    \item <1-> \alert<1> {整体-部分结构有很强的表达能力}
    \item <2-> \alert<2> {运用OO方法的基本概念可以自然而有效地解决许多
      在其他方法中用扩充概念解决的问题}
    \item <3-> \alert<3> {加强对基本概念的运用,不要轻易创造新的扩充概念
      }
  \end{itemize}
\end{frame}

\begin{frame}[label=current]
  \frametitle{两种结构的变通}
    %\begin{center}
      %\centering\includegraphics[width=0.45\hsize]{agg-gen-convert.pdf} %
      %\hfill \only<2>
      %{\includegraphics[width=0.5\hsize]{agg-gen-convert1.pdf}
    %} %
    %\only<3> { \includegraphics[width=0.45\hsize]{agg-gen-convert2.pdf}}
    %\end{center}
  \noindent\begin{tikzpicture}
    \tikzumlset{fill class=white}
    \umlemptyclass{冷藏车}
    \umlemptyclass[x=-1.3, y=3]{汽车}
    \umlemptyclass[x=1.3, y=3]{制冷设备}
    \umlinherit{冷藏车}{汽车}
    \umlinherit{冷藏车}{制冷设备}

    \node [single arrow, draw, text width=0.5cm, fill=blue!50] at (2.6, 1.5) {} ;
  \end{tikzpicture}%
  \only<2>{
    \hspace*{2ex}\begin{tikzpicture}
    \tikzumlset{fill class=white}
    \umlemptyclass{冷藏车}
    \umlemptyclass[y=3]{汽车}
    \umlemptyclass[x=3]{制冷设备}
    \umlinherit{冷藏车}{汽车}
    \umlaggreg{冷藏车}{制冷设备}
  \end{tikzpicture}
  }%
  \only<3>{
    \hspace*{2ex}\begin{tikzpicture}
    \tikzumlset{fill class=white}
    \umlemptyclass{冷藏车}
    \umlemptyclass[x=-1.3, y=-3]{汽车}
    \umlemptyclass[x=1.3, y=-3]{制冷设备}
    \umlaggreg{冷藏车}{汽车}
    \umlaggreg{冷藏车}{制冷设备}
  \end{tikzpicture}
  }%


\end{frame}

\end{document}
