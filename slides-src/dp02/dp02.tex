\documentclass[compress]{beamer}

\usepackage[nofonts]{ctex}
\setCJKmainfont[ItalicFont={Kaiti SC}]{Kaiti SC}%
%\setCJKmainfont[ItalicFont={AR PL KaitiM GB}]{AR PL KaitiM GB}%
%\setCJKsansfont{WenQuanYi Zen Hei}% 文泉驿的黑体

\mode<beamer>
{
     \useinnertheme{rectangles}
     %\useoutertheme{infolines}
     %\useoutertheme{split}
     \usecolortheme{rose}
     \usecolortheme{seahorse}

     \setbeamertemplate{navigation symbols}{}%remove navigation symbols

%     \expandafter\def\expandafter\insertshorttitle\expandafter{%
%     \insertshorttitle\hfill%
%     \insertframenumber\,/\,\inserttotalframenumber}
}

\defbeamertemplate*{footline}{mytheme}
{
  \leavevmode%
  \hbox{%
  \begin{beamercolorbox}[wd=1.0\paperwidth,ht=2.25ex,dp=1ex,center]{author in head/foot}%
    \raisebox{-1ex}{\includegraphics[width=3ex]{Overlays/logo.pdf}}%
    \hspace*{6ex}\insertframenumber{} / \inserttotalframenumber %
  \end{beamercolorbox}}%
  \vskip0pt%
}
\usebeamertemplate{mytheme}

\setbeamertemplate{headline}{}

%\setbeamercovered{transparent}

\mode<handout>
{
	\usetheme{default}
	\usepackage{pgfpages}
	\pgfpagesuselayout{2 on 1}[a4paper,landscape,border shrink=5mm]
}


\usepackage{amsmath,latexsym,amssymb,amsfonts,amsbsy}
\usepackage{graphicx}
\usepackage{array}
\usepackage{hyperref}
\usepackage{textpos}
\usepackage{ulem}
\usepackage{comment}
\usepackage{fancyvrb}
\fvset{frame=single, numbers=left, fontsize=\small}
\usepackage{tikz}
\usetikzlibrary{calc,arrows.meta, graphs, trees, shapes, positioning, decorations.markings, intersections, decorations.text}
\usepackage{tikz-uml}

\newcommand{\romannumber}[1]{{\textrm{\uppercase\expandafter{\romannumeral
#1}}}}

\setbeamercolor{dblue}{fg=white,bg=gray!70!blue} % for beamercolorbox
 \newenvironment{pblock}{\begin{beamercolorbox}[rounded=true,
          shadow=false]{dblue}}{\end{beamercolorbox}}
 \newenvironment{bblock}{\begin{beamercolorbox}[rounded=true,
          shadow=false]{fg=black, bg=white}}{\end{beamercolorbox}}

\graphicspath{{figure/}}

%%%%%%%%%%%%%%%%%%%%%%%%%%%%%%%%%%%%%%%%%%%%%%%%%%%%%%%%%%%%%%%%%
%    body                                                       %
%%%%%%%%%%%%%%%%%%%%%%%%%%%%%%%%%%%%%%%%%%%%%%%%%%%%%%%%%%%%%%%%%


\begin{document}

%\AtBeginSubsection[]
%\AtBeginSection[]
%{ 
%    \begin{frame}<beamer> 
%		\frametitle{内容提要} 
%		\tableofcontents[currentsection,currentsubsection] 
%	\end{frame} 
%} 

					
\title[装饰者模式]{面向对象设计模式 ~~ 装饰者}

\author[曹东刚]
{曹东刚\\\href{mailto:caodg@pku.edu.cn}{caodg@pku.edu.cn}}

\institute[北京大学]{北京大学信息学院研究生课程 - 面向对象的分析与设计 \\
    \href{http://sei.pku.edu.cn/~caodg/course/oo}{http://sei.pku.edu.cn/\~{}caodg/course/oo}}

\date{}

\titlegraphic{\includegraphics[height=0.10\textwidth]{Overlays/logo.pdf}}

\begin{frame}[plain]
	\titlepage
\end{frame}

\setcounter{framenumber}{0}

%\includeonlyframes{current}

\begin{frame}
  \frametitle{装饰对象(Decorating Objects)}
  本章将学习如何使用\uline{对象聚合}的方式,做到在\uline{运行时装饰类}
  \\[3ex]
  一旦熟悉了装饰的技巧,就可在\uline{不修改任何底层代码}的情况下,给对象
  \uline{赋予新的职责} \\[6ex]

  \framebox{\parbox{0.9\hsize}{\centering 运行时扩展远比编译时继承威力大
  !}}
  
\end{frame}

\begin{frame}
  \frametitle{Starbuzz咖啡店准备更新订单系统}
  \only<1> {
  \begin{tikzpicture}
    \tikzumlset{fill class=white}
    \umlclass[type=abstract, x=3]{Beverage}{description}{getDescription() \\
      \umlvirt{cost()} \\
      //other method
    }

    \umlclass[x=-1, y=-4]{DarkRoast}{}{cost()}
    \umlclass[x=-4, y=-4]{HouseBlend}{}{cost()}
    \umlclass[x=1.5, y=-4]{Decaf}{}{cost()}
    \umlclass[x=4, y=-4]{Espresso}{}{cost()}

    \umlVHVinherit[weight=0.4]{DarkRoast}{Beverage}
    \umlVHVinherit[weight=0.4]{HouseBlend}{Beverage}
    \umlVHVinherit[weight=0.4]{Decaf}{Beverage}
    \umlVHVinherit[weight=0.4]{Espresso}{Beverage}

    \umlnote[x=-2.5, width=5cm]{Beverage}{
      a. Beverage原先设计为抽象类\\
      b. cost()方法是抽象的,子类必须定义自己的实现\\
      c. description变量由子类设置,用于描述变量,如``Most Excellent Dark
      Roast''
    }
  \end{tikzpicture}
  }

  \only<2> {
    需求:\\[2ex]
    购买咖啡时,可以要求在其中加入各种调料,如Steamed Milk, Soy, Mocha或
    者Whipped Milk。店家会根据加入的调料收取不同的费用。订单必须考虑到这
    些调料部分。 \\[2ex]
    他们的第一个设计 ---
  }

  \only<3> {
    \scalebox{0.6}{
    \begin{tikzpicture}
    \tikzumlset{fill class=white}
    \umlclass[type=abstract]{Beverage}{description}{getDescription() \\
      \umlvirt{cost()} \\
      //other method
    }

    \umlclass[x=-1, y=-4]{DarkRoast}{}{cost()}
    \umlclass[x=-4, y=-4]{HouseBlend}{}{cost()}
    \umlclass[x=1.5, y=-4]{Decaf}{}{cost()}
    \umlclass[x=4, y=-4]{Espresso}{}{cost()}

    \umlinherit{DarkRoast}{Beverage}
    \umlinherit{HouseBlend}{Beverage}
    \umlinherit{Decaf}{Beverage}
    \umlinherit{Espresso}{Beverage}

    \umlclass[x=-5, y=-5]{HouseBlendWithSteamedMilkandMocha}{}{cost()}
    \umlinherit{HouseBlendWithSteamedMilkandMocha}{Beverage}

    \umlclass[x=-4, y=-6]{DecafWithSteamedMilkandMocha}{}{cost()}
    \umlinherit{DecafWithSteamedMilkandMocha}{Beverage}

    \umlclass[x=-3, y=-7]{EspressoWithSteamedMilkandMocha}{}{cost()}
    \umlinherit{EspressoWithSteamedMilkandMocha}{Beverage}

    \umlclass[x=4, y=-5.5]{DarkRoastWithSteamedMilkandCaramel}{}{cost()}
    \umlinherit{DarkRoastWithSteamedMilkandCaramel}{Beverage}

    \umlclass[x=5, y=-6.5]{DecafWithSteamedMilkandCaramel}{}{cost()}
    \umlinherit{DecafWithSteamedMilkandCaramel}{Beverage}

    \umlclass[x=-5, y=-8]{DarkRoastWithSteamedMilkandSoy}{}{cost()}
    \umlinherit{DarkRoastWithSteamedMilkandSoy}{Beverage}

    \umlclass[x=4, y=-8]{DecafWithSteamedMilkandSoy}{}{cost()}
    \umlinherit{DecafWithSteamedMilkandSoy}{Beverage}

    \umlclass[x=-6, y=-9]{HouseBlendWithSteamedMilkandSoy}{}{cost()}
    \umlinherit{HouseBlendWithSteamedMilkandSoy}{Beverage}

    \umlclass[x=-0, y=-9]{EspressoWithWhipandSoy}{}{cost()}
    \umlinherit{EspressoWithWhipandSoy}{Beverage}

    \umlclass[x=5, y=-9.5]{DecafWithSoyandMocha}{}{cost()}
    \umlinherit{DecafWithSoyandMocha}{Beverage}

    \umlclass[x=-3, y=-10]{DecafWithSoyandWhip}{}{cost()}
    \umlinherit{DecafWithSoyandWhip}{Beverage}
    
    \umlclass[x=4, y=-10]{HouseBlendWithWhip}{}{cost()}
    \umlinherit{HouseBlendWithWhip}{Beverage}

    \umlnote[x=-7, width=3cm]{HouseBlendWithSteamedMilkandMocha}{\Large 类爆炸:\\
    每种调料组合都要设计一个类}

  \end{tikzpicture}
    }
  }

  \only<4> {
    这种设计会造成严重的维护问题:
    \begin{itemize}
      \item 如果某种饮料价格变动,应如何处理?
      \item 如果要新增一种调料,应如何处理?
    \end{itemize}

    \vspace*{2ex}
    造成这种困境是因为该设计违背了如下设计原则
    \begin{itemize}
      \item [] \textbf{封装变化}
      \item [] \textbf{多用聚合、少用继承}
    \end{itemize}
  }

\end{frame}

{
\defverbatim{\verbbeverage}{%
  %\renewcommand{\baselinestretch}{1.1}
\begin{Verbatim}[label=Beverage的cost()方法伪码]
  public double cost() {
    float condimentCost = 0.0 ;  
    if (hasMilk())  
      condimentCost += milkCost ; 
    if (hasSoy()) 
      condimentCost += soyCost ; 
    if (hasMocha()) 
      condimentCost += mochaCost ; 
    if (hasWhip()) 
      condimentCost += whipCost ; 
    return condimentCost ;
  }
\end{Verbatim}
}

\defverbatim[colored]{\verbdarkroast}{%
  %\renewcommand{\baselinestretch}{1.1}
\begin{Verbatim}[label=DarkRoast的伪码, commandchars=\\\[\]]
public class DarkRoast extends Beverage {
  public DarkRoast() {
    description = "Most Excellent Dark Roast" ;
  }

  public double cost() {
    return 1.99 + \alert[super.cost()] ;
  }
}
\end{Verbatim}
}
\begin{frame}
  \frametitle{新的设计: 取消调料类}
  \only<1> {
  \scalebox{0.75}{
  \begin{tikzpicture}
    \umlclass[fill=white, type=abstract]{Beverage}{description\\
      milk \\
      soy \\
      mocha \\
      whip }{
        getDescription() \\
        \textcolor{red}{cost()} \\
        hasMilk() \\
        setMilk() \\
        hasSoy() \\
        setSoy() \\
        hasMocha() \\
        setMocha() \\
        hasWhip() \\
        setWhip() \\
      //other methods
      }

      \umlnote[x=7, y=3, anchor1=180, anchor2=60, width=6cm]{Beverage}{\Large 各种调料的布
      尔值}
      \umlnote[x=7, y=0, width=6cm]{Beverage}{\Large Beverage的cost()计算各种调料的价钱}
      \umlnote[x=7, y=-3, anchor1=180, anchor2=-60, width=6cm]{Beverage}{\Large 取得和设置
      调料的布尔值}
  \end{tikzpicture}
}
}

  \only<2> {
  \scalebox{0.75}{
  \begin{tikzpicture}
    \tikzumlset{fill class=white}
    \umlclass[type=abstract]{Beverage}{description\\
      milk \\
      soy \\
      mocha \\
      whip }{
        getDescription() \\
        \textcolor{red}{cost()} \\
        hasMilk() \\
        setMilk() \\
        hasSoy() \\
        setSoy() \\
        hasMocha() \\
        setMocha() \\
        hasWhip() \\
        setWhip() \\
      //other methods
      }

      \umlclass[x=5,y=4, width=3cm]{HouseBlend}{}{cost()}
      \umlclass[x=5,y=1.5, width=3cm]{DarkRoast}{}{cost()}
      \umlclass[x=5,y=-1, width=3cm]{Decaf}{}{cost()}
      \umlclass[x=5,y=-3.5, width=3cm]{Espresso}{}{cost()}

      \umlinherit{HouseBlend}{Beverage}
      \umlinherit{DarkRoast}{Beverage}
      \umlinherit{Decaf}{Beverage}
      \umlinherit{Espresso}{Beverage}

      \umlnote[x=10]{DarkRoast}{\normalsize 子类的cost()方法需要计算该饮料的价钱,然
      后再调用父类的cost()方法计算调料价钱,求和}

  \end{tikzpicture}
}
  }
  
  \only<3> {
    \verbbeverage
  }
  \only<4> {
    \verbdarkroast
  }

  \only<5> {
    该设计相比最初的方案有很大改善,但是:
    \begin{itemize}
      \item 调料价钱的改变会引发对代码的改动
      \item 一旦出现新的调料,就需要加上新的方法,并改变父类Beverage的
        cost()方法
      \item 将来可能会有新的饮料出现,如iced tea,某些调料就可能不适合该
        饮料,那么继承而来的hasWhip()等方法可能就不适合
      \item 顾客可能需要双倍/三倍mocha
    \end{itemize}

    \vspace*{2ex}
    \framebox{\parbox{0.9\hsize}{\centering 仍然需要改进设计!}}
  } 
\end{frame}
}

\begin{frame}
  \frametitle{开放-关闭设计原则}
  \begin{block}{设计原则:开放-关闭}
    类应该对 \uline{扩展} 开放,对 \uline{修改} 关闭
  \end{block}

  目标是允许``类容易扩展'',在\uline{不修改现有类代码}的情况下,就可以让
  类具有\uwave{新的行为} \\[2ex]

  这种设计的好处是具有\uwave{弹性},能够灵活应对改变的需求 \\[2ex]

  \textcolor{blue}{但:}没必要把系统的每部分都设计为可扩展的
\end{frame}

\begin{frame}
  \frametitle{认识装饰者模式(decorator)}
  采用``继承''的设计为Starbuzz带来\uline{类爆炸}、\uline{设计死板}、\uline{
  父类新功能无法适应所有子类}等问题,我们转而采用``装饰''的设计方法:
  \begin{enumerate}
    \item 拿一个DarkRoast对象
    \item 用Mocha对象装饰它
    \item 用Whip对装饰它
    \item 用任何调料装饰它
    \item 装饰对象的cost()方法: 该方法先调用被装饰对象的
      cost()方法获得被装饰对象的价钱,加上装饰对象的调料价格求和
  \end{enumerate}
\end{frame}

\begin{frame}
  \frametitle{以装饰者模式构造新订单系统}

  \only<1> {
  \centering\begin{tikzpicture}
    \tikzumlset{fill class=white}

    \umlclass[type=abstract, y=3]{Beverage} {}{cost()}
    \umlclass{DarkRoast} {}{cost()}
    \umlinherit{DarkRoast}{Beverage}
  \end{tikzpicture}
  }

  \only<2> {
  \centering\begin{tikzpicture}
    \tikzumlset{fill class=white}

    \umlclass[type=abstract, y=3]{Beverage} {}{cost()}
    \umlclass{DarkRoast} {}{cost()}
    \umlinherit{DarkRoast}{Beverage}

    \umlclass[x=-4]{Mocha}{Beverage beverage}{cost()}
    \umluniaggreg{Mocha}{DarkRoast}
    \umlinherit{Mocha}{Beverage}
  \end{tikzpicture}
  }

  \only<3> {
  \centering\begin{tikzpicture}
    \tikzumlset{fill class=white}

    \umlclass[type=abstract, y=3]{Beverage} {}{cost()}
    \umlclass{DarkRoast} {}{cost()}
    \umlinherit{DarkRoast}{Beverage}

    \umlclass[x=-4]{Mocha}{Beverage beverage}{cost()}
    \umluniaggreg{Mocha}{DarkRoast}
    \umlinherit[anchor1=90]{Mocha}{Beverage}

    \umlclass[x=-8.5]{Whip}{Beverage beverage}{cost()}
    \umluniaggreg{Whip}{Mocha}
    \umlinherit[anchor1=90]{Whip}{Beverage}
  \end{tikzpicture}
  }

  \only<4> {
  \centering\begin{tikzpicture}
    \tikzumlset{fill class=white}

    \umlclass[type=abstract, y=3]{Beverage} {}{cost()}
    \umlclass{DarkRoast} {}{cost()}
    \umlinherit{DarkRoast}{Beverage}

    \umlclass[x=-4]{Mocha}{Beverage beverage}{cost()}
    \umluniaggreg{Mocha}{DarkRoast}
    \umlinherit[anchor1=90]{Mocha}{Beverage}

    \umlclass[x=-8.5]{Whip}{Beverage beverage}{cost()}
    \umluniaggreg{Whip}{Mocha}
    \umlinherit[anchor1=90]{Whip}{Beverage}

    \node [minimum width=1cm] (totalcost) at (-9.5, 3.0) {总价格};
    \node [minimum width=1cm] (wcost) at (-9.5, -0.8) {};
    \node [minimum width=1cm] (mcost) at (-5, -0.8) {};
    \node [minimum width=1cm] (dcost) at (-0.5, -0.6) {};

    \draw [->, dashed, thick, red] (totalcost) to (wcost) ;
    \draw [->, dashed, thick, red] (wcost.east) to (mcost.west) ;
    \draw [->, dashed, thick, red] (mcost.east) to (dcost.west) ;

    \draw [->, dashed, thick, red, bend left=30] ([yshift=-0.2cm]dcost.west) to node[below]{\$0.99} ([yshift=-0.2cm]mcost.east) ;

    \draw [->, dashed, thick, red, bend left=30] ([yshift=-0.2cm]mcost.west) to node[below]{\$0.99 + \$0.20} ([yshift=-0.2cm]wcost.east) ;

    \draw [->, dashed, thick, red, bend right=30] ([yshift=+0.2cm]wcost.east) to ([yshift=-0.2cm]totalcost.east) node[xshift=+2cm]{\$0.99 + \$0.20 +\$0.10} ;

  \end{tikzpicture}
  }


  \only<5> {
    从前面的设计可以得知:
    \begin{itemize}
      \item 装饰者和被装饰者对象拥有一个\uline{相同的父类}
      \item 可以用\uline{多个}装饰者对象装饰一个被装饰者对象
      \item 由于装饰者和被装饰者的父类型相同,因此可以在任何需要原始对象
        的场合,用装饰过的对象\uline{取代}它
      \item 装饰者可以在被装饰者的行为之前/之后,加上自己的\uline{特定行
        为}
      \item 对象可以在任何时候被装饰,因此可以\uline{在运行时}动态地、不受限制地
        进行装饰
    \end{itemize}
  }

\end{frame}

\begin{frame}
  \frametitle{定义装饰者模式}

  \only<1> {
  \begin{block}{装饰者模式(decorator pattern)}
    装饰者模式动态地将责任附加到对象上。\\
    在扩展功能方面,装饰者提供
    了比继承\uline{更有弹性}的替代方案
  \end{block}
}

  \only<2> {
    \scalebox{0.9}{
    \begin{tikzpicture}
      \tikzumlset{fill class=white}

      \umlclass[type=abstract]{Component}{}{methodA()}
      \umlclass[x=-2, y=-3]{ConcreteComponent}{}{methodA()}
      \umlclass[x=2, y=-3, type=abstract]{Decorator}{}{methodA()}
      \umlclass[y=-6]{ConcreteDecoratorA}{Component wrapped}{methodA()}
      \umlclass[y=-6, x=5]{ConcreteDecoratorB}{Component wrapped}{methodA()}

      \umlinherit{ConcreteComponent}{Component}
      \umlinherit{Decorator}{Component}
      \umluniaggreg[geometry=|-]{Decorator}{Component}
      \umlinherit{ConcreteDecoratorA}{Decorator}
      \umlinherit{ConcreteDecoratorB}{Decorator}

      \umlnote[x=5, y=-0.3, width=4cm] {Decorator} {
        每个装饰者都有一个实例变量保存被装饰的 Component 对象
      }
    \end{tikzpicture}
  }
  }
\end{frame}

\begin{frame}
  \frametitle{用装饰者模式设计Starbuzz饮料系统}
  \noindent\scalebox{0.8}{\begin{tikzpicture}
    \tikzumlset{fill class=white}
    \umlclass[type=abstract, x=-3]{Beverage}{description}{getDescription() \\
      \umlvirt{cost()}
    }

    \umlclass[x=-4.5, y=-3.5] {Houseblend}{}{cost()}
    \umlclass[x=-1.5, y=-3.5] {DarkRoast}{}{cost()}
    \umlclass[x=-4, y=-6] {Espresso}{}{cost()}
    \umlclass[x=-1, y=-6] {Decaf}{}{cost()}

    \umlinherit{Houseblend}{Beverage}
    \umlinherit{DarkRoast}{Beverage}
    \umlinherit{Espresso}{Beverage}
    \umlinherit{Decaf}{Beverage}

    \umlclass[type=abstract, fill=yellow!60, x=4]{CondimentDecorator}{}{\umlvirt{getDescription()}} 
    \umlinherit{CondimentDecorator}{Beverage}
    \umluniaggreg[anchor1=160, anchor2=35, arg=装饰, pos=0.5]{CondimentDecorator}{Beverage}

    \umlclass[x=2, y=-3] {Milk}{Beverage beverage}{cost() \\ getDescription()}
    \umlclass[x=6, y=-3] {Mocha}{Beverage beverage}{cost() \\ getDescription()}
    \umlclass[x=3, y=-6.2] {Soy}{Beverage beverage}{cost() \\ getDescription()}
    \umlclass[x=7, y=-6.2] {Whip}{Beverage beverage}{cost() \\ getDescription()}

    \umlinherit{Milk}{CondimentDecorator}
    \umlinherit{Mocha}{CondimentDecorator}
    \umlinherit{Soy}{CondimentDecorator}
    \umlinherit{Whip}{CondimentDecorator}
  \end{tikzpicture}
  }
\end{frame}

\begin{frame}
  \frametitle{装饰者模式中的继承和聚合}
  \begin{overprint}
  \onslide<1-> {
    \textcolor{red}{Q:} 装饰者模式为什么仍然用到了继承? \\
  }
  \onslide<2-> {
    \textcolor{blue}{A:} 这里的继承是使装饰者和被装饰者具有\uline{一样的
    类型}, 而不是利用继承获得父类的\uwave{``行为''} \\[1ex]
  }

  \onslide<3-> {
    \textcolor{red}{Q:} 那\uwave{行为}从何而来? \\
  }
  \onslide<4-> {
    \textcolor{blue}{A:} 当将装饰者和被装饰者\uwave{聚合}在一起时,就是在增加新
    的行为。新行为非来自继承,而是来自\uwave{聚合} \\[1ex]
  }
  \onslide<5-> {
    \textcolor{red}{Q:} 这里的聚合相比继承的好处是什么? \\
  }
  \onslide<6-> {
    \textcolor{blue}{A:} 若依赖继承,那么类的行为就在编译时刻决定了。
    若使用聚合,即可在\textbf{\uline{运行时}}动态增加新行为,而不
    必修改已有代码 \\[2ex]
  }
  \onslide<7-> {
    \framebox{\parbox{0.9\hsize}{\centering \textbf{对扩展开放,对修改封
    闭}}}
  }
  \end{overprint}
\end{frame}

{
\defverbatim{\verbbeverage}{%
  %\renewcommand{\baselinestretch}{1.1}
\begin{Verbatim}[label=被装饰者抽象类]
public abstract class Beverage {
  String description = "Unknown Beverage";
  public String getDescription() { 
    return description;
  }
  public abstract double cost(); 
}
\end{Verbatim}
}

\defverbatim{\verbcondiment}{%
  %\renewcommand{\baselinestretch}{1.1}
\begin{Verbatim}[label=装饰者抽象类]
public abstract class CondimentDecorator
  extends Beverage {
  public abstract String getDescription(); 
}
\end{Verbatim}
}

\defverbatim{\verbespresso}{%
  %\renewcommand{\baselinestretch}{1.1}
\begin{Verbatim}[label=具体的被装饰者-饮料子类]
public class Espresso extends Beverage {
  public Espresso() { 
    description = "Espresso";
  }

  public double cost() { 
    return 1.99;
  } 
}
\end{Verbatim}
}

\defverbatim{\verbmocha}{%
  %\renewcommand{\baselinestretch}{1.1}
\begin{Verbatim}[label=具体的装饰者-调料子类]
public class Mocha extends CondimentDecorator { 
  Beverage beverage;
  public Mocha(Beverage beverage) { 
    this.beverage = beverage;
  }
  public String getDescription() {
    return beverage.getDescription() + ", Mocha";
  }
  public double cost() {
    return .20 + beverage.cost();
  }
}
\end{Verbatim}
}

\defverbatim{\verbtest}{%
  %\renewcommand{\baselinestretch}{1.1}
\begin{Verbatim}[label=测试类]
public class StarbuzzCoffee {
  public static void main(String args[]) { 
    Beverage beverage = new Espresso(); 
    System.out.println(beverage.getDescription() + " $" + 
      beverage.cost()); 
    Beverage beverage2 = new DarkRoast();
    beverage2 = new Mocha(beverage2); 
    beverage2 = new Mocha(beverage2); 
    beverage2 = new Whip(beverage2);
    System.out.println(beverage2.getDescription() + 
      " $" + beverage2.cost());
  } 
}
\end{Verbatim}
}

\defverbatim{\verboutput}{%
  %\renewcommand{\baselinestretch}{1.1}
\begin{Verbatim}[label=测试输出]
% java StarbuzzCoffee
Espresso $1.99
Dark Roast Coffee, Mocha, Mocha, Whip $1.49
%
\end{Verbatim}
}


  \begin{frame}
    \frametitle{使用了装饰者模式的Starbuzz新系统伪代码}
    \only<1> {
      \begin{columns}

        \column{0.5\hsize}

      \qquad\begin{tabular}{|lc|}
        \hline
        \multicolumn{2}{|c|}{\Large \textbf{Starbuzz Coffees}} \\
        \multicolumn{2}{|l|}{\textbf{\uline{Coffees}}} \\
        House Blend & 0.89 \\ 
        Dark Roast & 0.99 \\ 
        Decaf & 1.05 \\
        Espresso & 1.99 \\
        \multicolumn{2}{|l|}{\textbf{\uline{Condiments}}} \\
        SteamedMilk & 0.10 \\
        Mocha & 0.20 \\
        Soy & 0.15 \\
        Whip & 0.10 \\
        \hline
      \end{tabular}

      \column{0.5\hsize}

      \noindent 根据Starbuzz的需求,实现各种咖啡和调料的具体类
    \end{columns}
    }


    \only<2> {
      \verbbeverage
      \verbcondiment
    }
    \only<3> {
      \verbespresso
    }
    \only<4> {
      \verbmocha
    }
    \only<5> {
      \verbtest
    }
    \only<6> {
      \verboutput
    }
  \end{frame}
}

\begin{frame}
  \frametitle{讨论}
  \begin{overprint}
    \onslide<1-> {
      \textcolor{blue}{问题}:  如果咖啡店会举行一些活动,比如特定型号的咖啡打折,上述设计
      是否适当? \\[3ex]
    }
    \onslide<2-> {
      \textcolor{blue}{问题}:  装饰者知道其他装饰者的存在吗,如果不能,如何让他们知道? \\[3ex]
    }
    \onslide<3-> {
      \textcolor{blue}{问题}:  咖啡店决定区分容量大小,大杯、中杯、小杯的价格不同,应该如
      何调整设计? \\[3ex]
    }
    \onslide<4-> {
      \textcolor{blue}{问题}:  装饰者模式会引入很多\uline{小类},是否增
      加了系统复杂度?
    }
    \end{overprint} 
\end{frame}

\begin{frame}
  \frametitle{小结}
  \only<1> {
    \textcolor{blue}{关于装饰者模式}: 
    \begin{itemize}
      \item 继承可以扩展对象行为,但不是达到\uwave{弹性设计}的最佳方式
      \item 设计应允许扩展现有行为,而无须修改现有代码
      \item 聚合和委托可用于在运行时动态增加新行为
      \item 装饰者模式允许弹性地扩展现有行为
      \item 装饰者模式意味着很多装饰者类
      \item 可以用任意多个装饰者包装一个构件
      \item 过度使用装饰者模式会使程序变得复杂
  \end{itemize}
  }
  \only<2> {
    \textcolor{blue}{关于设计原则}: 
    \begin{enumerate}
      \item 封装变化
      \item 多用聚合、少用继承
      \item 针对接口编程,不针对实现编程
      \item 尽最大可能将要交互的对象设计为松耦合的
      \item 对扩展开放,对修改封闭
    \end{enumerate}
  }
\end{frame}

\end{document}
